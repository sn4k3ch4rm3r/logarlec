%---------------------------------------------------------------
%\\\\\\\\\\\\\\\\\\\\\\\\NE MÓDOSÍTSD///////////////////////////

\newenvironment{naplo}{\begin{longtable}{|l|l|p{2.5cm}|p{7.5cm}|}
        \hline
        \hline \multicolumn{1}{|c|}{\textbf{Kezdet}} & \multicolumn{1}{c|}{\textbf{Időtartam}} & \multicolumn{1}{c|}{\textbf{Résztvevők}} & \multicolumn{1}{c|}{\textbf{Leírás}}  \\ \hline 
        \endfirsthead
        
        \multicolumn{3}{c}%
        {{\bfseries}} \\
        \hline \multicolumn{1}{|c|}{\textbf{Kezdet}} & \multicolumn{1}{c|}{\textbf{Időtartam}} & \multicolumn{1}{c|}{\textbf{Résztvevők}} & \multicolumn{1}{c|}{\textbf{Leírás}}  \\ \hline 
        \endhead
        
}{\end{longtable}}

\newcommand{\naplotag}[4]{#1 & #2 & #3 & #4 \\
    \hline}

\newtheorem{defi}{Definíció}
\newtheorem{rov}{Rövidítés}
\newcommand{\urlref}[2]{
	\parbox{\linewidth}{\href{#1}{\RIGHTarrow \quad #2 - \texttt{#1}}}}

\newcommand{\diagram}[3]{
	\begin{figure}[H]
		\centering
		\includegraphics[width=#3]{#1}
		\caption{#2}
	\end{figure}}

\renewcommand{\comment}[1]{\begin{flushright}
		\color{red} 
		#1
\end{flushright} }

\newenvironment{class-template}[5]
{ 
    \hspace*{0.4em} $\blacksquare $ \large Felelősség \\  \normalsize #1 \\[12px]
    \hspace*{0.4em} $\blacksquare $ \large Ősosztályok \\  \normalsize #2 \\[12px]
    \hspace*{0.4em} $\blacksquare $ \large Interfészek \\  \normalsize #3 \\[12px]
    \hspace*{0.4em} $\blacksquare $ \large Attribútumok \\  \normalsize #4 \\[12px]
    \hspace*{0.4em} $\blacksquare $ \large Metódusok \\  \normalsize #5 \\[12px]
}{}

\newenvironment{class-template-responsibility}{\hspace*{0.4em} $\blacksquare $ \large Felelősség \\[5px]  \normalsize}{\\[12px]}
\newenvironment{class-template-baseclass}{\hspace*{0.4em} $\blacksquare $ \large Ősosztály \\[5px]  \normalsize}{\\[12px]}
\newenvironment{class-template-interface}{\hspace*{0.4em} $\blacksquare $ \large Interfészek \\[5px]  \normalsize}{\\[12px]}
\newenvironment{class-template-method}{\hspace*{0.4em} $\blacksquare $ \large Metódusok \\[5px]  \normalsize}{\\[12px]}
\newenvironment{class-template-attribute}{\hspace*{0.4em} $\blacksquare $ \large Attribútumok \\[5px]  \normalsize}{\\[12px]}
\newenvironment{class-template-statechart}{\hspace*{0.4em} $\blacksquare$ \large Állapotgép \\[5px] \normalsize}{\\[12px]}

\newenvironment{enum-template}[2]
{ 
    \hspace*{0.4em} $\blacksquare $ \large Felelősség \\[5px]  \normalsize #1 \\[12px]
    \hspace*{0.4em} $\blacksquare $ \large Literálok \\[5px]  \normalsize #2 \\[12px]
}{}	

{\vspace*{-1cm}}
\newcommand{\classitem}[2]{
    \hspace*{25px}\parbox{0.9\linewidth}{$\Diamond $ \textit{#1} - #2}
}

\newcommand{\baseclass}{ $\rightarrow$ }

\newenvironment{funkovetelmeny}[6]
{
    \noindent\begin{tabular}{| p{2cm} | p{2cm} | p{2.2cm} | p{5.3cm} | p{2.5cm} |}
        \hline
        \hline
        \textbf{Azonosító} & \textbf{Prioritás} & \textbf{Forrás} & \textbf{Use-case} & \textbf{Ellenőrzés} \\
        \hline
        #1 & #2 & #3 & #4 & #5 \\
        \hline
    \end{tabular}
    \\[1ex]
    \textbf{Leírás:} #6 \\[0.2ex]
    \itshape}
{\\[5px]}

\newenvironment{kovetelmeny}[5]
{ 
    \noindent\begin{tabular}{| p{2cm} | p{3.5cm} | p{3.5cm} | p{3.5cm} |}
        \hline
        \hline
        \textbf{Azonosító} & \textbf{Prioritás} & \textbf{Forrás} & \textbf{Ellenőrzés} \\
        \hline
        #1 & #2 & #3 & #4 \\
        \hline
    \end{tabular}
    \\[1ex]
    \textbf{Leírás:} #5 \\[0.2ex]
    \itshape}
{\\[5px]}

\newenvironment{use-case}[3]
{ 
    
    \vspace{10px}
    \noindent\begin{tabular}{| p{3cm} | p{12cm} |}
        \hline
        \hline
        \vspace{0.2ex} \textsc{\textbf{Use-case neve:}} & \vspace{0.2ex} \textsc{#1} \\[0.2ex]
        \hline
        \textbf{Rövid leírás:} & #2 \\
        \hline
        \textbf{Aktorok:} & #3 \\
        \hline
        \textbf{Forgatókönyv:} &  
    }
    {	\\
        \hline
\end{tabular} }

\newenvironment{fajllista}{
    \begin{longtable}{|p{4.5cm}|l|l| p{4.5cm} |}
        \hline
        \hline \multicolumn{1}{|c|}{\textbf{Fájl neve}} & \multicolumn{1}{c|}{\textbf{Méret}} & \multicolumn{1}{c|}{\textbf{Keletkezés ideje}} & \multicolumn{1}{c|}{\textbf{Tartalom}}  \\ \hline 
        \endfirsthead
        
        \multicolumn{3}{c}%
        {{\bfseries}} \\
        \hline \multicolumn{1}{|c|}{\textbf{Fájl neve}} & \multicolumn{1}{c|}{\textbf{Méret}} & \multicolumn{1}{c|}{\textbf{Keletkezés ideje}} & \multicolumn{1}{c|}{\textbf{Tartalom}}  \\ \hline 
        \endhead
        
    }{\end{longtable}}

\newcommand{\fajl}[4] {
    #1 & #2 & #3 & #4 \\ \hline
}

\newenvironment{ertekeles}{
    \begin{figure}[h]
        \centering
        \noindent\begin{tabular}{| p{5cm} | p{4cm} | p{4cm} |}
            \hline
            \hline
            \textbf{Tag neve} & \textbf{Tag neptun} & \textbf{Munka százalékban} \\	\hline
}{\end{tabular}\end{figure}}

\newcommand{\ertekelestag}[3]{ 
    #1 & #2 & #3 \\ \hline
}

\newcommand{\ertekelestagk}[2]{ 
    #1 & #2 \\ \hline
}

\newenvironment{ertekelesOra}{
    \begin{figure}[h]
        \centering
        \noindent\begin{tabular}{| p{5cm} | p{4cm} | p{4cm} |}
            \hline
            \hline
            \textbf{Tag neve} & \textbf{Tag neptun} & \textbf{Munka órában} \\	\hline
            
        }{\end{tabular}	\end{figure}}

\newenvironment{ertekelesKod}{
    \begin{figure}[h]
        \centering
        \noindent\begin{tabular}{| p{5cm} | p{4cm} |}
            \hline
            \hline
            \textbf{Fázis} & \textbf{Kódsorok száma} \\	\hline
            
        }{\end{tabular}	\end{figure}}

\newcommand{\test}[3]{
    \vspace{10px}
    \noindent\begin{tabular}{| p{3.5cm} | p{12cm} |}
        \hline
        \hline
        \vspace{0.2ex} \textsc{\textbf{Teszt-eset neve}} & \vspace{0.2ex} \textsc{#1} \\[0.2ex]
        \hline
        \textbf{Rövid leírás} & #2 \\
        \hline
        \textbf{Teszt célja} & #3 \\
        \hline
    \end{tabular}
}

\newenvironment{test-case}[4]
{ 
    \hspace*{0.4em} $\blacksquare $ \large Leírás   \\  \normalsize #1   \\[12px]
    \hspace*{0.4em} $\blacksquare $ \large Ellenőrzött funkcionalitás, várható hibahelyek  \\  \normalsize #2  \\[12px]
    \hspace*{0.4em} $\blacksquare $ \large Bemenet  \\  \normalsize #3  \\[12px]
    \hspace*{0.4em} $\blacksquare $ \large Elvárt kimenet \\  \normalsize #4 \\[12px]
}{}

\newenvironment{test-case-description}{\hspace*{0.4em} $\blacksquare $ \large Leírás \\[5px]  \normalsize}{\\[12px]}
\newenvironment{test-case-function}{\hspace*{0.4em} $\blacksquare $ \large Ellenőrzött funkcionalitás, várható hibahelyek \\[5px]  \normalsize}{\\[12px]}
\newenvironment{test-case-input}{\hspace*{0.4em}\parbox{\linewidth}{$\blacksquare $ \large Bemenet\vspace*{5px}}\normalsize}{\leavevmode\newline}
\newenvironment{test-case-output}{\hspace*{0.4em}\parbox{\linewidth}{$\blacksquare $ \large Elvárt kimenet\vspace{5px}}\normalsize}{\leavevmode\newline}

\newcommand{\testOK}[2]{
    \vspace{10px}
    \noindent\begin{tabular}{| p{4.2cm} | p{12cm} |}
        \hline
        \hline
        \textbf{Tesztelő neve} & \textsc{#1} \\[0.2ex]
        \hline
        \textbf{Teszt időpontja} & #2 \\
        \hline
    \end{tabular}
}

\newcommand{\testFAIL}[5]{
    \vspace{10px}
    \noindent\begin{tabular}{| p{4.2cm} | p{12cm} |}
        \hline
        \hline
        \textbf{Tesztelő neve} & \textsc{#1} \\[0.2ex]
        \hline
        \textbf{Teszt időpontja} & #2 \\
        \hline
        \textbf{Teszt eredménye} & #3 \\
        \hline
        \textbf{Lehetséges hibaokok} & #4 \\
        \hline
        \textbf{Változtatások} & #5 \\
        \hline
    \end{tabular}
}

\newenvironment{terv}{	
    \begin{longtable}{| l | p{8cm} | l |}
        \hline
        \hline
        \multicolumn{1}{|c|}{\textbf{Határidő}} & \multicolumn{1}{c|}{\textbf{Feladat}} & \multicolumn{1}{c|}{\textbf{Felelős}}  \\ \hline 
        \endfirsthead
        
        \multicolumn{3}{c}%
        {{\bfseries}} \\
        \hline \multicolumn{1}{|c|}{\textbf{Határidő}} & \multicolumn{1}{c|}{\textbf{Feladat}} & \multicolumn{1}{c|}{\textbf{Felelős}}  \\ \hline 
        \endhead
        
}{\end{longtable}}

\newcommand{\tervitem}[3]{ #1 & #2 & #3 \\ \hline}

\newenvironment{szotar}{
    \begin{longtable}{ p{4cm}  p{11cm} }
        
}{\end{longtable}}

\newcommand{\szotaritem}[2]{ \textbf{#1} & #2 \\[10px]}