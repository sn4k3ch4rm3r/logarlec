\chapter{Prototípus koncepciója}
% \comment{A prototípus program célja annak demonstrálása, hogy a program elkészült, helyesen működik, valamennyi feladatát teljesíti. A prototípus változat egy elkészült program kivéve a kifejlett grafikus interfészt. Ez a program is parancssorból futtatható és karakteres ernyőkezelést alkalmaz. Az ütemezés, az aktív objektumok kezelése megoldott. A business objektumok -a megjelenítésre vonatkozó részeket kivéve -valamennyi metódusa a végleges algoritmusokat kell, hogy tartalmazza. A megjelenítés és működtetés egy alfanumerikus képernyőn vezérelhető és követhető, ugyanakkor a vezérlés fájlból is történhet és a megjelenítés fájlba is logolható, ezzel megteremtve a rendszer tesztelésének lehetőségét. Különös figyelmet kell fordítani a parancssori interfész logikájára, felépítésére, valamint arra, hogy az mennyiben tükrözi és teszi láthatóvá a program működését, a beavatkozások hatásait.}
% \comment{\hrule}
% \comment{Amennyiben változott a modell:}
\setcounter{section}{-1}
\section{Változás hatása a modellre}
\subsection{Módosult osztálydiagram}
% \comment{Az analízis modell osztálydiagramja a változások figyelembevételével.}

\diagram{img/diagrams/class/effects}{}{14cm}
\diagram{img/diagrams/class/game-objects}{}{18cm}
\diagram{img/diagrams/class/items}{}{14cm}

\clearpage

\subsection{Új vagy megváltozott metódusok}
% \comment{Az analízis modell osztályleírásaiból azon metódusok újbóli felsorolása leírással együtt, amelyek a változtatás miatt módosultak vagy újonnan be lettek vezetve}
\textbf{A bevezetett új osztályok}
\subsubsection{AirFreshener}
\begin{class-template-responsibility}
    A játékbeli tárgy reprezentálása a modellben.
\end{class-template-responsibility}
\begin{class-template-baseclass}
    Item
\end{class-template-baseclass}
\begin{class-template-method}
    -
\end{class-template-method}

\subsubsection{CleanEffect}
\begin{class-template-responsibility}
    A játékbeli effektus reprezentálása a modellben.
\end{class-template-responsibility}
\begin{class-template-baseclass}
    Effect
\end{class-template-baseclass}
\begin{class-template-method}
    -
\end{class-template-method}

\subsubsection{FakeCodeOfStudies}
\begin{class-template-responsibility}
    A játékbeli tárgy reprezentálása a modellben.
\end{class-template-responsibility}
\begin{class-template-baseclass}
    Item \baseclass CodeOfStudies
\end{class-template-baseclass}
\begin{class-template-method}
    -
\end{class-template-method}

\subsubsection{FakeMask}
\begin{class-template-responsibility}
    A játékbeli tárgy reprezentálása a modellben.
\end{class-template-responsibility}
\begin{class-template-baseclass}
    Item \baseclass Mask
\end{class-template-baseclass}
\begin{class-template-method}
    -
\end{class-template-method}

\subsubsection{FakeSlideRule}
\begin{class-template-responsibility}
    A játékbeli tárgy reprezentálása a modellben.
\end{class-template-responsibility}
\begin{class-template-baseclass}
    Item \baseclass SlideRule
\end{class-template-baseclass}
\begin{class-template-method}
    -
\end{class-template-method}

\subsubsection{Janitor}
\begin{class-template-responsibility}
    A játékbeli tárgy reprezentálása a modellben.
\end{class-template-responsibility}
\begin{class-template-baseclass}
    GameObject \baseclass Person
\end{class-template-baseclass}
\begin{class-template-method}
    -   
\end{class-template-method}

\subsubsection{JanitorEffect}
\begin{class-template-responsibility}
    A játékbeli effektus reprezentálása a modellben.
\end{class-template-responsibility}
\begin{class-template-baseclass}
    Effect \baseclass CleanEffect
\end{class-template-baseclass}
\begin{class-template-method}
    -
\end{class-template-method}

\textbf{A korábbi osztályokban történt módosítások}

\subsection{Effect}
\begin{class-template-responsibility}
    Ősosztály biztosítása a játékban található effektusoknak.
\end{class-template-responsibility}
\begin{class-template-interface}
    Updatable
\end{class-template-interface}
\begin{class-template-baseclass}
    -
\end{class-template-baseclass}
\begin{class-template-attribute}
    \classitem{- timeRemaining : double}{A hatásból hátralévő idő.}
\end{class-template-attribute}
\begin{class-template-method}
    \classitem{applyToStudent(target : Student) : void}{Az effektus rátétele egy hallgatóra.}
    \classitem{applyToTeacher(target : Teacher) : void}{Az effektus rátétele egy oktatóra.}
    \classitem{applyToRoom(room : Room) : void}{Az effektus rátétele egy szobára.}
    \classitem{interactCleanEffect(cleanEffect : CleanEffect) : void}{Az effektus interakciója egy CleanEffecttel.}
\end{class-template-method}

\subsection{Inventory}
\begin{class-template-responsibility}
    Tárgyak tárolását teszi lehetővé Person objektumok számára.
\end{class-template-responsibility}
\begin{class-template-interface}
    Updatable
\end{class-template-interface}
\begin{class-template-baseclass}
    -
\end{class-template-baseclass}
\begin{class-template-attribute}
    \classitem{- itemCount: int}{Hány tárgy van benne.}
\end{class-template-attribute}
\begin{class-template-method}
    \classitem{add(item : Item) : bool}{Tárgy hozzáadása az inventoryhoz.}
    \classitem{remove(item : Item) : void}{Tárgy eltávolítása az inventoryból.}
    \classitem{setRoom(room : Room) : void}{Az összes tárgy értesítése a szoba váltásról.}
    \classitem{protectFrom(teacher : Teacher) : void}{Az inventory-ban tárolt tárgyakkal megpróbál védekezni oktató ellen.}
    \classitem{dropRandomItem() : void}{Az inventory eldobja egy véletlenszerűen választott tárgyát.}
\end{class-template-method}

\subsection{Person}
\begin{class-template-responsibility}
    Alaposztály szolgáltatása a mozgó játékbeli entitások számára.
\end{class-template-responsibility}
\begin{class-template-interface}
    -
\end{class-template-interface}
\begin{class-template-baseclass}
    GameObject
\end{class-template-baseclass}
\begin{class-template-attribute}
    \classitem{- name : String}{Az ember neve.}
    \classitem{- knockOutTime : double}{Az elájulás időtartama.}
\end{class-template-attribute}
\begin{class-template-method}
    \classitem{protectFromTeacher(target : Teacher) : void}{Egy személyt valami megment egy oktatótól.}
    \classitem{enterRoom(Room room) : void}{Az ember belép egy szobába.}
    \classitem{dropItem(item : Item) : void}{Az ember ledobja egy tárgyát a földre.}
    \classitem{setKnockOut() : void}{Az elájulás időtartamának beállítása.}
    \classitem{getOut() : void}{Az embert kidobják a szobából, át kell mennie egy szomszédosba.}
    \classitem{dropRandomItem() : void}{Az ember eldobja egy véletlenszerűen választott tárgyát.}
\end{class-template-method}

\subsection{Room}
\begin{class-template-responsibility}
    Ő birtokolja a benne lévő Door, Person és (nem Personnál lévő) Item objektumokat.
    Ő engedélyezi vagy tagadja meg a Personok belépését.
    Ő értesíti a benne lévő Personokat új Person belépéséről, mérges gázról, táblatörlő rongy használatáról. 
\end{class-template-responsibility}
\begin{class-template-interface}
    -
\end{class-template-interface}
\begin{class-template-baseclass}
    GameObject
\end{class-template-baseclass}
\begin{class-template-attribute}
    \classitem{- capacity : int}{Megadja, hogy maximum hány ember fér el a szobában.}
    \classitem{- numOfPeople : int}{A szobába jelenleg tartózkodó emberek száma.}
    \classitem{- visitorsSinceClean : int}{A szoba kitakarítása óta hány ember lépett be a szobába.}
\end{class-template-attribute}
\begin{class-template-method}
    \classitem{enter(Person person) : bool} { A személy belép a szobába és a többi személy értesül róla, és reagál, ha kell.}
    \classitem{leave(Person person) : void} { A szobát elhagyna egy személy, a szobában lévő emberek száma csökken.}
    \classitem{merge(Room room) : void} { Két szoba összeolvad.}
    \classitem{split() : void} {A szoba a szabályoknak megfelelően osztódik.}
    \classitem{moveContents(Room room) : void} {A szobában található dolgokat átrakja egy másikba.}
    \classitem{addDoor(Room room) : void} {Ajtó adódik a szobához.}
    \classitem{removeDoor(Room room) : void} {Ajtó eltávolítása a szobából.}
    \classitem{hideDoors() : void} {Az eltűnő ajtók eltűnnek.}
    \classitem{showDoors() : void} {Az eltűnt ajtók megjelennek.}
    \classitem{interactCleanEffect(CleanEffect cleanEffect) : void}{A szoba interakciója egy cleanEffecttel, megszünteti a gázt.}
    \classitem{getOut(Person person) : void}{Áttesz egy személyt egy szomszédos szobába.}
    \classitem{clean() : void}{Megszünteti a szoba ragadósságát.}
    \classitem{isClean() : bool}{Visszaadja, hogy ragadós-e a szoba.}
\end{class-template-method}

\subsection{Szekvencia-diagramok}
% \comment{Az analízis modell szekvenciadiagramjaiból a változás által érintett, előírt, módosítottdiagramok}

\diagram{img/diagrams/sequence/person-add-item-new}{Megváltozott tárgyfelvétel}{14cm}
\diagram{img/diagrams/sequence/effect-clean-apply}{}{14cm}
\diagram{img/diagrams/sequence/effect-janitor-apply-room}{}{14cm}
\diagram{img/diagrams/sequence/effect-janitor-apply}{}{14cm}
\diagram{img/diagrams/sequence/item-air-freshener-use}{}{14cm}
\diagram{img/diagrams/sequence/item-fake-use}{}{14cm}
\diagram{img/diagrams/sequence/effect-beer-apply-new}{Megváltozott BeerEffect}{14cm}
\diagram{img/diagrams/sequence/janitor-enter-room}{}{14cm}

\clearpage

\section{Prototípus interface-definíciója}
% \comment{Definiálni kell a teszteket leíró nyelvet. Külön figyelmet kell fordítani arra, hogy ha a rendszer véletlen elemeket is tartalmaz, akkor a véletlenszerűség ki-bekapcsolható legyen, és a program determinisztikusan is tesztelhető legyen.}

\subsection{Az interfész általános leírása}
% \comment{A protó (karakteres) input és output felületeit úgy kell kialakítani, hogy az input fájlból is vehető legyen illetőleg az output fájlba menthető legyen, vagyis kommunikációra csak a szabványos be- és kimenet használható.}

A prototípus karakteres interfészen kommunikál a felhasználóval. Az prototípusnak bemenő parancsokat a standard bemenetre írja a felhasználó. Az prototípus a standard kimenetre írt üzenetekkel válaszol a felhasználónak.

\subsection{Bemeneti nyelv}
% \comment{Definiálni kell a teszteket leíró nyelvet. Külön figyelmet kell fordítani arra, hogy ha a rendszer véletlen elemeket is tartalmaz, akkor a véletlenszerűség ki-bekapcsolható legyen, és a program determinisztikusan is futtatható legyen. A szálkezelést is tesztelhető, irányítható módon kell megoldani. A programot egy adott konfigurációból is el kell tudni indítani, vagyis kell olyan parancs, amivel konkrét előre megadott állapotból indul a rendszer (pl. load).}

Az interfész parancsokat vár a bemenetre. Egy parancs egy egyszavas parancsnévből és egyszavas argumentumokból áll. A parancs a parancsnévvel kezdődik, utána szóközzel elválasztva következnek az argumentumok. A parancsok:

\begin{itemize}
\item Create <class> <name> [<constructor params>]
    \begin{itemize}
        \item Leírás: Létrehoz egy objektum példányt a prototípusban megadott osztályból, megadott névvel. 
        Az első paraméter határozza meg, hogy milyen osztályú objektumot hozunk létre. A paraméternek egyeznie kell valamelyik létrehozható osztály nevével. A második paraméter az objektum neve a prototípusban. Ilyen néven kell hivatkozni rá további parancsok kiadásakor.
        Az osztályok, amikből lehet példányt így létrehozni: A Person-ból származó osztályok, Room, Door, az Item leszármazottjai, az Effect leszármazottjai.
        \item Opciók: Ha olyan objektumot hozunk létre, amelynek van nem alapértelmezett konstruktora is, a név után megadhatjuk már a prototípusban létező objektumok nevét is. Ilyenkor a plusz argumentumok az új objektum konstruktorának paraméterei lesznek.
    \end{itemize}
    \item Add <object1> <object2>
    \begin{itemize}
        \item Leírás: Hozzáadja az első paraméterül megadott objektum megfelelő tárolójához a második paraméterül megadott objektumot.
        Az első paraméter osztálya lehet Room vagy a Person osztály valamely leszármazottja. Ha az első paraméter Room, akkor a második lehet az Effect, Item vagy Person osztályok leszármazottja. Ha az első paraméter Person leszármazott, akkor a második lehet Effect vagy Item leszármazott.
    \end{itemize}
    \item Move <person> <door>
    \begin{itemize}
        \item Leírás: Az első paraméterként megadott Person leszármazott objektum használja a második paraméterként megadott Door objektumot.
    \end{itemize}
    \item Update [<object>]
    \begin{itemize}
        \item Leírás: Meghívja a paraméterül megadott Updatable interfész implementáló objektum frissítő metódusát.
        \item Opciók: Az object nem kötelező paraméter, ha nincs megadva, akkor minden objektumot frissít (ahogyan ez játék közben is történne).
    \end{itemize}
    \item Merge <room1> <room2>
    \begin{itemize}
        \item Leírás: Beleolvasztja az első paraméterül megadott Room objektumba a második paraméterül megadott Room objektumot.
    \end{itemize}
    \item Split <room>
    \begin{itemize}
        \item 
        Leírás: Szétválasztja a paraméterül megadott Room objektumbot.
    \end{itemize}
    \item Use <person> <item>
    \begin{itemize}
        \item Leírás: Az első paraméterként megadott Person leszármazott objektum használja a második paraméterként megadott Item leszármazott objektumot, ha az nála van.
    \end{itemize}
    \item Drop <person> <item>
    \begin{itemize}
        \item Leírás: Az első paraméterként megadott Person leszármazott objektum eldobja a második paraméterként megadott Item leszármazott objektumot, ha az nála van.
    \end{itemize}
    \item Link <transistor1> <transistor2>
    \begin{itemize}
        \item Leírás: Összelinkeli a két megadott Transistor objektumot.
    \end{itemize}
    \item Oneway <door> <value>
    \begin{itemize}
        \item Leírás: Megadott Door objektum egyirányúra állítása
        \item Opciók: value egy boolean, lehet true vagy false, hogy egyirányú-e az ajtó.
    \end{itemize}
    \item Status <object>
    \begin{itemize}
        \item Leírás: Kiiratja a megadott objektum aktuális állapotát.
    \end{itemize}
    \item Seed <seed>
    \begin{itemize}
        \item Leírás: Véletlen szám generálásához használt seed beállítása. Ezzel determinisztikussá tehető minden véletlen esemény a tesztelés elősegítése érdekében.
    \end{itemize}
\end{itemize}
Példák a bemeneti nyelvre:
\begin{verbatim}
    Create Room r1
    Create Room r2
    Create Door d r1 r2
    Status d

    Create Student s
    Create Mask m
    Add s m
\end{verbatim}

% \comment{Ha szükséges, meg kell adni a konfigurációs (pl. pályaképet megadó) fájlok nyelvtanát is.}

\subsection{Kimeneti nyelv}
% \comment{Egyértelműen definiálni kell, hogy az egyes bemeneti parancsok végrehajtása után előálló állapot milyen formában jelenik meg a szabványos kimeneten. A program képes legyen olyan kimenetet előállítani, amellyel az objektumok állapota ellenőrizhető (pl. save). Ebben az alfejezetben is precízen definiálni kell, hogy a kimenet nyelve milyen elemekből és milyen szintakszissal áll elő.}

A Status parancs hatására a program kiírja a kért objektum adatatait. Egy sorba kiírja az osztályt és nevet. Ezután a belső állapotának változóit írja ki külön sorokba. Ha egy belső változó egy kollekció akkor ennek elemeit írja ki szóközzel elválasztva. (Ha az elemek objektumok, akkor a prototípusban szereplő nevüket írja ki.) \newline Példák a kimeneti nyelvre:

\begin{verbatim}
    Student student1
    Knock-out time: 0
    Eliminated: false
    Inventory: mask1 mask2 tvsz
    Effects: maskeffect

    Door dora
    Room 1: room1
    Room 2: room2
    One-way: true

    CodeOfStudies tvsz
    Uses left: 2
    Holder: student1
    Room: room1
\end{verbatim}

\section{Összes részletes use-case}
\begin{use-case}
    %név
    {Ajtóhasználat}
    %rövid leírás
    {Egy hallgató megkísérel átlépni az ajtón. (Oktató esetében a lefolyás hasonló.)}
    %aktorok
    {A felhasználó}
    %forgatókönyv
    \textbf{A.1} Az ajtó jó irányba átjárható.
    \newline\textbf{A.1.1} Van elég hely, az átlépés sikeres.
    \newline\textbf{A.1.2} Nincs elég hely, az átlépés megtagadva.
    \newline\textbf{A.2} Az ajtó rossz irányba átjárható, az átlépés megtagadva.
\end{use-case}

\begin{use-case}
    %név
    {Takarító belép a szobába}
    %rövid leírás
    {Egy takarító lép a szobába.}
    %aktorok
    {Takarító}
    %forgatókönyv
    Forgatókönyv \newline 
    \textbf{A.1} A takarító belép a szobába és kiküldi a bent lévő (nem ájult/bénult) hallgatókat és oktatókat, valamint tisztává teszi a szobát.\newline 
    \textbf{A.1.1} Amennyiben a szoba mérgező volt, a takarító kiszellőztet, megszüntetve a mérgezőséget.
\end{use-case}

\begin{use-case}
    %név
    {Maszkfelvétel}
    %rövid leírás
    {Egy hallgató felvesz egy tárgyat. (A lefolyás az oktatók és minden tárgy esetében hasonló).}
    %aktorok
    {A felhasználó}
    %forgatókönyv
    \textbf{A.1} A szoba tiszta. \newline
    \textbf{A.1.1} A hallgató tárgykészeltében még van hely, így felveszi a maszkot. \newline
    \textbf{A.1.2} A hallgató tárgykészeltében már nincs hely, így nem veszi fel a maszkot.  \newline   
     \textbf{A.2} A szoba nem elég tiszta, a tárgy nem vehető fel.
\end{use-case}

\begin{use-case}
    %név
    {Hamis tárgy használat.}
    %rövid leírás
    {Egy hallgató használni akar egy hamis tárgyat.}
    %aktorok
    {A felhasználó}
    %forgatókönyv
    Nem történik semmi.  
\end{use-case}

\begin{use-case}
    %név
    {Maszkhasználat}
    %rövid leírás
    {Egy hallgató használ egy maszkot.}
    %aktorok
    {A felhasználó}
    %forgatókönyv
    Mérgező szobába lépve a hallgatónál van maszk, így az kifejti a hatását.    
\end{use-case}

\begin{use-case}
    %név
    {Légfrissítő használat}
    %rövid leírás
    {Egy hallgató használ egy légfrissítőt.}
    %aktorok
    {A felhasználó}
    %forgatókönyv
    Ha mérgező szobába lépve a hallgatónál van légfrissítő használhatja a szoba kitisztítására. 
\end{use-case}

\begin{use-case}
    %név
    {TVSZ használat}
    %rövid leírás
    {Egy hallgató használ egy TVSZ-t.}
    %aktorok
    {A felhasználó}
    %forgatókönyv
    Olyan szobába lépve, ahol van oktató, a hallgatónál van TVSZ, így az kifejti a hatását.    
\end{use-case}

\begin{use-case}
    %név
    {Söröspohár használat}
    %rövid leírás
    {Egy hallgató használ egy söröspoharat.}
    %aktorok
    {A felhasználó}
    %forgatókönyv
    {A hallgató aktiválja a nála lévő söröspoharat, ami egy időre védettséget biztosít számára az oktatókkal szemben, egyúttal elejt egy nála lévő véletlen tárgyat.}
\end{use-case}

\begin{use-case}
    %név
    {Camembert használat}
    %rövid leírás
    {Egy hallgató használ egy Camembert-t.}
    %aktorok
    {A felhasználó}
    %forgatókönyv 
    A hallgató döntésére a a Camembert mérgezővé teszi a szoba levegőjét.
\end{use-case}

\begin{use-case}
    %név
    {Tranzisztor összekapcsolása}
    %rövid leírás
    {Egy hallgató összekapcsol két tranzisztort.}
    %aktorok
    {A felhasználó}
    %forgatókönyv
    \textbf{A.1} A hallgatónak van két tranzisztora. Ekkor a hallgató döntésére összekapcsolja a kettőt.  
\end{use-case}

\begin{use-case}
    %név
    {Tranzisztor használat}
    %rövid leírás
    {Egy hallgató használ két Tranzisztort.}
    %aktorok
    {A felhasználó}
    %forgatókönyv
    \textbf{A.1} Ha van két összekapcsolt tranzisztora megpróbál átlépni a kijelölt szobába.
    \newline \textbf{A.1.1} A célszobában van hely, a hallgató átlép.
    \newline \textbf{A.1.2} A célszobában nincs hely, a hallgató nem tud átlépni.      
\end{use-case}

\begin{use-case}
    {Sör hatása}
    {Aktív sör hatás megvédi a hallgatót}
    {A felhasználó}
    {A hallgató aktiválta a söröspoharat, ezért védett az oktatókkal szemben.}
\end{use-case}

\begin{use-case}
    %név
    {Sima szobák összeolvadása}
    %rövid leírás
    {Két szoba összeolvad, egyik sem különleges.}
    %aktorok
    {A felhasználó}
    %forgatókönyv
    Két szoba, room1 és room2 összeolvadnak. room2 tárgyai és a benne lévő emberek átkerülnek a room1-be, az ajtó ami eddig a room2 és room3 között volt, most már a room1 és room3 között lesz. 
\end{use-case}

\begin{use-case}
    %név
    {Különleges szobák összeolvadása}
    %rövid leírás
    {Két szoba összeolvad, legalább az egyik különleges.}
    %aktorok
    {A felhasználó}
    %forgatókönyv
    Két szoba, room1 és room2 összeolvadnak. room2 tárgyai és a benne lévő emberek átkerülnek. A közöttük lévő ajtó megszűnik. Az új szobában érvényesül a hatás, amelyik az eredeti szobákon volt.
\end{use-case}

\begin{use-case}
    %név
    {Elátkozott szoba ajtók eltűnése}
    %rövid leírás
    {Elátkozott szoba ajtajai eltűnnek.}
    %aktorok
    {A felhasználó}
    %forgatókönyv
    A room elrejti a door és door2 ajtajait.
\end{use-case}

\begin{use-case}
    %név
    {Elátkozott szoba ajtók megjelnítése}
    %rövid leírás
    {Elátkozott szoba ajtajai megjelennek.}
    %aktorok
    {A felhasználó}
    %forgatókönyv
    A room megjeleníti a door és door2 ajtajait.
\end{use-case}

\begin{use-case}
    %név
    {Maszkeldobás}
    %rövid leírás
    {Hallgató eldob egy maszkot (a lefutás az összes tárgy esetében azonos).}
    %aktorok
    {A felhasználó}
    %forgatókönyv
    A hallgató eltávolítja a maszkot a tárgykészletéből, ami hozzáadódik a szoba tárgyaihoz.
\end{use-case}

\begin{use-case}
    %név
    {Hallgató felveszi a logarlécet.}
    %rövid leírás
    {A hallgató felveszi a logarlécet.}
    %aktorok
    {A felhasználó}
    %forgatókönyv
    Amennyiben a hallgatónál van még hely és a szoba tiszta, felveszi a Logarlécet és a játék véget ér.
\end{use-case}

\begin{use-case}
    %név
    {Oktató felveszi a logarlécet.}
    %rövid leírás
    {Az oktató felveszi a logarlécet.}
    %aktorok
    {A felhasználó}
    %forgatókönyv
    Az oktató felveszi a Logalécet és addig magánál tartja, amíg pl. el nem ájul.
\end{use-case}

\begin{use-case}
    %név
    {Szoba osztódás}
    %rövid leírás
    {Egy szoba osztódik.}
    %aktorok
    {A felhasználó}
    %forgatókönyv
   Új szoba keletkezik - newRoom. A két szoba közé egy új ajtó kerül: door. Az oldDoor eltávolításra kerül a régi szobából és az új szobához adódik hozzá. 
\end{use-case}

\begin{use-case}
    %név
    {Szoba frissítése: hallgatók}
    %rövid leírás
    {Egy szoba frissül, amiben kettő hallgató található.}
    %aktorok
    {A felhasználó}
    %forgatókönyv
    A szoba frissül, és frissíti a benne található hallgatókat.
\end{use-case}

\begin{use-case}
    %név
    {Szoba frissítése: hallgató és oktató}
    %rövid leírás
    {Egy szoba frissül, amiben egy hallgató és egy oktató található.}
    %aktorok
    {A felhasználó}
    %forgatókönyv
    A szoba frissül, és frissíti a benne található hallgatót és oktatót. Az oktató elveszi a hallgató lelkét.
\end{use-case}

\begin{use-case}
    %név
    {Szoba frissítése: hallgató maszkkal és gáz effekt}
    %rövid leírás
    {Egy szoba frissül, amiben egy hallgató és egy gáz effekt található. A hallgatón van egy maszk effekt.}
    %aktorok
    {A felhasználó}
    %forgatókönyv
    A szoba frissül, és frissíti a benne található hallgatót és gáz hatást. A hallgató frissíti a maszk hatást. A felhasználó döntése alapján:
    \newline \textbf{A.1} A gáz hatás megszűnik.
    \newline    \textbf{A.1.1} A maszk hatás megszűnik.
    \newline    \textbf{A.1.1} A maszk hatás nem szűnik meg.
    \newline \textbf{A.2} A gáz hatás nem szűnik meg és elkábítja a hallgatót.
    \newline    \textbf{A.2.1} A maszk hatás megszűnik.
    \newline    \textbf{A.2.2} A maszk hatás nem szűnik meg és megszünteti a hallgató kábult állapotát.
\end{use-case}

\begin{use-case}
    %név
    {Szoba frissítése: oktató és rongy effekt}
    %rövid leírás
    {Egy szoba frissül, amiben egy oktató és egy rongy effekt található.}
    %aktorok
    {A felhasználó}
    %forgatókönyv
    A szoba frissül, és frissíti a benne található oktatót és rongy hatást. A felhasználó döntése alapján:
    \newline \textbf{A.1} A rongy hatás megszűnik.
    \newline \textbf{A.2} A rongy hatás nem szűnik meg és megbékíti az oktatót.
\end{use-case}

\section{Tesztelési terv}
% \comment{A tesztelési tervben definiálni kell, hogy a be- és kimeneti fájlok egybevetésével miként végezhető el a program tesztelése. Meg kell adni teszt forgatókönyveket. Az egyes teszteket elég informálisan, szabad szövegként leírni. Teszt-esetenként egy-öt mondatban. Minden teszthez meg kell adni, hogy mi a célja, a proto mely funkcionalitását, osztályait stb. teszteli. Az alábbi táblázat minden teszt-esethez külön-külön elkészítendő.}

\subsection{A tesztek felépítése}
A tesztesetek megtervezésénél az Arrange, Act, Assert mintát alkalmaztuk. A teszteket úgy rendeztük sorba, hogy először a legalapvetőbb funkciókat tesztelje, majd térjen rá az egyre összetettebb funkciók ellenőrzésére. Minden teszt csakis egyetlen egy funkciót tesztel.

\subsection{Egyszerű működést ellenörző tesztek}
\test{CreateRoom}{A teszt létrehoz egyetlen szobát. \newline Módosítást nem hajt rajta végre. \newline Ellenőrzi a létrehozott szoba tulajdonságait.}{Megbizonyosodni arról, hogy a szoba osztály konstruktora megbízható.}

\test{CreateStudent}{A teszt létrehoz egy Hallgatót. \newline Más módosítást nem hajt rajta végre. \newline Ellenőrzi, hogy a Hallgató a megfelelő tulajdonságokkal rendelkezik. }{Megbizonyosodni arról, hogy a hallgató osztály konstruktora megbízható.}

\test{CreateInventory}{A teszt létrehoz egy Hallgatót. \newline Más módosítást nem halyt rajta végre. \newline Ellenőrzi, hogy a Hallgató rendelkezik-e az elvárt belső tulajdonságú Inventory-val. }{Megbizonyosodni arról, hogy az invetory osztály konstruktora megbízható.}
Megjegyzés: Ezt a tesztesetet a Teacher és Janitor objektumra is elvégezhetnénk, de a tesztben releváns metódusok a Student, Teacher és Janitor esetében azonosak, mivel mind a ketten a Person osztály leszármazottai.

\test{PlaceStudent}{A teszt létrehoz egy Szobát és egy Hallgatót. \newline A Hallgatót elhelyezi a Szobában. \newline Ellenőrzi, hogy a szoba ténylegesen tartalmazza-e a Hallgatót.  }{Megbizonyosodni arról, hogy az elvárt belső állapotválltozások mennek végre, amikor egy Person típusú entitást egy szobába rakunk.}
Megjegyzés: Ezt a tesztesetet a Teacher és Janitor objektumra is elvégezhetnénk, de a tesztben releváns metódusok a Student, Teacher és Janitor esetében azonosak, mivel mind a ketten a Person osztály leszármazottai.

\test{CreateTeacher}{A teszt létrehoz egy Oktatót. \newline Más módosítást nem hajt rajta végre. \newline Ellenőrzi, hogy az Oktató a megfelelő tulajdonságokkal rendelkezik. }{Megbizonyosodni arról, hogy az Oktató osztály konstruktora megbízható.}

\test{AddDoor}{A teszt létrehoz két szobát és egy ajtót. \newline Az ajtót a két szoba közé rakja. \newline Ellenőrzi, hogy az ajtó objektumban a két tárolt szoba értéke megfelelő lett-e. }{Megbizonyosodni arról, hogy az ajtó elhelyezés két szoba közt megfelelően működik. }

\test{ApplyRagEffectToRoom}{A teszt létrehoz egy Diákot és egy Nedves táblatörlőrongyot.. \newline A Rongyot a diáknak átadja. \newline Ellenőrzi, hogy a diák rendelkezik-e a megfelelő paraméterű RagEffect-el. }{A Táblatörlő által hordozott Effect Diákra való terjedésének tesztelése.}

\test{ApplyGasEffectToRoom}{A teszt létrehoz egy Camambert és egy Szobát. \newline A Camambert elhelyezi a szobában. \newline Ellenőrzi, hogy a szoba rendelkezik-e a megfelelő paraméterű GasEffect-el.}{A Camambert által hordozott Effect szobára való terjedésének tesztelése.}

\test{ApplyCleanEffectToRoom}{Léterhoz egy AirFreshener-t egy Szobát és egy GasEffectet \newline A GasEffet-et a szobára helyezi, a szobában elhelyezi a Légfrisítőt. \newline Ellenőrzi, hogy a szobában nincs többé GasEffect.}{A légfrissítő tesztelése.}

\test{InheritGasEffect}{A teszt léterhoz egy szobát, egy Camambert-et, és egy diákot. \newline A Camambert-et elhelyezi a szobában. A játékost belerakja a szobába. }{Megbizonyosodni arról, hogy a szoba által számontartott Effectet átveszik-e a benne lévő szereplők.}
Megjegyzés: Ezt a tesztesetet a Teacher és Janitor objektumra is elvégezhetnénk, de a tesztben releváns metódusok a Student, Teacher és Janitor esetében azonosak, mivel mind a ketten a Person osztály leszármazottai.

\test{ApplyBeerEffectToStudent}{A teszt létrehoz egy Diákot és egy Sört. \newline A Sört a diáknak átadja. \newline Ellenőrzi, hogy a diák rendelkezik-e a megfelelő paraméterű BeerEffect-el. }{A Sör által hordozott Effect Diákra való terjedésének tesztelése.}

\test{BeerEffectOnStudent}{A teszt létrehoz egy Diákot, egy Táblatörlőrongyot és egy Sört. \newline A Sört és a Rongyot  adiáknak átadja. \newline Ellenőrzi, hogy a diák elejtette-e a Rongyot. }{A BeerEffect mellékhatásának tesztelése}
Megjegyzés: egynél több tárgyat nem szükséges a Diáknak adni, mivel a specifikáció csak azt kéri, hogy egy tárgyat dobjon el. Az eldobandó tárgy kiválasztása nem érdekes.

\test{ApplyMaskEffectToStudent}{A teszt létrehoz egy Diákot és egy Maszkot. \newline A Maszkot a diáknak átadja. \newline Ellenőrzi, hogy a diák rendelkezik-e a megfelelő paraméterű MaskEffect-el.}{A Maszk által hordozott Effect Diákra való terjedésének tesztelése.}

\test{ApplyFakeMaskEffectToStudent}{A teszt létrehoz egy Diákot és egy Hamis Maszkot. \newline A Hamis Maszkot a diáknak átadja. \newline Ellenőrzi, hogy a diák nem rendelkezik MaskEffect-el.}{A Hamis Maszk Effect nélküliségének tesztelése.}

Megjegyzés: A fenti két teszt elvégezhető lenne az összes tárgyra és szereplőre is, de mivel az ősosztályaik közösek és a FakeItem implementációknál metódusok kerülnek törlésre, ezért ezekre külön teszteket nem terveztünk.

\test{MoveStudentFromRoomToRoom}{A teszt létrehoz két szobát(szobaA és szobaB), egy diákot és egy kétirányú ajtót. \newline Az ajtót elhelyezi szobaA és szobaB közé. A diákot elhelyezi szobaA-ban. A diákot átlépteti szobaB-be. \newline Ellenőrzi, hogy a diák átkerült-e szobaB-be. }{Megbizonyosodni arról, hogy a szereplők képesek használni az szobák közti ajtókat.}

\test{MovementRemoveStudentFromOrigin}{A teszt létrehoz két szobát(szobaA és szobaB), egy diákot és egy kétirányú ajtót. \newline Az ajtót elhelyezi szobaA és szobaB közé. A diákot elhelyezi szobaA-ban. A diákot átlépteti szobaB-be. \newline Ellenőrzi, hogy a Diák el lett-e távolítva szobaA-ból. }{Megbizonyosodni arról, hogy a szereplők szobák közti mozgása során az elhagyott szobából kikerül a szereplő.}

\test{DenyStudentMovementFromRoomToRoomDueToDirection}{A teszt létrehoz két szobát(szobaA és szobaB), egy diákot és egy Ajtót --> szobaA irányú ajtót. \newline Az ajtót elhelyezi szobaA és szobaB közé. A diákot elhelyezi szobaA-ban. A diákot megpróbálja átléptetni szobaB-be. }{Megbizonyosodni arról, hogy az ajtók rossz irányú mozgást nem engednek át.}

\test{DenyStudentMovementFromRoomToRoomDueToCapacity}{A teszt létrehoz két szobát(szobaA és szobaB), egy diákot és egy Ajtót --> szobaA irányú ajtót. \newline Az ajtót elhelyezi szobaA és szobaB közé. A szobaB kapacitását 0-ra állítja. A diákot elhelyezi szobaA-ban. A diákot megpróbálja átléptetni szobaB-be. }{Megbizonyosodni arról, hogy a modell megtagadja a megtelt szobába való belépést.}

\test{UpdateBeerEffect}{A teszt létrehoz egy Diákot és egy Sört. \newline A Sört a diáknak átadja. Meghívja a Diák update függvényét. \newline Ellenőrzi, hogy frissült-e az effect.}{Az effect frissítésének ellenőrzése.}

\test{UpdateRagEffect}{A teszt létrehoz egy Diákot és egy Nedves táblatörlőt. \newline A Táblatörlőt a diáknak átadja. Meghívja a Diák update függvényét. \newline Ellenőrzi, hogy frissült-e az effect.}{Az effect frissítésének ellenőrzése.}

\test{UpdateMaskEffect}{A teszt létrehoz egy Diákot és egy Maszkot. \newline A Maszokt a diáknak átadja. Meghívja a Diák update függvényét. \newline Ellenőrzi, hogy frissült-e az effect.}{Az effect frissítésének ellenőrzése.}

\test{UpdateGasEffect}{A teszt létrehoz egy Szobát és egy Camamber-et. \newline A Camambert-et a szobába helyezi. Meghívja a Szoba update függvényét. \newline Ellenőrzi, hogy frissült-e az effect.}{Az effect frissítésének ellenőrzése.}

\test{RoomUpdateStudent}{A teszt létrehoz egy Szobát és egy Diákot. \newline A Diákot a szobába rakja. Meghívja a szoba update függvényét. \newline Ellenőrzi, hogy a szobában lévő diák is frissült-e.}{Az update esemény terjedésének ellenörzése Diákra.}

\test{RoomUpdateStudent}{A teszt létrehoz egy Szobát és egy Diákot. \newline A Diákot a szobába rakja. Meghívja a szoba update függvényét. \newline Ellenőrzi, hogy a szobában lévő diák is frissült-e.}{Az update esemény terjedésének ellenörzése Diákra.}
Megjegyzés: Ezt a tesztesetet a Teacher objektumra is elvégezhetnénk, de a tesztben releváns metódusok a Student és Teacher esetében azonosak, mivel mind a ketten a Person osztály leszármazottai.

\test{RoomUpdateEffect}{A teszt létrehoz egy Szobát és egy BeerEffect-et. \newline Az effectet a szobához adja. Meghívja a szoba update függvényét. \newline Ellenőrzi, hogy a szobában lévő effect is frissült-e.}{Az update esemény terjedésének ellenörzése Effectekre.}
Megjegyzés: A BeerEffectet tetszőlegesen választottuk a teszthez. Minden effect Update metódusa ugyn úgy viselkedik, ezért szükségtelen mindegyiket tesztelni.

\test{MergeRooms}{A teszt létrehoz két szobát(szobaA és szobaB) \newline SzobaA meghívja a merge(szobaB) metódust. \newline A teszt ellenőrzi, hogy a szobaösszeolvasztás sikeresen végbement-e.}{A szobaösszeolvadás ellenőrzése.}

\test{SplitRoom}{A teszt létrehoz egy szobát. \newline A szobának meghívja a split() metódusáőát. \newline Ellenőrzi, hogy a szobaosztódás megfelelően végbement-e.}{A szobaosztódás funkció tesztelése.}

\test{PickUpItemWithInventorySpace}{A teszt létrehoz egy szobát, egy hallgatót és egy Sört \newline A hallgatót és a Sört egy szobába helyezi. A hallgatóval felveteti a Sört. \newline Ellenőrzi, hogy  Sör a Hallgató birtokába került-e}{A tárgyfelvétel funkció ellenőrzése.}
Megjegyzés: Elegendő egy tetszőleges típusú Item-mel és egy tetszőleges típusú tárgyal tesztelni, mivel a teszetesethez releváns metódusok az absztrakt osztályokban kerültek definiálásra.

\test{PickUpItemWithoutInventorySpace}{A teszt létrehoz egy szobát, egy hallgatót és 6 darab Sört. \newline A hallgatót és az egyik Sört egy szobába helyezi. A hallgatónak adja az 5 darab másik sört. A hallgatóval felveteti az utoldó Sört. \newline Ellenőrzi, hogy sikertelen lett-e a tárgyfelvétel.}{A tárgyfelvétel funkció ellenőrzése.}
Megjegyzés: Elegendő egy tetszőleges típusú Item-mel és egy tetszőleges típusú tárgyal tesztelni, mivel a tszetesethez releváns metódusok az absztrakt osztályokban kerültek definiálásra.

\test{DropItem}{A teszt létrehoz egy szobát, egy hallgatót és egy Sört \newline A hallgatót a szobába helyezi és neki adja a Sört. A hallgatóval eldobatja a Sört. \newline Ellenőrzi, hogy a Sör el lett-e távolítva a Hallgató tulajdonából és a szobába került-e. }{Az eldobás funkció ellenőrzése.}
Megjegyzés: Elegendő egy tetszőleges típusú Item-mel és egy tetszőleges típusú tárgyal tesztelni, mivel a tszetesethez releváns metódusok az absztrakt osztályokban kerültek definiálásra.

\subsection{Összetett működést vizsgáló tesztek.}

\test{TeleportUsingTransistors}{A teszt létrehoz két szobát(szobaA, szobaB), két tranzisztort(tranzisztorA, tranzisztorB) és egy hallgatót. \newline A hallgatót elhelyezi szobaA-ban. A tranzisztorokat összelinkeli. TranzisztorA-t odaadja a hallagtónak, tranzisztorB-t pedig elhelyezi szobaB-ben. A hallgatót teleportáltatja TranzisztorA segítségével. \newline Ellenőrzi, hogy a diák átkerült-e szobaB-be. }{A teleportálás funkció működésének bizonyítása.}

\test{BeerProtectStudent}{A teszt létrehoz egy szobát, egy Hallgatót és egy Oktatót és egy Sört. \newline A Sört odaadja a Hallgatónak. A Hallgatót és Oktatót a Szobába rakja. \newline Ellenőrzi, hogy az oktató nem támadja meg a Hallgatót.}{A Sör védelemnyújtásának ellenőrzése.}

\test{KnockOutWithOutMask}{A teszt létrehoz egy Szobát, egy Diákot és egy Camamber-et. \newline A Camambert-et és a Diákot a szobába helyezi. \ Ellenőrzzi, hogy a Diák elájult-e.}{A Cammabret működésének ellenőrzése.}
Megjegyzés: Ezt a tesztesetet a Teacher objektumra is elvégezhetnénk, de a tesztben releváns metódusok a Student és Teacher esetében azonosak, mivel mind a ketten a Person osztály leszármazottai.


\test{KnockOutWithMask}{A teszt létrehoz egy Szobát, egy Diákot, Maszkot és egy Camamber-et. \newline A Maszkot a Diákra helyezi. A Camambert-et és a Diákot a szobába helyezi. \ Ellenőrzzi, hogy a Diák nem elájult}{A Maszk működésének ellenőrzése.}
Megjegyzés: Ezt a tesztesetet a Teacher objektumra is elvégezhetnénk, de a tesztben releváns metódusok a Student és Teacher esetében azonosak, mivel mind a ketten a Person osztály leszármazottai.

\test{JanitorClearRoom}{A teszt létrehoz két Szobát, egy Diákot, egy ajtót és egy Takarítót. \newline A két szoba közé helyezi az ajtót, melyet kétirányúra állít. A diákot és a takarítót két különböző szobába helyezi. A Takarítót belépteti a diák szobájába. \newline Ellenőrzi hogy a diák kikerült-e a szobából.}{A szoba clean() metódusának működésének tesztelése.}

\test{JanitorClearRoomWithNonMoving}{A teszt létrehoz két Szobát, egy Diákot, egy ajtót, egy GasEffect-et és egy Takarítót. \newline A két szoba közé helyezi az ajtót, melyet kétirányúra állít. A diákot és a takarítót két különböző szobába helyezi. A diák szobájára helyezi a gasEffectet, ezzel megbénítva őt. A Takarítót belépteti a diák szobájába. \newline Ellenőrzi, hogy a Diák és a Takarító egy szobában marad-e.}{A szoba clean() metódusának működésének tesztelése.}

\test{Teszteset neve}{Leírás}{A végrehajtott teszt célja}

\section{Tesztelést támogató segéd- és fordítóprogramok specifikálása}

A tesztelés megkönnyítéséhez egy teszt futtató program készül szintén Java-ban, ugyan azon feltételek mellett fordítható, és futtatható, mint a fő program. 

A teszteseteket szövegfájlokként olvassa be külün a futtatandó parancsokat és az ezekre a parancsokra elvárt kimenetet.