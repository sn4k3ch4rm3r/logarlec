\chapter{Szkeleton tervezése}


\section{A szkeleton modell valóságos use-case-ei}
\subsection{Use-case diagram}
\diagram{img/diagrams/usecase/test-use-cases.png}{Use-case diagram}{11cm}
\clearpage
\subsection{Use-case leírások}
\begin{use-case}
    %név
    {Ajtóhasználat}
    %rövid leírás
    {Egy hallgató megkísérel átlépni az ajtón. (Oktató esetében a lefolyás hasonló.)}
    %aktorok
    {A felhasználó}
    %forgatókönyv
    \textbf{A.1} Az ajtó jó irányba átjárható.
    \newline\textbf{A.1.1} Van elég hely, az átlépés sikeres.
    \newline\textbf{A.1.2} Nincs elég hely, az átlépés megtagadva.
    \newline\textbf{A.2} Az ajtó rossz irányba átjárható, az átlépés megtagadva.
\end{use-case}

\begin{use-case}
    %név
    {Maszkfelvétel}
    %rövid leírás
    {Egy hallgató felvesz egy tárgyat. (A lefolyás az oktatók és minden tárgy esetében hasonló).}
    %aktorok
    {A felhasználó}
    %forgatókönyv
    \textbf{A.1} A hallgató tárgykészeltében még van hely, így felveszi a maszkot. \newline
    \textbf{A.2} A hallgató tárgykészeltében már nincs hely, így nem veszi fel a maszkot.    
\end{use-case}

\begin{use-case}
    %név
    {Maszkhasználat}
    %rövid leírás
    {Egy hallgató használ egy maszkot.}
    %aktorok
    {A felhasználó}
    %forgatókönyv
    Mérgező szobába lépve a hallgatónál van maszk, így az kifejti a hatását.    
\end{use-case}

\begin{use-case}
    %név
    {TVSZ használat}
    %rövid leírás
    {Egy hallgató használ egy TVSZ-t.}
    %aktorok
    {A felhasználó}
    %forgatókönyv
    Olyan szobába lépve, ahol van oktató, a hallgatónál van TVSZ, így az kifejti a hatását.    
\end{use-case}

\begin{use-case}
    %név
    {Söröspohár használat}
    %rövid leírás
    {Egy hallgató használ egy söröspoharat.}
    %aktorok
    {A felhasználó}
    %forgatókönyv
    {A hallgató aktiválja a nála lévő söröspoharat, ami egy időre védettséget biztosít számára az oktatókkal szemben.}
\end{use-case}

\begin{use-case}
    %név
    {Camembert használat}
    %rövid leírás
    {Egy hallgató használ egy Camembert-t.}
    %aktorok
    {A felhasználó}
    %forgatókönyv 
    A hallgató döntésére a a Camembert mérgezővé teszi a szoba levegőjét.
\end{use-case}

\begin{use-case}
    %név
    {Tranzisztor összekapcsolása}
    %rövid leírás
    {Egy hallgató összekapcsol két tranzisztort.}
    %aktorok
    {A felhasználó}
    %forgatókönyv
    \textbf{A.1} A hallgatónak van két tranzisztora. Ekkor a hallgató döntésére összekapcsolja a kettőt.  
\end{use-case}

\begin{use-case}
    %név
    {Tranzisztor használat}
    %rövid leírás
    {Egy hallgató használ két Tranzisztort.}
    %aktorok
    {A felhasználó}
    %forgatókönyv
    \textbf{A.1} Ha van két összekapcsolt tranzisztora megpróbál átlépni a kijelölt szobába.
    \newline \textbf{A.1.1} A célszobában van hely, a hallgató átlép.
    \newline \textbf{A.1.2} A célszobában nincs hely, a hallgató nem tud átlépni.      
\end{use-case}

\begin{use-case}
    {Sör hatása}
    {Aktív sör hatás megvédi a hallgatót}
    {A felhasználó}
    {A hallgató aktiválta a söröspoharat, ezért védett az oktatókkal szemben.}
\end{use-case}

\begin{use-case}
    %név
    {Sima szobák összeolvadása}
    %rövid leírás
    {Két szoba összeolvad, egyik sem különleges.}
    %aktorok
    {A felhasználó}
    %forgatókönyv
    Két szoba, room1 és room2 összeolvadnak. room2 tárgyai és a benne lévő emberek átkerülnek a room1-be, az ajtó ami eddig a room2 és room3 között volt, most már a room1 és room3 között lesz. 
\end{use-case}

\begin{use-case}
    %név
    {Szobák összeolvadása}
    %rövid leírás
    {Két szoba összeolvad, legalább az egyik különleges.}
    %aktorok
    {A felhasználó}
    %forgatókönyv
    Két szoba, room1 és room2 összeolvadnak. room2 tárgyai és a benne lévő emberek átkerülnek. A közöttük lévő ajtó megszűnik. Az új szobában érvényesül a hatás, amelyik az eredeti szobákon volt.
\end{use-case}

\begin{use-case}
    %név
    {Elátkozott szoba ajtók eltűnése}
    %rövid leírás
    {Elátkozott szoba ajtajai eltűnnek.}
    %aktorok
    {A felhasználó}
    %forgatókönyv
    A room elrejti a door és door2 ajtajait.
\end{use-case}

\begin{use-case}
    %név
    {Elátkozott szoba ajtók megjelnítése}
    %rövid leírás
    {Elátkozott szoba ajtajai megjelennek.}
    %aktorok
    {A felhasználó}
    %forgatókönyv
    A room megjeleníti a door és door2 ajtajait.
\end{use-case}

\begin{use-case}
    %név
    {Maszkeldobás}
    %rövid leírás
    {Hallgató eldob egy maszkot (a lefutás az összes tárgy esetében azonos).}
    %aktorok
    {A felhasználó}
    %forgatókönyv
    A hallgató eltávolítja a maszkot a tárgykészletéből, ami hozzáadódik a szoba tárgyaihoz.
\end{use-case}

\begin{use-case}
    %név
    {Hallgató felveszi a logarlécet.}
    %rövid leírás
    {A hallgató felveszi a logarlécet.}
    %aktorok
    {A felhasználó}
    %forgatókönyv
    Amennyiben a hallgatónál van még hely, felveszi a Logarlécet és a játék véget ér.
\end{use-case}

\begin{use-case}
    %név
    {Oktató felveszi a logarlécet.}
    %rövid leírás
    {Az oktató felveszi a logarlécet.}
    %aktorok
    {A felhasználó}
    %forgatókönyv
    Az oktató felveszi a Logalécet és addig magánál tartja, amíg pl. el nem ájul.
\end{use-case}

\begin{use-case}
    %név
    {Szoba osztódás}
    %rövid leírás
    {Egy szoba osztódik.}
    %aktorok
    {A felhasználó}
    %forgatókönyv
   Új szoba keletkezik - newRoom. A két szoba közé egy új ajtó kerül: door. Az oldDoor eltávolításra kerül a régi szobából és az új szobához adódik hozzá. 
\end{use-case}

\begin{use-case}
    %név
    {Szoba frissítése: hallgatók}
    %rövid leírás
    {Egy szoba frissül, amiben kettő hallgató található.}
    %aktorok
    {A felhasználó}
    %forgatókönyv
    A szoba frissül, és frissíti a benne található hallgatókat.
\end{use-case}

\begin{use-case}
    %név
    {Szoba frissítése: hallgató és oktató}
    %rövid leírás
    {Egy szoba frissül, amiben egy hallgató és egy oktató található.}
    %aktorok
    {A felhasználó}
    %forgatókönyv
    A szoba frissül, és frissíti a benne található hallgatót és oktatót. Az oktató elveszi a hallgató lelkét.
\end{use-case}

\begin{use-case}
    %név
    {Szoba frissítése: hallgató maszkkal és gáz effekt}
    %rövid leírás
    {Egy szoba frissül, amiben egy hallgató és egy gáz effekt található. A hallgatón van egy maszk effekt.}
    %aktorok
    {A felhasználó}
    %forgatókönyv
    A szoba frissül, és frissíti a benne található hallgatót és gáz hatást. A hallgató frissíti a maszk hatást. A felhasználó döntése alapján:
    \newline \textbf{A.1} A gáz hatás megszűnik.
    \newline    \textbf{A.1.1} A maszk hatás megszűnik.
    \newline    \textbf{A.1.1} A maszk hatás nem szűnik meg.
    \newline \textbf{A.2} A gáz hatás nem szűnik meg és elkábítja a hallgatót.
    \newline    \textbf{A.2.1} A maszk hatás megszűnik.
    \newline    \textbf{A.2.2} A maszk hatás nem szűnik meg és megszünteti a hallgató kábult állapotát.
\end{use-case}

\begin{use-case}
    %név
    {Szoba frissítése: oktató és rongy effekt}
    %rövid leírás
    {Egy szoba frissül, amiben egy oktató és egy rongy effekt található.}
    %aktorok
    {A felhasználó}
    %forgatókönyv
    A szoba frissül, és frissíti a benne található oktatót és rongy hatást. A felhasználó döntése alapján:
    \newline \textbf{A.1} A rongy hatás megszűnik.
    \newline \textbf{A.2} A rongy hatás nem szűnik meg és megbékíti az oktatót.
\end{use-case}
\clearpage
\section{A szkeleton kezelői felületének terve, dialógusok}

A szkeleton egy konzol alapú menüvezérelt program lesz. Ez a program standard bementeről fog teszteseteket beolvasni, majd ezekre egy egyszerű szekvenciát kiírni a terminálablakba. Ezeket a szekvencia kiírásokat az érintett osztályok meghívásának segítségével oldja meg.

A program indításakor egy főmenű jelenik meg. Ebben a főmenűben sorszámozva jelennek meg a különböző tesztelhető use-cse-ek. Egy use-case-t a sorszámának begépelésével lehet kiválasztani. A főmenű első pár sora így fog megjelenni:
\space

\begin{verbatim}
    1.  Room: hide and show doors
    2.  Room: merge with an effect on
    3.  Room: merge
    ...
    The chosen test number: 
\end{verbatim}
\space

A kiválasztást követően a következő példa szerint ábrázoljuk a függvényhívásokat:

 
\begin{verbatim}
    ┌ Object()
    └ obj
    ┌ obj.fn()
    │  ┌ obj.fn2()
    │  └ void
    └ ret_val
\end{verbatim}

Az egymásba ágyazott hívásokat beljebb húzva ábrázoljuk. A kiírásnál a konstruktorhívások visszatérési érétke az objektum neve lesz. 

Ha szükséges egyes tesztesetek lefutása közben a program felhasználó inputokat is kérhet egy-egy prompt formájában.
\clearpage

\section{Szekvencia diagramok a belső működésre}
\diagram{img/diagrams/test-sequence/door-use}{}{14cm}
\diagram{img/diagrams/test-sequence/room-hide-doors}{}{14cm}
\diagram{img/diagrams/test-sequence/room-merge-with-effect}{}{14cm}
\diagram{img/diagrams/test-sequence/room-merge}{}{14cm}
\diagram{img/diagrams/test-sequence/room-show-doors}{}{14cm}
\diagram{img/diagrams/test-sequence/room-split}{}{14cm}
\diagram{img/diagrams/test-sequence/room-update-gas-and-mask}{}{14cm}
\diagram{img/diagrams/test-sequence/room-update-no-teachers}{}{14cm}
\diagram{img/diagrams/test-sequence/room-update-rag-effect}{}{14cm}
\diagram{img/diagrams/test-sequence/room-update-student-teacher}{}{14cm}
\diagram{img/diagrams/test-sequence/student-drop-mask}{}{14cm}
\diagram{img/diagrams/test-sequence/student-link-transistor}{}{14cm}
\diagram{img/diagrams/test-sequence/student-pickup-mask}{}{14cm}
\diagram{img/diagrams/test-sequence/student-pickup-sliderule}{}{14cm}
\diagram{img/diagrams/test-sequence/student-protected-by-beer}{}{14cm}
\diagram{img/diagrams/test-sequence/student-use-beer}{}{14cm}
\diagram{img/diagrams/test-sequence/student-use-camebert}{}{14cm}
\diagram{img/diagrams/test-sequence/student-use-code-of-studies}{}{14cm}
\diagram{img/diagrams/test-sequence/student-use-mask}{}{14cm}
\diagram{img/diagrams/test-sequence/student-use-transistor}{}{14cm}
\diagram{img/diagrams/test-sequence/teacher-pickup-sliderule}{}{14cm}

\clearpage
\section{Szekvencia diagramok inicializációra}
\diagram{img/diagrams/test-create-sequence/link-transistor-create}{}{14cm}
\diagram{img/diagrams/test-create-sequence/room-merge-create}{}{14cm}
\diagram{img/diagrams/test-create-sequence/room-merge-toxic-create}{}{14cm}
\diagram{img/diagrams/test-create-sequence/room-show-hide-doors-create}{}{14cm}
\diagram{img/diagrams/test-create-sequence/room-split-create}{}{14cm}
\diagram{img/diagrams/test-create-sequence/room-update-gas-and-mask-create}{}{14cm}
\diagram{img/diagrams/test-create-sequence/room-update-no-teachers-create}{}{14cm}
\diagram{img/diagrams/test-create-sequence/room-update-rag-effect}{}{14cm}
\diagram{img/diagrams/test-create-sequence/room-update-student-teacher-create}{}{14cm}
\diagram{img/diagrams/test-create-sequence/student-drop-mask-create}{}{14cm}
\diagram{img/diagrams/test-create-sequence/student-entered-room-create}{}{14cm}
\diagram{img/diagrams/test-create-sequence/student-pickup-mask-create}{}{14cm}
\diagram{img/diagrams/test-create-sequence/student-pickup-sliderule-create}{}{14cm}
\diagram{img/diagrams/test-create-sequence/student-protected-by-beer-create}{}{14cm}
\diagram{img/diagrams/test-create-sequence/student-use-beer-create}{}{14cm}
\diagram{img/diagrams/test-create-sequence/student-use-camembert-create}{}{14cm}
\diagram{img/diagrams/test-create-sequence/student-use-code-of-studies-create}{}{14cm}
\diagram{img/diagrams/test-create-sequence/student-use-mask-create}{}{14cm}
\diagram{img/diagrams/test-create-sequence/student-use-transistor-create}{}{14cm}
\diagram{img/diagrams/test-create-sequence/teacher-pickup-sliderule-create}{}{14cm}