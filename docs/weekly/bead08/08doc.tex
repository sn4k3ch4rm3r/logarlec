
\chapter{Részletes tervek}
% \comment{A dokumentum célja, hogy pontosan specifikálja az implementálandó osztályokat, beleértve a privát attribútumokat és metódusokat, ezek definícióját is.A dokumentum második fele részletesen be kell mutassa a korábban definiált be-és kimeneti nyelv szintakszisát felhasználva, hogy mely tesztekkel lesz a prototípus ellenőrizve.}
\setcounter{section}{-1}
\section{Javítások}

\subsection{Prototípus interface definíciója}

\subsubsection{Bemeneti nyelv}
% \comment{Definiálni kell a teszteket leíró nyelvet. Külön figyelmet kell fordítani arra, hogy ha a rendszer véletlen elemeket is tartalmaz, akkor a véletlenszerűség ki-bekapcsolható legyen, és a program determinisztikusan is futtatható legyen. A szálkezelést is tesztelhető, irányítható módon kell megoldani. A programot egy adott konfigurációból is el kell tudni indítani, vagyis kell olyan parancs, amivel konkrét előre megadott állapotból indul a rendszer (pl. load).}

Az interfész parancsokat vár a bemenetre. Egy parancs egy egyszavas parancsnévből és egyszavas argumentumokból áll. A parancs a parancsnévvel kezdődik, utána szóközzel elválasztva következnek az argumentumok. A parancsok:

\begin{itemize}
    \item Add <holder> <object>
    \begin{itemize}
        \item Leírás: Hozzáadja az első paraméterül megadott szoba vagy személy megfelelő tárolójához a második paraméterül megadott objektumot.
        
        \item Első paraméter: Az első paraméter Person leszármazott, vagy szoba.
        \begin{itemize}
            \item Janitor
            \item Room
            \item Student
            \item Teacher
        \end{itemize}
        \item Második paraméter: A második paraméter lehet az Effect, Person vagy Item osztályok leszármazottjának példánya.
        \begin{itemize}
            \item AirFreshener
            \item Beer
            \item BeerEffect
            \item Camembert
            \item CleanEffect
            \item CodeOfStudies
            \item Door
            \item FakeCodeOfStudies
            \item FakeMask
            \item FakeSlideRule
            \item GasEffect
            \item Janitor (Person leszármazott esetén nem)
            \item JanitorEffect
            \item Mask
            \item MaskEffect
            \item RagEffect
            \item Slide Rule
            \item Student (Person leszármazott esetén nem)
            \item Teacher (Person leszármazott esetén nem) 
            \item Transistor
            \item WetRag
        \end{itemize}
    \end{itemize}
    
    \item Create <class> <name> [<constructor params>]
    \begin{itemize}
        \item Leírás: Létrehoz egy objektum példányt a prototípusban megadott osztályból, megadott névvel.
        \item Első paraméter: Az első paraméter határozza meg, hogy milyen osztályú objektumot hozunk létre. A paraméternek egyeznie kell valamelyik létrehozható osztály nevével:
        \begin{itemize}
        \item AirFreshener
        \item Beer
        \item BeerEffect
        \item Camembert
        \item CleanEffect
        \item CodeOfStudies
        \item Door: a konstruktor paraméterekhez oda kell írni a két szoba nevét, amelyeket az ajtó összeköt:
         \texttt{Create Door d1 r1 r2}
        \item FakeCodeOfStudies
        \item FakeMask
        \item FakeSlideRule
        \item GasEffect
        \item Janitor
        \item JanitorEffect
        \item Mask
        \item MaskEffect
        \item RagEffect
        \item Room: opcionális paraméter a kapacitás, alapértelmezetten négy.
        \item Student
        \item SlideRule
        \item Teacher
        \item Transistor
        \item WetRag
        \end{itemize}

        \item Második paraméter: A második paraméter az objektum neve a prototípusban. Ilyen néven kell hivatkozni rá további parancsok kiadásakor.
        
    \end{itemize}

    \item Drop <person> <item>
    \begin{itemize}
        \item Leírás: Az első paraméterként megadott Person leszármazott objektum eldobja a második paraméterként megadott Item leszármazott objektumot, ha az nála van.
    \end{itemize}

    \item Link <transistor1> <transistor2>
    \begin{itemize}
        \item Leírás: Összelinkeli a két megadott Transistor objektumot.
    \end{itemize}

    \item Merge <room1> <room2>
    \begin{itemize}
        \item Leírás: Beleolvasztja az első paraméterül megadott Room objektumba a második paraméterül megadott Room objektumot.
    \end{itemize}
    
    \item Move <person> <door>
    \begin{itemize}
        \item Leírás: Az első paraméterként megadott Person leszármazott objektum használja a második paraméterként megadott Door objektumot.
    \end{itemize}
    
    \item Oneway <door> <value>
    \begin{itemize}
        \item Leírás: Megadott Door objektum egyirányúra állítása
        \item Opciók: value egy boolean, lehet true vagy false, hogy egyirányú-e az ajtó.
    \end{itemize}
    
    \item Pickup <person> <item>
    \begin{itemize}
        \item Leírás: Egy ember felvesz egy tárgyat.
        \item Első paraméter: Az első paraméter a Person osztály valamely leszármazottjának, tehát az alábbi osztályok példányának neve. 
        \begin{itemize}
            \item Janitor
            \item Student
            \item Teacher
        \end{itemize}
        \item Második paraméter: A második paraméter lehet az Effect vagy Item osztályok leszármazottjának példánya.
        \begin{itemize}
            \item AirFreshener
            \item Beer
            \item BeerEffect
            \item Camembert
            \item CleanEffect
            \item CodeOfStudies
            \item FakeCodeOfStudies
            \item FakeMask
            \item FakeSlideRule
            \item GasEffect
            \item JanitorEffect
            \item Mask
            \item MaskEffect
            \item RagEffect
            \item SlideRule
            \item Transistor
            \item WetRag
        \end{itemize}
    \end{itemize}
    
    \item Seed <seed>
    \begin{itemize}
        \item Leírás: Véletlen szám generálásához használt seed beállítása. Ezzel determinisztikussá tehető minden véletlen esemény a tesztelés elősegítése érdekében.
    \end{itemize}

    \item Split <room>
    \begin{itemize}
        \item 
        Leírás: Szétválasztja a paraméterül megadott Room objektumbot.
    \end{itemize}

    \item Status <object>
    \begin{itemize}
        \item Leírás: Kiiratja a megadott objektum aktuális állapotát.
    \end{itemize}
    
    \item Use <person> <item>
    \begin{itemize}
        \item Leírás: Az első paraméterként megadott Person leszármazott objektum használja a második paraméterként megadott Item leszármazott objektumot, ha az nála van.
    \end{itemize}
    
    \item Update <deltaTime> [<object>]
    \begin{itemize}
        \item Leírás: Meghívja a paraméterül megadott objektum frissítő metódusát.
        \item Első paraméter: az előző Update óta eltelt idő.
        \item Második paraméter: Egy Updatable interfészt megvalósító objektum:
        \begin{itemize}
            \item BeerEffect
            \item CleanEffect
            \item GasEffect
            \item Janitor
            \item JanitorEffect
            \item MaskEffect
            \item RagEffect
            \item Room
            \item Student
            \item Teacher
        \end{itemize}
        \item Opciók: Az object nem kötelező paraméter, ha nincs megadva, akkor minden objektumot frissít (ahogyan ez játék közben is történne).
    \end{itemize}
\end{itemize}

Példák a bemeneti nyelvre:
\begin{verbatim}
    Create Room r1
    Create Room r2
    Create Door d r1 r2
    Status d

    Create Student s
    Create Mask m
    Add s m
    Update
\end{verbatim}

% \comment{Ha szükséges, meg kell adni a konfigurációs (pl. pályaképet megadó) fájlok nyelvtanát is.}

\subsubsection{Kimeneti nyelv}
% \comment{Egyértelműen definiálni kell, hogy az egyes bemeneti parancsok végrehajtása után előálló állapot milyen formában jelenik meg a szabványos kimeneten. A program képes legyen olyan kimenetet előállítani, amellyel az objektumok állapota ellenőrizhető (pl. save). Ebben az alfejezetben is precízen definiálni kell, hogy a kimenet nyelve milyen elemekből és milyen szintakszissal áll elő.}

Az alábbi parancsok végrehajtása során a futtatás sikeressége vagy eredmény a kimeneten megjelenik.

\begin{itemize}
    \item Merge
    \begin{itemize}
        \item Van elég hely az új szobában: Nincs kimenet
        \item Nincs elég hely az új szobában: Minden olyan emberhez, akinek nincs helye, egy sor fog tartozni a kimenetben. Ha át tudott menni egy szomszédos szobába, az alábbi sor jelenik meg:
        \begin{verbatim}
    <person> moved to <room>.
        \end{verbatim}
        Amennyiben már nincs szomszédos szoba, melybe át lehet menni, az ember meghal:
        \begin{verbatim}
    <person> died.
        \end{verbatim}
    \end{itemize}

    \item Move
    \begin{itemize}
        \item Van elég hely az ajtó túloldalán, és az ajtó járható erre:
        \begin{verbatim}
    <person> moved to <room>.
        \end{verbatim}
        \item Nem járható az ajtó ebben az irányban:
        \begin{verbatim}
    <person> couldn't use the door.
        \end{verbatim}
        \item Járható az ajtó, de nincs hely a túloldalon:
        \begin{verbatim}
    The other room is full.
        \end{verbatim}
    \end{itemize}

    \item Pickup
    \begin{itemize}
        \item Sikerült felvenni: Nincs kimenet.
        \item Nem sikerült felvenni, tele volt az inventory:
        \begin{verbatim}
    Inventory is full.
        \end{verbatim}
    \end{itemize}

    \item Update
    \begin{itemize}
        \item Paraméter nélküli hívás esetén a kimenet megegyezik azzal, mintha az összes szobára létrehozási sorrendben meghívnánk az Update-et.
        \item Room típusú paraméter esetén először az effektusokon megy végig, amennyiben az effektus ideje lejárt, azt az alábbi módon kiírja:
        \begin{verbatim}
    <effect> ran out of time.
        \end{verbatim}
        Ezen kívül a CleanEffect és a JanitorEffect a szobára is alkalmazható, ekkor az alábbi üzenet jelenik meg:
        \begin{verbatim}
    <cleanEffect> cleaned the room.
        \end{verbatim}
        Ha ezalatt a mérgező gázt is eltávolították, utána ez fog megjelenni:
        \begin{verbatim}
    <cleanEffect> removed <gasEffect>.
        \end{verbatim}
        Ezután az emberek következnek, rájuk kerülnek a szoba effektusai:
        \begin{verbatim}
    <person> got kicked out.
    <person> got knocked out.
    <teacher> became peaceful.
    <student> was saved by beer.
        \end{verbatim}
        Majd az effekt rákerülése után az emberek saját maguk is frissülnek:
        \begin{verbatim}
    <student> was protected by mask.
    <teacher> attacked everyone.
    <student> was eliminated.
    <student> was revived.
        \end{verbatim}
    \end{itemize}
    


    \item Status
    \begin{itemize}
        \item A parancs hatására a program kiírja a kért objektum adatatait. Egy sorba kiírja az osztályt és nevet. Ezután a belső állapotának változóit írja ki külön sorokba. Ha egy belső változó egy kollekció akkor ennek elemeit írja ki szóközzel elválasztva. (Ha az elemek objektumok, akkor a prototípusban szereplő nevüket írja ki.) 
        \newline A kimenet különböző osztályok esetén:
    \begin{itemize}
        \item AirFreshener:

        \begin{verbatim}
        AirFreshener <name>
        Person: <person>
        Room: <room>
        \end{verbatim}

        \item Beer:

        \begin{verbatim}
        Beer <name>
        Person: <person>
        Room: <room>
        \end{verbatim}

        \item BeerEffect:

        \begin{verbatim}
        BeerEffect <name>
        Holder: <holder>
        Time remaining: <timeRemaining>
        \end{verbatim}

        \item Camembert:

        \begin{verbatim}
        Camembert <name>
        Person: <person>
        Room: <room>
        \end{verbatim}

        \item CleanEffect:

        \begin{verbatim}
        CleanEffect <name>
        Holder: <holder>
        \end{verbatim}

        \item CodeOfStudies:

        \begin{verbatim}
        CodeOfStudies <name>
        Person: <person>
        Room: <room>
        Uses: <uses>
        \end{verbatim}
        
        \item Door:
        
        \begin{verbatim}
        Door <name>
        Hidden: <hidden>
        One-way: <oneWay>
        Room 1: <room1>
        Room 2: <room2>
        \end{verbatim}

        \item FakeCodeOfStudies:

        \begin{verbatim}
        FakeCodeOfStudies <name>
        Person: <person>
        Room: <room>
        \end{verbatim}

        \item FakeMask:

        \begin{verbatim}
        FakeMask <name>
        Person: <person>
        Room: <room>
        \end{verbatim}

        \item FakeSlideRule:

        \begin{verbatim}
        FakeSlideRule <name>
        Person: <person>
        Room: <room>
        \end{verbatim}

        \item GasEffect:

        \begin{verbatim}
        GasEffect <name>
        Holder: <holder>
        Time remaining: <timeRemaining>
        \end{verbatim}

        \item Janitor:

        \begin{verbatim}
        Janitor <name>
        Effects: <effects>
        Inventory: <items in inventory>
        Knock-out time: <knockOutTime>
        \end{verbatim}

        \item JanitorEffect:

        \begin{verbatim}
        JanitorEffect <name>
        Holder: <holder>
        \end{verbatim}

        \item Mask:

        \begin{verbatim}
        Mask <name>
        Person: <person>
        Room: <room>
        Uses: <uses>
        \end{verbatim}

        \item MaskEffect:

        \begin{verbatim}
        MaskEffect <name>
        Holder: <holder>
        Time remaining: <timeRemaining>
        \end{verbatim}

        \item RagEffect:

        \begin{verbatim}
        RagEffect <name>
        Holder: <holder>
        Time remaining: <timeRemaining>
        \end{verbatim}

        \item Room:

        \begin{verbatim}
        Room <name>
        Capacity: <capacity>
        Doors: <doors>
        Effects: <effects>
        Items: <items>
        People: <people>
        \end{verbatim}

        \item SlideRule:

        \begin{verbatim}
        SlideRule <name>
        Person: <person>
        Room: <room>
        \end{verbatim}

        \item Student:

        \begin{verbatim}
        Student: <name>
        Effects: <effects>
        Eliminated: <eliminated>
        Inventory: <items in inventory>
        Knock-out time: <knockOutTime>
        Room: <room>
        \end{verbatim}

        \item Teacher:

        \begin{verbatim}
        Teacher: <name>
        Effects: <effects>
        Inventory: <items in inventory>
        Knock-out time: <knockOutTime>
        Room: <room>
        \end{verbatim}

        \item Transistor:

        \begin{verbatim}
        Transistor <name>
        Pair: <other>
        Person: <person>
        Room: <room>
        Target room: <target>
        \end{verbatim}

        \item WetRag:

        \begin{verbatim}
        WetRag <name>
        Person: <person>
        Room: <room>
        \end{verbatim}

        
    \end{itemize}

    \item Egyéb:

    Amikor egy hallgató felvette a logarlécet, az alábbi szöveg jelenik meg:

    \begin{verbatim}
        The students have won.
    \end{verbatim}

    \end{itemize}

\end{itemize}

\subsection{A teszttervek}
Ebbe a dokumentumba csak azokat a tesztterveket raktuk bele, melyeken változtatnunk kellett.

\test{ApplyMaskEffectToStudent}{A teszt létrehoz egy Diákot és egy Maszkot. \newline A Maszkot a diákkal felveteti. \newline Ellenőrzi, hogy a diák rendelkezik-e a megfelelő paraméterű MaskEffect-el.}{A Maszk által hordozott Effect Diákra való terjedésének tesztelése.}

Megjegyzés: A fenti két teszt elvégezhető lenne az összes tárgyra és szereplőre is, de mivel az ősosztályaik közösek és a FakeItem implementációknál metódusok kerülnek törlésre, ezért ezekre külön teszteket nem terveztünk.

\test{UpdateBeerEffect}{A teszt létrehoz egy Diákot és egy Beer effectet. \newline Az effecet a diákra rakja. Meghívja a Diák update függvényét. \newline Ellenőrzi, hogy frissült-e az effect.}{Az effect frissítésének ellenőrzése.}

\test{UpdateRagEffect}{A teszt létrehoz egy Diákot és egy RagEffectet. \newline Az effectet a diákra rakja. Meghívja a Diák update függvényét. \newline Ellenőrzi, hogy frissült-e az effect.}{Az effect frissítésének ellenőrzése.}

\test{UpdateMaskEffect}{A teszt létrehoz egy Diákot és egy Maszkot. \newline A Maszokt a diákkal felveteti. Meghívja a Diák update függvényét. \newline Ellenőrzi, hogy frissült-e az effect.}{Az effect frissítésének ellenőrzése.}

\test{UpdateGasEffect}{A teszt létrehoz egy Szobát, egy Diákot és egy Camember-et. \newline A Camembert-et felveteti és kibontatja a Diákkal. Meghívja a Szoba update függvényét. \newline Ellenőrzi, hogy frissült-e az effect.}{Az effect frissítésének ellenőrzése.}

\test{PickUpItem}{A teszt létrehoz egy szobát, egy hallgatót és egy Sört \newline A hallgatót és a Sört a szobába helyezi. A hallgatóval felveteti a Sört. \newline Ellenőrzi, hogy a Sör a hallgatóhoz került-e.}{A felvétel funkció ellenőrzése.}
Megjegyzés: Elegendő egy tetszőleges típusú Player-el és egy tetszőleges típusú tárgyal tesztelni, mivel a teszetesethez releváns metódusok az absztrakt osztályokban kerültek definiálásra.

\test{TeleportUsingTransistors}{A teszt létrehoz két szobát(szobaA, szobaB), két tranzisztort(tranzisztorA, tranzisztorB) és egy hallgatót. \newline A hallgatót elhelyezi szobaA-ban. A tranzisztorokat összelinkeli. TranzisztorA-t odaadja a hallagtónak, tranzisztorB-t pedig elhelyezi szobaB-ben. A hallgatót teleportáltatja TranzisztorA segítségével. \newline Ellenőrzi, hogy a diák átkerült-e szobaB-be. }{A teleportálás funkció működésének bizonyítása.}

\test{BeerProtectStudent}{A teszt létrehoz egy szobát, egy Hallgatót és egy Oktatót egy Sört és egy maszkot. \newline A Sört és a maszkot felveteti a Hallgatóval. A Hallgatót és Oktatót a Szobába rakja. \newline Ellenőrzi, hogy az oktató nem támadja meg a Hallgatót.}{A Sör védelemnyújtásának ellenőrzése, miközben a hallgató elejti a nála lévő másik tárgyat.}

\test{KnockOutWithOutMask}{A teszt létrehoz egy Szobát, egy Diákot és egy Camember-et. \newline A Diákot a szobába helyezi. A diákkal felveteti és kibontatja a sajtot \newline Ellenőrzzi, hogy a Diák elájult-e.}{A Cammabret működésének ellenőrzése.}
Megjegyzés: Ezt a tesztesetet a Teacher és Janitor objektumra is elvégezhetnénk, de a tesztben releváns metódusok a Student és Teacher esetében azonosak, mivel mind a hárman a Person osztály leszármazottai.

\test{KnockOutWithMask}{A teszt létrehoz egy Szobát, egy Diákot, Maszkot és egy Camember-et. \newline A Maszkot a Diákra helyezi. A Camembert-et felveteti és kibontatja a Diákkal. A Diákot a szobába helyezi.\newline Ellenőrzzi, hogy a Diák nem elájult}{A Maszk működésének ellenőrzése.}
Megjegyzés: Ezt a tesztesetet a Teacher és Janitor objektumra is elvégezhetnénk, de a tesztben releváns metódusok a Student és Teacher esetében azonosak, mivel mind a hárman a Person osztály leszármazottai.

\clearpage

\section{Osztályok és metódusok tervei}

\subsection{AirFreshener}
\begin{class-template-responsibility}
    A tárgy használatakor a szoba levegőjét megtisztítja.
\end{class-template-responsibility}
\begin{class-template-interface}
    -
\end{class-template-interface}
\begin{class-template-baseclass}
    Item
\end{class-template-baseclass}
\begin{class-template-attribute}
    -
\end{class-template-attribute}
\begin{class-template-method}
   \classitem{+use() : void}{Létrehoz egy CleanEffectet a szobára, majd önmegsemimisít.}
\end{class-template-method}

\subsection{Beer}
\begin{class-template-responsibility}
    Megvédi a hallgatót az oktatóktól.
\end{class-template-responsibility}
\begin{class-template-interface}
    -
\end{class-template-interface}
\begin{class-template-baseclass}
    Item
\end{class-template-baseclass}
\begin{class-template-attribute}
    -
\end{class-template-attribute}
\begin{class-template-method}
    \classitem{+use() : void}{Egy BeerEffectet rak a személyre, akinél van, majd önmegsemimisít. Használakor a hallgató elejt egy nála lévő tárgyat.}
    \begin{verbatim}
        create beerEffect
        add beerEffect to holder
        remove self from the holder
        make the holder drop a random item
    \end{verbatim}
    .
\end{class-template-method}

\subsection{BeerEffect}
\begin{class-template-responsibility}
    A Beer hatásának megvalósítása: védelem az oktatókkal szemben.
\end{class-template-responsibility}
\begin{class-template-interface}
    Updatable
\end{class-template-interface}
\begin{class-template-baseclass}
    Effect
\end{class-template-baseclass}
\begin{class-template-attribute}
    -
\end{class-template-attribute}
\begin{class-template-method}
    \classitem{+applyToStudent(target : Student) : void}{Megvédi a  játékost az oktatóktól, közben a játékos elejt egy nála lévő tárgyat.}
\end{class-template-method}

\subsection{Camembert}
\begin{class-template-responsibility}
    A tárgy használatakor a szoba levegőjét mérgzővé teszi.
\end{class-template-responsibility}
\begin{class-template-interface}
    -
\end{class-template-interface}
\begin{class-template-baseclass}
    Item
\end{class-template-baseclass}
\begin{class-template-attribute}
    -
\end{class-template-attribute}
\begin{class-template-method}
   \classitem{+use() : void}{Egy GasEffectet rak a szobára, ahol van, majd önmegsemimisít.}
\end{class-template-method}

\subsection{CleanEffect}
\begin{class-template-responsibility}
    A mérgező szobák levegőjét megtisztítja.
\end{class-template-responsibility}
\begin{class-template-interface}
    Updatable
\end{class-template-interface}
\begin{class-template-baseclass}
    Effect
\end{class-template-baseclass}
\begin{class-template-attribute}
    -
\end{class-template-attribute}
\begin{class-template-method}
    \classitem{+applyToRoom(target : Room) : void}{A szoba levegőjét megtisztíttatja.}
\end{class-template-method}

\subsection{CodeOfStudies}
\begin{class-template-responsibility}
    A hallgató használhatja, hogy megvédje magát egy oktatóval szemben.
\end{class-template-responsibility}
\begin{class-template-interface}
    -
\end{class-template-interface}
\begin{class-template-baseclass}
    Item
\end{class-template-baseclass}
\begin{class-template-attribute}
    \classitem{-uses : int }{A tárgy használatának száma.}
\end{class-template-attribute}
\begin{class-template-method}
   \classitem{+useAgainst(target : Teacher) : void}{Megvédi a birtokosát az oktatótól. Ha a $uses$ 0-ra csökken megsemmisül.}
\end{class-template-method}

\subsection{Door}
\begin{class-template-responsibility}
    Ajtó. Két Room között megy. Lehet egyirányú vagy kétirányú. Rajta keresztül lépnek át a személyek a szobákba.
\end{class-template-responsibility}
\begin{class-template-interface} -
\end{class-template-interface}
\begin{class-template-baseclass} -
\end{class-template-baseclass}
\begin{class-template-attribute}
    \classitem{-from : Room}{Első szoba, ahova az ajtó nyílik.}
    \classitem{-to : Room}{Második szoba, ahova az ajtó nyílik.}
    \classitem{-oneWay : boolean}{Ha igaz, egyirányú az ajtó. Ilyen esetben csak\newline$from\longrightarrow to$ irányban lehet átlépni rajta. Alapértelmezetten false.}
    \classitem{-hidden : boolean}{Ha igaz, rejtve van az ajtó. Ilyen esetben nem lehet rajta átlépni.}
\end{class-template-attribute}
\begin{class-template-method}
    \classitem{+Door(from : Room, to : Room)}{Konstruktor, ami megadja a szobákat. A szobákhoz hozzáadja az ajtót.}
    \classitem{+use(person : Person, useFrom : Room) : void}{Ajtó használata: a személy átlép a másik szobába, ha teheti. $person$: a személy, aki használja az ajtót. $useFrom$: a szoba ahonnan jön a személy. Rejtett ajtón nem lehet átlépni. A $useFrom$ paraméternek egyeznie kell az ajtóhoz tartozó egyik szobával. Ha az ajtó egyirányú, akkor csak abban az esetben lehet átlépni rajta, ha a $useFrom$ paraméter megegyezik az ajtó $from$ attribútumával. Ha az ajtón át tud lépni, akkor megpróbálja beléptetni a $person$-t a másik szobába. Ha az átlépés sikeres, a person eltávolítódik a kiinduló szobából.}
    \begin{verbatim}
        if hidden:
            return
        if oneWay and (the direction is not correct)
            return
        if entering the other room is successful:
            leave the previous room
    \end{verbatim}
    \classitem{+move(moveFrom : Room, moveTo : Room) : void}{Ajtó mozgatása: Ha a $moveFrom$ paraméter egyezik az ajtó egyik szobájával, akkor ehelyett a szoba helyett a $moveTo$ lesz az a szobája. Ha az áthelyezés miatt az ajtó mindkét szobája ugyan az lenne, akkor az ajtó törlődik.}
    \begin{verbatim}
        if moveTo = from or moveTo = to:
            remove door from moveTo
        else:
            remove door from moveFrom
            if moveFrom = from:
                from = moveTo
            else if moveFrom = to
                to = moveFrom
            add this door to moveTo
    \end{verbatim}
    \classitem{+hide() : void}{A $hidden$ attribútum hamisra állítása. }
    \classitem{+show() : void}{A $hidden$ attribútum igazra állítása. }
    \classitem{+setOneWay(oneWay : boolean) : void}{Setter.}
\end{class-template-method}
\begin{class-template-statechart}
    \diagram{img/diagrams/state/door}{Ajtó állapota}{14cm}
\end{class-template-statechart}

\subsection{Effect}
\begin{class-template-responsibility}
    A játék objektumainak hatását megvalósító effectek ősosztálya.
\end{class-template-responsibility}
\begin{class-template-interface}
    Updatable
\end{class-template-interface}
\begin{class-template-baseclass}
    - 
\end{class-template-baseclass}
\begin{class-template-attribute}
    \classitem{- timeRemaining : double}{Mennyi ideig érvényes még a hatás.}
    \classitem{- holder : GameObject}{Megadja a hatás birtokosát.}
\end{class-template-attribute}
\begin{class-template-method}
    \classitem{+applyToStudent(target : Student) : void}{A hatás alkalmazása a hallgatóra.}
    \classitem{+applyToTeacher(target : Teacher) : void}{A hatás alkalmazása az oktatóra.}
    \classitem{+applyToRoom(room : Room) : void}{A hatás alkalmazása a szobára.}
    \classitem{+interactCleanEffect(cleanEffect : CleanEffect) : void}{A tisztasg hatás.}
    \classitem{+setHolder(holder : GameObject) : void}{Beállítja a hatás birtokosát.}
\end{class-template-method}

\subsection{FakeCodeOfStudies}
\begin{class-template-responsibility}
    A TVSZ hamis megfelelője, a felvételekor nem történik semmi, nem használható semmire.
\end{class-template-responsibility}
\begin{class-template-interface}
    -
\end{class-template-interface}
\begin{class-template-baseclass}
    Item\baseclass CodeOfStudies
\end{class-template-baseclass}
\begin{class-template-attribute}
    -
\end{class-template-attribute}
\begin{class-template-method}
   \classitem{+useAgainst(target : Teacher) : void}{Nem történik semmi.}
\end{class-template-method}

\subsection{FakeMask}
\begin{class-template-responsibility}
    A Mask hamis megfelelője, a felvételekor nem történik semmi, nem használható semmire.
\end{class-template-responsibility}
\begin{class-template-interface}
    -
\end{class-template-interface}
\begin{class-template-baseclass}
    Item\baseclass Mask
\end{class-template-baseclass}
\begin{class-template-attribute}
    -
\end{class-template-attribute}
\begin{class-template-method}
    \classitem{+usePassive() : void}{Nem történik semmi.}
\end{class-template-method}

\subsection{FakeSlideRule}
\begin{class-template-responsibility}
    A Logarléc hamis megfelelője, a felvételekor nem történik semmi.
\end{class-template-responsibility}
\begin{class-template-interface}
    -
\end{class-template-interface}
\begin{class-template-baseclass}
    Item \baseclass SlideRule
\end{class-template-baseclass}
\begin{class-template-attribute}
    -
\end{class-template-attribute}
\begin{class-template-method}
   \classitem{+setPerson(person : Person) : void}{Nem történik semmi.}
\end{class-template-method}

\subsection{GasEffect}
\begin{class-template-responsibility}
    A mérgező szobák mérgező levegőjének megvalósítása: emberek elájulását okozza.
\end{class-template-responsibility}
\begin{class-template-interface}
    Updatable
\end{class-template-interface}
\begin{class-template-baseclass}
    Effect
\end{class-template-baseclass}
\begin{class-template-attribute}
    -
\end{class-template-attribute}
\begin{class-template-method}
     \classitem{+applyToStudent(target : Student) : void}{Kiüti a hallgatót meghatározott időre.}
     \classitem{+applyToTeacher(target : Teacher) : void}{Kiüti az oktatót meghatározott időre.}
     \classitem{+interactCleanEffect(cleanEffect : CleanEffect) : void}{Eltávolítja az effectet a szobáról.}
\end{class-template-method}

\subsection{GameObject}
\begin{class-template-responsibility}
    Ősosztály az olyan játékban szereplő objektumoknak, amelyek Effecteket és Itemeket tárolnak. 
\end{class-template-responsibility}
\begin{class-template-interface}
    Updatable
\end{class-template-interface}
\begin{class-template-baseclass} -  
\end{class-template-baseclass}
\begin{class-template-attribute}
    \classitem{\#effects [0..*] : Effect}{Effektek}
\end{class-template-attribute}
\begin{class-template-method}
    \classitem{+addItem(item : Item) : void}{Tárgy hozzáadása az objektumhoz.}
    \classitem{+removeItem(item : Item) : void}{Tárgy eltávolítása az objektumról.}
    \classitem{+addEffect(effect : Effect) : void}{Hatás hozzáadása az objektumhoz.}
    \classitem{+removeEffect(effect : Effect) : void}{Hatás eltávolítása az objektumról.}
    \classitem{+applyEffect(effect : Effect) : void}{Hatás alkalmazása az objektumra.}
    \classitem{+interactTeacher(teacher : Teacher) : void}{Absztrakt. Az objektum interakciója oktatóval.}
\end{class-template-method}

\subsection{Inventory}
\begin{class-template-responsibility}
    Felszerelés. Egy személynél van. Ez tárolja a személy tárgyait.
\end{class-template-responsibility}
\begin{class-template-interface}-
\end{class-template-interface}
\begin{class-template-baseclass} -
\end{class-template-baseclass}
\begin{class-template-attribute}
    \classitem{-items : Item[0..5]}{A személy tárgyai.}
\end{class-template-attribute}
\begin{class-template-method}
    \classitem{+add(item : Item) : boolean}{Az $item$ hozzáadása az $items$-hez, ha az kevesebb, mint 5 tárgyat tartalmaz. Ha a művelet sikeres volt igazat ad vissza, különben hamisat.}
    \classitem{+remove(item : Item) : void}{Eltávolítja az $item$-et az $items$-ből, ha az benne volt.}
    \classitem{+setRoom(room : Room) : void}{Az $items$ minden elemére meghívja a $setRoom(room)$ metódust.}
    \classitem{+protectFrom(teacher : Teacher) : void}{Az $items$ minden elemére meghívja a $useAgainst(teacher)$ metódust.}
    \classitem{+dropRandomItem() : void}{ Véletlenszerű elem eldobása az $items$-ből. }
\end{class-template-method}

\subsection{Item}
\begin{class-template-responsibility}
    Ősosztály biztosítása a játék tárgyai számára.
\end{class-template-responsibility}
\begin{class-template-interface}
       -
\end{class-template-interface}
\begin{class-template-baseclass}
    - 
\end{class-template-baseclass}
\begin{class-template-attribute}
    \classitem{\#room : Room}{A szoba ahol a tárgy van.}
    \classitem{\#person : Person}{Az ember akinél a tárgy van.}
\end{class-template-attribute}
\begin{class-template-method}
    \classitem{+use() : void}{A tárgyat használja valaki.}
    \classitem{+useAgainst(target : Teacher) : void}{A tárgyat egy oktató ellen használják.}
    \classitem{+usePassive() : bool}{A tárgy működését a környezet váltja ki.}
    \classitem{+useItem(item : Item) : void}{A tárgyy használata egy másik tárggyal.}
    \classitem{+link(other : Transistor) : void}{Két tranzisztor összekapcsolása}
    \classitem{+drop() : void}{A tárgy eldobódik.}
    \classitem{+setRoom(room : Room) : void}{A tárgynak beállítódik a szoba, ahol van.}
    \classitem{+setPerson(person : Person) : void}{A tárgynak beállítódik az ember, akinél van.}
\end{class-template-method}

\subsection{Janitor}
\begin{class-template-responsibility}
    A mérgező szobák hatását szüntetit meg, a szobákat tisztává teszi és kirakja a Student és Teacher objektumokat. 
\end{class-template-responsibility}
\begin{class-template-interface}
    Updatable
\end{class-template-interface}
\begin{class-template-baseclass}
    GameObject \baseclass Person
\end{class-template-baseclass}
\begin{class-template-attribute}
    -
\end{class-template-attribute}
\begin{class-template-method}
    \classitem{+enterRoom(room : Room) : void}{Belép a szobába és létrehoz egy JanitorEffectet a szobán.}
    \classitem{+interactTeacher(teacher : Teacher) : void}{Nem csinál semmit}
    \classitem{+protectFromTeacher(teacher : Teacher) : void}{Nem csinál semmit}
    \classitem{+applyEffect(effect : Effect) : void}{Nem csinál semmit}
    \classitem{+pickedUpSlideRule() : void}{Nem csinál semmit}
\end{class-template-method}

\subsection{JanitorEffect}
\begin{class-template-responsibility}
    A szobákat megtisztítja és kirakja a benne tartózkodókat.
\end{class-template-responsibility}
\begin{class-template-interface}
    Updatable
\end{class-template-interface}
\begin{class-template-baseclass}
    Effect\baseclass CleanEffect
\end{class-template-baseclass}
\begin{class-template-attribute}
    -
\end{class-template-attribute}
\begin{class-template-method}
    \classitem{+applyToStudent(target : Student) : void}{Kirakja a szobából a hallgatót}
    \classitem{+applyToTeacher(target : Teacher) : void}{Kirakja a szobából az oktatót.}
    \classitem{+applyToRoom(target : Room) : void}{Megtisztítja a szobát.}
\end{class-template-method}

\subsection{Mask}
\begin{class-template-responsibility}
    Megvédi a hallgatót az mérgező levegőjű szobkáktól.
\end{class-template-responsibility}
\begin{class-template-interface}
    -
\end{class-template-interface}
\begin{class-template-baseclass}
    Item
\end{class-template-baseclass}
\begin{class-template-attribute}
    \classitem{-uses : int}{Megadja hányszor használták a maszkot.}
    \classitem{-effect : MaskEffect}{A tárgy által kifejtett hatás, ha éppen aktív.}
\end{class-template-attribute}
\begin{class-template-method}
    \classitem{+usePassive() : void}{Akkor hívódik meg, ha birtokosa gázzal teli szobában van és épp nincs rajta MaskEffect, ekkor új MaskEffect-et alkalmaz birtokosára. Ha a $uses$ 0-ra csökken, megsemmisül.}
    \classitem{+drop() : void}{Amikor valaki eldobja a maszkot, ha a hatása éppen aktív volt, az megszűnik.}
\end{class-template-method}

\subsection{MaskEffect}
\begin{class-template-responsibility}
    A Mask hatásának megvalósítása: védelem a mérgező levegővel szemben.
\end{class-template-responsibility}
\begin{class-template-interface}
    Updatable
\end{class-template-interface}
\begin{class-template-baseclass}
    Effect
\end{class-template-baseclass}
\begin{class-template-attribute}
    -
\end{class-template-attribute}
\begin{class-template-method}
    \classitem{+applyToStudent(target : Student) : void}{Megvédi a hallgatót a mérgező gáztól.}
    \classitem{+applyToTeacher(target : Teacher) : void}{Megvédi az oktatót a mérgező gáztól.}
\end{class-template-method}

\subsection{Person}  
\begin{class-template-responsibility}
    A játékban szereplő emberek közös őse.
\end{class-template-responsibility}
\begin{class-template-interface}
    Updatable
\end{class-template-interface}
\begin{class-template-baseclass}
    GameObject
\end{class-template-baseclass}
\begin{class-template-attribute}
    \classitem{-name : String}{A személy neve.}
    \classitem{-currentRoom : Room}{A szoba, ahol aktuálisan van az ember.}
    \classitem{-knockOutTime : double}{Mennyi idő van hátra az ájult állapotból. Ha 0, akkor nincs ájult állapot.}
    \classitem{\#inventory : Inventory}{A felszerelés. Ez tárolja az ember tárgyait.}
\end{class-template-attribute}
\begin{class-template-method}
    \classitem{+protectFromTeacher(target : Teacher) : void}{Absztrakt. Személy reagál arra, hogy oktatóval találkozik.}
    \classitem{+enterRoom(room : Room) : void}{Frissíti a $currentRoom$-ot és az $inventory$-t is értesíti a szoba váltásról.}
    \classitem{+dropItem(item : Item) : void}{Tárgy eldobása: az $item$-et eltávolítja az $inventory$-ból, és a $currentRoom$-hoz adja.}
    \classitem{+setKnockout(value : double) : void}{Setter.}
    \classitem{+update(deltaTime : double) : void}{Az $effects$ összes elemére: először frissíti, majd alkalmazza magára.}
    \classitem{+addItem(item : Item) : void}{Az $item$ hozzáadása az $inventory$-hoz és az $item$ person-ja ez az objektum lesz.}
    \classitem{+removeItem(item : Item) : void}{Az $item$ eltávolítása az $inventory$-ból.}
    \classitem{+pickedUpSlideRule() : void}{Absztrakt. A személy felvette a logarlécet, és erre reagál.}
    \classitem{+getOut() : void}{A személynek el kell hagynia a szobát.}
    \classitem{+dropRandomItem() : void}{Véletlenszerű tárgy eldobása az $inventory$-ból.}
\end{class-template-method}
\begin{class-template-statechart}
    \diagram{img/diagrams/state/person}{Ember állapota}{14cm}
\end{class-template-statechart}

\subsection{RagEffect}
\begin{class-template-responsibility}
    A WetRag hatásának megvalósítása: oktatók megbénítása.
\end{class-template-responsibility}
\begin{class-template-interface}
    Updatable
\end{class-template-interface}
\begin{class-template-baseclass}
    Effect
\end{class-template-baseclass}
\begin{class-template-attribute}
    -
\end{class-template-attribute}
\begin{class-template-method}
    \classitem{applyToTeacher(target : Teacher) : void}{Megbénítja az oktatót.}
\end{class-template-method}

\subsection{Room}
\begin{class-template-responsibility}
    Szoba. Személyek, tárgyak, effektek találhatók benne. Ezek a Room-on keresztül lépnek kapcsolatba, és a Room értesíti őket az idő múlásáról.
\end{class-template-responsibility}
\begin{class-template-interface}
    Updatable
\end{class-template-interface}
\begin{class-template-baseclass}
    GameObject
\end{class-template-baseclass}
\begin{class-template-attribute}
    \classitem{-people [0..*] : Person}{Személyek}
    \classitem{-items [0..*] : Item}{Tárgyak}
    \classitem{-doors [0..*] : Door}{Ajtók}
    \classitem{-capacity : int}{Kapacitás : ennél több személy nem lehet a szobában}
    \classitem{-visitorsSinceClean : int}{A legutóbbi tisztítás óta mennyien látogatták a szobát. Ha valaki belép a szobába ez eggyel nő.}
\end{class-template-attribute}
\begin{class-template-method}
    \classitem{+Room()}{Konstruktor, ami alapértelmezett kapacitású, üres szobát hoz létre, aminek nincs ajtaja.}
    \classitem{+Room(effects : Effect[0..*], capacity : int)}{Konstruktor, ami megadja a hatásokat és a kapacitást. }
    \classitem{+enter(person : Person) : boolean}{Személy megpróbál belépni a szobába. Ha kevesebben vannak a szobában, mint a $capacity$, akkor a művelet sikeres: a $person$ a szobába lép és a visszatérési érték igaz, különben nem történik semmi és a visszatérési érték hamis.}
    \classitem{+leave(person : Person) : void}{A $person$ elhagyja a szobát.}
    \classitem{+merge(room : Room) : void}{A $room$ beleolvasztása ebbe a szobába. Ha a két szobának nincs közös ajtaja, vagy a nagyobbik kapacitás kevesebb, mint a két szobában található személyek összesen, akkor nem történik semmi. Különben a $room$ tartalmát ebbe a szobába helyezi át és kapacitását beállítja a kettő közül a nagyobbra.}
    \classitem{-moveContents(room : Room) : void}{A szoba objektumait (emberek, tárgyak, hatások és azon ajtók, amelyek nem a két szoba köz mennek) a $room$-ba helyezi át.}
    \classitem{+split() : Room}{Létrehoz egy új szobát. Az $items$ és $people$ tartalmának véletlenszerű részét átadja az új szobának. A $doors$ véletlenszerű részét átmozgatja az új szobára. Az $effects$ minden tagjának másolatát átadja az új szobának. Létrehoz egy új ajtót ami ebből a szobából vezet az újba. Végül vissza adja az új szoba objektumot.}
    \begin{verbatim}
        create newRoom
        for item in items:
            get a random integer between 0 and 1
            if random int = 0:
                move the current item to the newRoom
        for person in people:
            get a random integer between 0 and 1
            if randomInt = 0:
                move the current person to the newRoom
        for door in doors:
            get random integer between 0 and 1
            if random int = 0:
                move the current door to the newRoom
        for effect in effects:
            copy the current effect to the newRoom
        create a door between room and newRoom
        return newRoom
    \end{verbatim}
    \classitem{+interactTeacher(teacher : Teacher) : void}{A szobában tartózkodó összes embert értesíti az oktatóval való interakcióról.}
    \classitem{+addDoor(door : Door) : void}{Ajtó hozzáadása.}
    \classitem{+removeDoor(door : Door) : void}{Ajtó eltávolítása.}
    \classitem{+interactCleanEffect(effect : CleanEffect) : void}{Az $effects$ minden elemét értesíti az interakcióról.}
    \classitem{+isClean() : boolean}{Visszaadja, hogy ragadós-e a szoba. Attól függ, a $visitorsSinceClean$ egy adott értéknél nagyobb-e.}
    \classitem{+clean() : void}{A $visitorsSinceClean$ értékét 0-ra állítja.}
    \classitem{+update(deltaTime : double) : void}{Frissíti az $effects$ összes elemét. A $people$ minden elemére először alkalmazza a szoba hatásait, majd frissíti őket.}
    \classitem{+getOut(person : Person) : void}{A személy a doors minden elemét használja, amíg valamelyiken keresztül el nem hagyja a szobát. Ha egyiken kereszül se tud távozni, akkor marad a szobában. A használat sorrendje az ajtók a szobához adásának sorrendje.}
\end{class-template-method}
\begin{class-template-statechart}
    \diagram{img/diagrams/state/room}{Szoba állapota}{14cm}
\end{class-template-statechart}

\subsection{SlideRule}
\begin{class-template-responsibility}
    A felvételével véget ér a játék.
\end{class-template-responsibility}
\begin{class-template-interface}
    -
\end{class-template-interface}
\begin{class-template-baseclass}
    Item
\end{class-template-baseclass}
\begin{class-template-attribute}
    -
\end{class-template-attribute}
\begin{class-template-method}
   \classitem{+setPerson(person : Person) : void}{Értesíti a $person$-t, hogy felvette a logarlécet.}
\end{class-template-method}

\subsection{Student}
\begin{class-template-responsibility}
    A játékosok reprezentálása a játékban.
\end{class-template-responsibility}
\begin{class-template-interface}
    Updatable
\end{class-template-interface}
\begin{class-template-baseclass}
    GameObject \baseclass Person
\end{class-template-baseclass}
\begin{class-template-attribute}
    \classitem{-eliminated: boolean}{Azt adja meg, hogy a hallgató játékban van-e még.}
    \classitem{-immuneToTeacher : Teacher[0..*]}{Oktatók akik ellen TVSZ segítségével védekezett a hallgató.}
\end{class-template-attribute}
\begin{class-template-method}
    \classitem{+setEliminated(value : bool) : void}{Beállítja, hogy a játékos játékban van-e.}
    \classitem{+interactTeacher(teacher : Teacher) : void}{Interakcióba kerül egy oktatóval, megpróbálja megvédeni magát a rá ható hatások és nála lévő tárgyak segítségével.}
    \classitem{+protectFromTeacher(target : Teacher) : void}{Sikeresen megvédte magát az oktatóval szemben TVSZ segítségével.}
    \classitem{+applyEffect(effect : Effect) : void}{Értesíti a hatást, hogy hallgatóra kell alkalmazódnia.}
    \classitem{+pickedUpSlideRule() : void}{A hallgatók nyertek, vége a játéknak.}
\end{class-template-method}

\subsection{Teacher}
\begin{class-template-responsibility}
    Oktatók, a hallgatók lelkét igyekeznek elvenni.
\end{class-template-responsibility}
\begin{class-template-interface}
    Updatable
\end{class-template-interface}
\begin{class-template-baseclass}
    GameObject\baseclass Person
\end{class-template-baseclass}
\begin{class-template-attribute}
    \classitem{-peaceful : boolean}{Azt adja meg, hogy az oktató bénult állapotban van-e.}
\end{class-template-attribute}
\begin{class-template-method}
    \classitem{+interactTeacher(teacher : Teacher) : void}{Nem csinál semmit}
    \classitem{+protectFromTeacher(teacher : Teacher) : void}{Nem csinál semmit}
    \classitem{+setPeaceful(value : boolean) : void}{Az oktatót bénult állapotba helyezi.}
    \classitem{+applyEffect(effect : Effect) : void}{Értesíti az $effect$-et, hogy oktatóra kell alkalmazódnia.}
    \classitem{+update(deltaTime : double) : void}{Az ősosztály szerint elvégzi a frissítést, majd értesíti a szobát, ahol éppen tartózkodik, hogy oktató van ott.}
    \classitem{+pickedUpSlideRule() : void}{Nem csinál semmit}
\end{class-template-method}
\begin{class-template-statechart}
    \diagram{img/diagrams/state/teacher}{Oktató állapota}{14cm}
\end{class-template-statechart}

\subsection{Transistor}
\begin{class-template-responsibility}
    Szobák közötti teleportációra használható.
\end{class-template-responsibility}
\begin{class-template-interface}
    -
\end{class-template-interface}
\begin{class-template-baseclass}
    Item
\end{class-template-baseclass}
\begin{class-template-attribute}
    \classitem{- other : Transistor}{A tranzisztor párja, amivel össze lett kapsolva.}
    \classitem{- target : Room}{A szoba, ahova a tranzosztorokkal el lehet jutni.}
\end{class-template-attribute}
\begin{class-template-method}
    \classitem{setTarget(target : Room) : void}{A tranzisztor-teleportáció célszobájának beállítása}
    \classitem{setPair(pair : Transistor) : void}{A tranzisztor párjának beállítása.}
    \classitem{+useItem(item : Item) : void}{Megpróbálja párosítani magát a másik tárggyal.}
    \classitem{+link(other : Transistor) : void}{Párosítja magát a másik tranzisztorral.}
    \classitem{+use() : void}{A tranzisztor birtokosa használja. Ha nincs párosítva, nem történik semmi. Ha párosítva van, de nem aktív, akkor lehelyeződik és aktiválja párját a cél szoba beállításával. Ha aktív, akkor eldobódik és birtokosát a cél szobába teleportálja.}
    \begin{verbatim}
        if not linked:
            return
        if linked and not active:
            set the current room as target for other
            drop this
        if linked and active:
            drop this
            teleport holder to the target room
    \end{verbatim}
    .
\end{class-template-method}
\begin{class-template-statechart}
\diagram{img/diagrams/state/transistor}{Tranzisztor állapota}{14cm}
\end{class-template-statechart}

\subsection{Updatable}
\begin{class-template-responsibility}
    Interfész, mely lehetőséget ad az őt megvalósító objektumok frissítésére.
\end{class-template-responsibility}
\begin{class-template-interface} -
\end{class-template-interface}
\begin{class-template-baseclass} -
\end{class-template-baseclass}
\begin{class-template-method}
    \classitem{+update(deltaTime : double) : void}{Objektum frissítése: Telt az idő, és erre reagál az objektum. deltaTime: eltelt idő a legutóbbi update óta.}
\end{class-template-method}

\subsection{WetRag}
\begin{class-template-responsibility}
    Megvédi a hallgatót azzal, hogy a szobában lévő oktatókat megbénítja.
\end{class-template-responsibility}
\begin{class-template-interface}
    -
\end{class-template-interface}
\begin{class-template-baseclass}
    Item
\end{class-template-baseclass}
\begin{class-template-attribute}
    \classitem{-effect : RagEffect}{A tárgy által kifejtett hatás.}
\end{class-template-attribute}
\begin{class-template-method}
    \classitem{+setRoom(room :Room) : void}{Amikor a rongyot használó hallgató átmegy egy másik szobába a hatása vele együtt átvándorol.}
\end{class-template-method}

\section{A tesztek részletes tervei, leírásuk a teszt nyelvén}
% \comment{A tesztek részletes tervei alatt meg kell adni azokat a bemeneti adatsorozatokat, amelyekkel a program működése ellenőrizhető. Minden bemenő adatsorozathoz definiálni kell, hogy az adatsorozat végrehajtásától a program mely részeinek, funkcióinak ellenőrzését várjuk és konkrétan milyen eredményekre számítunk, ezek az eredmények hogyan vethetők össze a bemenetekkel.A tesztek leírásakor az előző dokumentumban (proto koncepciója) megadott szintakszist kell használni.}
\subsection{AddDoor}
\begin{test-case-description}
    A teszt célja megbizonyosodni arról, hogy az ajtó elhelyezés két szoba közt megfelelően működik.
\end{test-case-description}
\begin{test-case-function}
    Az ajtó működését vizsgáljuk.
\end{test-case-function}
\begin{test-case-input}
    \begin{verbatim}
    Create Room room1
    Create Room room2
    Create Door door room1 room2
    Status door
    \end{verbatim}
\end{test-case-input}
\begin{test-case-output}
    \begin{verbatim}
    Door door
    Hidden: false
    One-way: false
    Room 1: room1
    Room 2: room2
    \end{verbatim}
\end{test-case-output}

\subsection{ApplyCleanEffectToRoom}
\begin{test-case-description}
    A teszt célja a légfrissítő működésének ellenőrzése.
\end{test-case-description}
\begin{test-case-function}
    A GasEffect eltávolítása légfrissítő által.
\end{test-case-function}
\begin{test-case-input}
    \begin{verbatim}
    Create Student student
    Create AirFreshener freshener
    Create Room room
    Create GasEffect effect
    Add room freshener
    Add room effect
    Add room student
    Pickup student freshener 
    Use student freshener
    Update 1
    Status room
    \end{verbatim}
\end{test-case-input}
\begin{test-case-output}
    \begin{verbatim}
    Room room
    Capacity: 4
    Doors:
    Effects:
    Items:
    People: student 
    \end{verbatim}
\end{test-case-output}

\subsection{ApplyBeerEffectToStudent}
\begin{test-case-description}
    A teszt célja a BeerEffect Diákra való terjedésének ellenőrzése.
\end{test-case-description}
\begin{test-case-function}
    A BeerEffect terjedését vizsgáljuk
\end{test-case-function}
\begin{test-case-input}
    \begin{verbatim}
    Create Student student
    Create Beer beer
    Pickup student beer
    Use student Beer
    Status student
    \end{verbatim}
\end{test-case-input}
\begin{test-case-output}
    \begin{verbatim}
    Student student
    Effects: beerEffect
    Eliminated: false
    Inventory:
    Knock-out time: -5
    Room:
    \end{verbatim}
\end{test-case-output}

\subsection{ApplyGasEffectToRoom}
\begin{test-case-description}
    A teszt célja a Camembert által hordozott Effect szobára való terjedésének tesztelése.
\end{test-case-description}
\begin{test-case-function}
    Az GasEffect terjedését vizsgáljuk.
\end{test-case-function}
\begin{test-case-input}
    \begin{verbatim}
    Create Student student
    Create Camembert cheese
    Create Room room
    Add room student 
    Add room cheese
    Pickup student cheese
    Use student cheese
    Status room
    \end{verbatim}
\end{test-case-input}
\begin{test-case-output}
    \begin{verbatim}
    Room room
    Capacity: 4
    Doors:
    Effects: gasEffect
    Items:
    People: student
    \end{verbatim}
\end{test-case-output}

\subsection{ApplyMaskEffectToStudent}
\begin{test-case-description}
    A teszt célja a MaskEffect Diákra való terjedésének ellenőrzése.
\end{test-case-description}
\begin{test-case-function}
    A MaskEffect terjedését vizsgáljuk
\end{test-case-function}
\begin{test-case-input}
    \begin{verbatim}
    Create Room room 
    Create GasEffect effect
    Create Student student
    Create Mask mask
    Add room student
    Add room effect
    Add room mask
    Pickup student mask
    Update 1
    Status student
    Status mask
    \end{verbatim}
\end{test-case-input}
\begin{test-case-output}
    \begin{verbatim}
    student got knocked out.
    student was protected by mask.
    
    Student student
    Effects: maskeffect
    Eliminated: false
    Inventory: mask
    Knock-out time: -5
    Room: room

    Mask mask
    Person: student
    Room: room
    Uses: 4
    \end{verbatim}
\end{test-case-output}

\subsection{ApplyRagEffectToRoom}
\begin{test-case-description}
    A teszt célja a Táblatörlő által hordozott Effect szobára való terjedésének tesztelése.
\end{test-case-description}
\begin{test-case-function}
    Az tárgy Effect terjedése.
\end{test-case-function}
\begin{test-case-input}
    \begin{verbatim}
    Create WetRag rag
    Create Room room 4
    Add rag room
    Status room
    \end{verbatim}
\end{test-case-input}
\begin{test-case-output}
    \begin{verbatim}
    Room room
    Capacity: 4
    Effects: wetRagEffect
    Items:
    People: student
    \end{verbatim}
\end{test-case-output}

\subsection{AttackStudent}
\begin{test-case-description}
    Az oktató támadásának tesztelése
\end{test-case-description}
\begin{test-case-function}
    A tanár támadása
\end{test-case-function}
\begin{test-case-input}
    \begin{verbatim}
    Create Room room
    Create Teacher teacher
    Create Student student
    Add room teacher
    Add room student
    Update 1
    Status student
    \end{verbatim}
\end{test-case-input}
\begin{test-case-output}
    \begin{verbatim}
    teacher attacked everyone.
    student was eliminated.
    
    Student: student
    Effects: 
    Eliminated: true
    Inventory: 
    Knock-out time: -5
    Room: room
    \end{verbatim}
\end{test-case-output}

\subsection{BeerProtectStudent}
\begin{test-case-description}
    A Söröspohár védelme a tanárral szemben, közben a hallgató elejt egy tárgyat.
\end{test-case-description}
\begin{test-case-function}
    A tanár támadása
\end{test-case-function}
\begin{test-case-input}
    \begin{verbatim}
    Create Room room
    Create Teacher teacher
    Create Student student
    Create Mask mask
    Create Beer beer
    Add room teacher
    Add room beer
    Add room mask
    Add room student
    Pickup student beer
    Pickup student mask
    Update 1
    Status student
    \end{verbatim}
\end{test-case-input}
\begin{test-case-output}
    \begin{verbatim}
    teacher attacked everyone.
    student was eliminated.
    student was revived.
    
    Student: student
    Effects: 
    Eliminated: false
    Inventory: 
    Knock-out time: -5
    Room: room
    \end{verbatim}
\end{test-case-output}

\subsection{CleanCheese}
\begin{test-case-description}
    A teszt célja tesztelni a Camembert által hordozott effect kitisztítását a szobából egy Takarító belépését követően.
\end{test-case-description}
\begin{test-case-function}
    A Takarító tesztelése, a sajt tesztelése, az effect terjedés tesztelése
\end{test-case-function}
\begin{test-case-input}
    \begin{verbatim}
    Create Room roomA
    Create Room roomB
    Create Door door roomA roomB
    Create Janitor janitor
    Create Student student
    Create Camembert cheese
    Add room cheese
    Pickup student cheese
    Add roomA student
    Add roomB janitor
    Use student cheese
    Move janitor door
    Status roomA
    Status roomB
    \end{verbatim}
\end{test-case-input}
\begin{test-case-output}
    \begin{verbatim}
    Room roomA
    Capacity: 4
    Effects:
    Items:
    People: janitor

    Room roomB
    Capacity: 4
    Effects: 
    Items:
    People: student
    \end{verbatim}
\end{test-case-output}

\subsection{DenyStudentMovementDueToCapacity}
\begin{test-case-description}
    A teszt célja megbizonyosodni arról, hogy hogy az ajtók rossz irányú mozgást nem
engednek át.
\end{test-case-description}
\begin{test-case-function}
    Az ajtók működése
\end{test-case-function}
\begin{test-case-input}
    \begin{verbatim}
    Create Room roomA
    Create Room roomB 0
    Create Student student
    Create door roomA roomB
    Add roomA student
    Move person door
    \end{verbatim}
\end{test-case-input}
\begin{test-case-output}
    \begin{verbatim}
    The other room is full.
    \end{verbatim}
\end{test-case-output}

\subsection{DenyStudentMovementDueToDirection}
\begin{test-case-description}
    A teszt célja megbizonyosodni arról, hogy hogy az ajtók rossz irányú mozgást nem
engednek át.
\end{test-case-description}
\begin{test-case-function}
    Az ajtók működése
\end{test-case-function}
\begin{test-case-input}
    \begin{verbatim}
    Create Room roomA
    Create Room roomB
    Create Student student
    Create door roomA roomB
    Oneway door true
    Add roomB student
    Move student door
    Status roomA
    \end{verbatim}
\end{test-case-input}
\begin{test-case-output}
    \begin{verbatim}
    student couldn't use the door.
    Room roomA
    Capacity: 4
    Effects:
    Items: 
    People:
    \end{verbatim}
\end{test-case-output}

\subsection{DropItem}
\begin{test-case-description}
    A teszt célja az eldobás funkció ellenőrzése
\end{test-case-description}
\begin{test-case-function}
    A tárgyeldobás funkció ellenőrzése.
\end{test-case-function}
\begin{test-case-input}
    \begin{verbatim}
    Create Student student
    Create Room room
    Create Beer beer
    Add room student
    Add room beer
    Pickup student beer
    Drop student beer
    Status student
    Status room
    \end{verbatim}
\end{test-case-input}
\begin{test-case-output}
    \begin{verbatim}
    Student student
    Effects:
    Eliminated: false
    Inventory: 
    Knock-out time: -5
    Room: room

    Room room
    Capacity: 4
    Effects:
    Items: beer
    People: student
    \end{verbatim}
\end{test-case-output}

\subsection{InheritGasEffect}
\begin{test-case-description}
    A teszt célja megbizonyosodni arról, hogy a szoba által számontartott Effectet
átveszik-e a benne lévő szereplők
\end{test-case-description}
\begin{test-case-function}
    Az Effcet terjedése.
\end{test-case-function}
\begin{test-case-input}
    \begin{verbatim}
    Create Room room
    Create Student student
    Create Camembert cheese
    Add room student
    Add student cheese
    Use student cheese
    Update 1
    Status student
    \end{verbatim}
\end{test-case-input}
\begin{test-case-output}
    \begin{verbatim}
    student got knocked out.
    
    Student student
    Effects:
    Eliminated: false
    Inventory: 
    Knock-out time: 5
    Room: room
    \end{verbatim}
\end{test-case-output}


\subsection{KnockOutWithMask}
\begin{test-case-description}
    A teszt célja a Camembert működésének ellenőrzése.
\end{test-case-description}
\begin{test-case-function}
    A Camembert működésének ellenőrzése.
\end{test-case-function}
\begin{test-case-input}
    \begin{verbatim}
    Create Room room
    Create Student student
    Create Mask mask
    Create GasEffect effect
    Add room effect
    Add room mask
    Add room student
    Pickup student mask
    Update 1 room
    Status student
    \end{verbatim}
\end{test-case-input}
\begin{test-case-output}
    \begin{verbatim}
    student got knocked out.
    student was protected by mask.
    
    Student: student
    Effects: 
    Eliminated: false
    Inventory: 
    Knock-out time: -5
    Room: room
    \end{verbatim}
\end{test-case-output}

\subsection{KnockOutWithoutMask}
\begin{test-case-description}
    A teszt célja a Camembert működésének ellenőrzése.
\end{test-case-description}
\begin{test-case-function}
    A Camembret működésének ellenőrzése.
\end{test-case-function}
\begin{test-case-input}
    \begin{verbatim}
    Create Room room
    Create Student student
    Create GasEffect effect
    Add room effect
    Add room student
    Update 1 room
    Status student
    \end{verbatim}
\end{test-case-input}
\begin{test-case-output}
    \begin{verbatim}
    student got knocked out.
    
    Student: student
    Effects: 
    Eliminated: false
    Inventory: 
    Knock-out time: 5
    Room: room
    \end{verbatim}
\end{test-case-output}

\subsection{MergeRooms}
\begin{test-case-description}
    A teszt célja a szobaösszeolvadás ellenőrzése
\end{test-case-description}
\begin{test-case-function}
    A szobaösszeolvadás ellenőrzése
\end{test-case-function}
\begin{test-case-input}
    \begin{verbatim}
    Create Room roomA 2
    Create Room roomB 4
    Create Student student
    Add roomB student
    Merge roomA roomB
    Status roomA
    \end{verbatim}
\end{test-case-input}
\begin{test-case-output}
    \begin{verbatim}
    Room roomA
    Capacity: 4
    Effects:
    Items: 
    People: student
    \end{verbatim}
\end{test-case-output}

\subsection{MoveStudentFromRoomToRoom}
\begin{test-case-description}
    A teszt célja megbizonyosodni arról, hogy a szereplők képesek használni az szobák közti ajtókat.
\end{test-case-description}
\begin{test-case-function}
    Az ajtók működése
\end{test-case-function}
\begin{test-case-input}
    \begin{verbatim}
    Create Room roomA
    Create Room roomB
    Create Student student
    Create door roomA roomB
    Add roomA student
    Move student door
    Status roomB
    Status roomA
    \end{verbatim}
\end{test-case-input}
\begin{test-case-output}
    \begin{verbatim}
    Room roomB
    Capacity: 4
    Effects:
    Items: 
    People: student
    Room roomA
    Capacity: 4
    Effects:
    Items: 
    People:
    \end{verbatim}
\end{test-case-output}

\subsection{PickUpItemWithInventorySpace}
\begin{test-case-description}
    A teszt célja a tárgyfelvétel funkció ellenőrzése.
\end{test-case-description}
\begin{test-case-function}
    A tárgyfelvétel funkció ellenőrzése.
\end{test-case-function}
\begin{test-case-input}
    \begin{verbatim}
    Create Student student
    Create Room room
    Create Beer beer
    Add room beer
    Add room student
    Pickup student beer
    Status student
    \end{verbatim}
\end{test-case-input}
\begin{test-case-output}
    \begin{verbatim}
    Student student
    Effects:
    Eliminated: false
    Inventory: beer
    Knock-out time: -5
    Room: room
    \end{verbatim}
\end{test-case-output}

\subsection{PickUpItemWithoutInventorySpace}
\begin{test-case-description}
    A teszt célja a tárgyfelvétel funkció ellenőrzése.
\end{test-case-description}
\begin{test-case-function}
    A tárgyfelvétel funkció ellenőrzése.
\end{test-case-function}
\begin{test-case-input}
    \begin{verbatim}
    Create Student student
    Create Room room
    Create Beer beer1
    Create Beer beer2
    Create Beer beer3
    Create Beer beer4
    Create Beer beer5
    Create Beer beer6
    Add room student
    Add room beer1
    Add room beer2
    Add room beer3
    Add room beer4
    Add room beer5
    Pickup student beer1
    Pickup student beer2
    Pickup student beer3
    Pickup student beer4
    Pickup student beer5
    Add room beer6
    Pickup student beer6
    \end{verbatim}
\end{test-case-input}
\begin{test-case-output}
    \begin{verbatim}
    Inventory is full.
    \end{verbatim}
\end{test-case-output}

\subsection{PlaceStudent}
\begin{test-case-description}
    A teszt célja megbizonyosodni arról, hogy az elvárt belső állapotválltozások mennek végre, amikor egy Person típusú entitást egy szobába rakunk.
\end{test-case-description}
\begin{test-case-function}
    Az elhelezést vizsgáljuk.
\end{test-case-function}
\begin{test-case-input}
    \begin{verbatim}
    Create Room room
    Create Student student
    Add room student
    Status room
    \end{verbatim}
\end{test-case-input}
\begin{test-case-output}
    \begin{verbatim}
    Room room
    Capacity: 4
    Effects:
    Items:
    People: student
    \end{verbatim}
\end{test-case-output}

\subsection{RoomUpdateEffect}
\begin{test-case-description}
    A teszt célja az update esemény terjedésének ellenörzése Effectekre.
\end{test-case-description}
\begin{test-case-function}
    Az effect frissülésének terjedésének ellenőrzése.
\end{test-case-function}
\begin{test-case-input}
    \begin{verbatim}
    Create Room room
    Create BeerEffect effect
    Add room effect
    Update 1 room
    Status effect
    \end{verbatim}
\end{test-case-input}
\begin{test-case-output}
    \begin{verbatim}
    BeerEffect effect
    Holder: student 
    Time remaining: 4
    \end{verbatim}
\end{test-case-output}

\subsection{SplitRooms}
\begin{test-case-description}
    A teszt célja a szobaszétválás elleőrzése
\end{test-case-description}
\begin{test-case-function}
    A szobaszétválás elleőrzése
\end{test-case-function}
\begin{test-case-input}
    \begin{verbatim}
    Create Room room
    Split room
    Status room_S1
    \end{verbatim}
\end{test-case-input}
\begin{test-case-output}
    \begin{verbatim}
    Room room_s1
    Capacity: 4
    Effects:
    Items: 
    People:
    \end{verbatim}
\end{test-case-output}

\subsection{TeleportUsingTransistors}
\begin{test-case-description}
    A teszt célja a teleportálás funkció ellenőrzése
\end{test-case-description}
\begin{test-case-function}
    A teleportálás funkció ellenőrzése.
\end{test-case-function}
\begin{test-case-input}
    \begin{verbatim}
    Create Room roomA
    Create Room roomB
    Create Door door roomA roomB
    Create Transistor transistorA
    Create Transistor transistorB
    Create Student student
    Add roomB student
    Add roomB transistorA
    Add roomB transistorB
    Pickup student transistorA
    Pickup student transistorB
    Link transistorA transistorB
    Use student transistorA
    Move student door
    Use student transistorB
    Status roomB
    \end{verbatim}
\end{test-case-input}
\begin{test-case-output}
    \begin{verbatim}
    Room roomB
    Capacity: 4
    Effects:
    Items:
    People: student
    \end{verbatim}
\end{test-case-output}

\subsection{UpdateBeerEffect}
\begin{test-case-description}
    A teszt célja az effect frissülésének ellenőrzése.
\end{test-case-description}
\begin{test-case-function}
    Az effect frissülésének ellenőrzése.
\end{test-case-function}
\begin{test-case-input}
    \begin{verbatim}
    Create Student student
    Create BeerEffect effect
    Add student effect
    Update 1 student
    Status effect
    \end{verbatim}
\end{test-case-input}
\begin{test-case-output}
    \begin{verbatim}
    BeerEffect effect
    Holder: student
    Time remaining: 4
    \end{verbatim}
\end{test-case-output}

\subsection{UpdateGasEffect}
\begin{test-case-description}
    A teszt célja az effect frissülésének ellenőrzése.
\end{test-case-description}
\begin{test-case-function}
    Az effect frissülésének ellenőrzése.
\end{test-case-function}
\begin{test-case-input}
    \begin{verbatim}
    Create Room room
    Create GasEffect effect
    Add room effect
    Update 1 room
    Status effect
    \end{verbatim}
\end{test-case-input}
\begin{test-case-output}
    \begin{verbatim}
    GasEffect effect
    Holder: room
    Time remaining: 4
    \end{verbatim}
\end{test-case-output}

\subsection{UpdateMaskEffect}
\begin{test-case-description}
    A teszt célja az effect frissülésének ellenőrzése.
\end{test-case-description}
\begin{test-case-function}
    Az effect frissülésének ellenőrzése.
\end{test-case-function}
\begin{test-case-input}
    \begin{verbatim}
    Create Student student
    Create MaskEffect effect
    Add student effect
    Update 1 student
    Status effect
    \end{verbatim}
\end{test-case-input}
\begin{test-case-output}
    \begin{verbatim}
    MaskEffect effect
    Holder: student
    Time remaining: 14
    \end{verbatim}
\end{test-case-output}

\subsection{UpdateRagEffect}
\begin{test-case-description}
    A teszt célja az effect frissülésének ellenőrzése.
\end{test-case-description}
\begin{test-case-function}
    Az effect frissülésének ellenőrzése.
\end{test-case-function}
\begin{test-case-input}
    \begin{verbatim}
    Create Student student
    Create RagEffect effect
    Add student effect
    Update 1 student
    Status effect
    \end{verbatim}
\end{test-case-input}
\begin{test-case-output}
    \begin{verbatim}
    RagEffect effect
    Holder: student
    Time remaining: 4
    \end{verbatim}
\end{test-case-output}

\clearpage
\section{A tesztelést támogató programok tervei}
% \comment{A tesztadatok előállítására, a tesztek eredményeinek kiértékelésére szolgáló segédprogramok részletes terveit kell elkészíteni.}
A tesztadatok struktúrája:
\begin{verbatim}
    tests/
    ├── test1/
    │   ├─ info.txt (optional)
    │   ├─ in.txt
    │   └─ out.txt
    │
    └── test2/
        ├─ info.txt (optional)
        ├─ in.txt
        └─ out.txt
\end{verbatim}
Az egyes tesztesetek mappákban helyezkednek el. Minden tesztesethez tartozik egy \texttt{in.txt}, ez tartalmazza a teszt megvalósításához szükséges bemenetet, illetve egy \texttt{out.txt}, ami az elvárt kimenetet tartalmazza. Az \texttt{info.txt}-t nem kötelező megadni, ez tartalmazhatja a teszt programban megjelenő nevét (amennyiben ez eltér a mappa nevétől) illetve a teszt rövid leírását.

A teszt futtató program viselkedése parancssori argumentumok segítségével adható meg: 
\begin{itemize}
    \item \texttt{testrunner run <test-case>} - Egy bizonyos tesztesetet futtat és megadja, hogy sikeres volt-e vagy sem. \texttt{test-case} paraméternek az adott teszteset mappáját kell megadni.
    \item \texttt{testrunner runall <tests>} - Az összes tesztet lefuttatja, mindegyik tesztesetről megadja, hogy sikeres volt-e vagy sem, illetve a végén egy statisztikát ad, hogy hány teszt volt sikeres/sikertelen. \texttt{tests} paraméternek a teszteseteket tartalmazó mappát kell megadni.
    \item \texttt{testrunner interactive <tests>} - Interaktív mód, megjeleníti az elérhető teszteseteket, ezután a felhasználó megadhatja, hogy melyik tesztesetet szeretné futtatni. \texttt{tests} paraméternek a teszteseteket tartalmazó mappát kell megadni.
    \item \texttt{testrunner command} - Parancs mód, manuálisan lehet parancsokat begépelni, illetve a program megadja az adott parancshoz tartozó kimenetet.
\end{itemize}