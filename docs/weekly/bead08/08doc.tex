\chapter{Részletes tervek}
\comment{A dokumentum célja, hogy pontosan specifikálja az implementálandó osztályokat, beleértve a privát attribútumokat és metódusokat, ezek definícióját is.A dokumentum második fele részletesen be kell mutassa a korábban definiált be-és kimeneti nyelv szintakszisát felhasználva, hogy mely tesztekkel lesz a prototípus ellenőrizve.}
\section{Osztályok és metódusok tervei}
\subsection{Osztály1}
\begin{class-template-responsibility}
    Felelősség leírása
\end{class-template-responsibility}
\begin{class-template-interface}
    Megvalósított interfészek felsorolása
\end{class-template-interface}
\begin{class-template-baseclass}
    Ős-Ősosztály \baseclass Ősosztály... 
\end{class-template-baseclass}
\begin{class-template-attribute}
    \classitem{+A [0..*]}{adattag A}
\end{class-template-attribute}
\comment{Milyen publikus, protected és privát  metódusokkal rendelkezik. Metódusonként precíz leírás, ha szükséges, activity diagram is a metódusban megvalósítandó algoritmusról. Minden olyan metódusnak szerepelnie kell, amelyiket az osztály megvalósít vagy felüldefiniál.}
\begin{class-template-method}
    \classitem{+B(A a) : void}{metódus B}
\end{class-template-method}

\subsection{Osztály2}
\begin{class-template-responsibility}
    Felelősség leírása
\end{class-template-responsibility}
\begin{class-template-statechart}
    A belső működéshez tartozó állapotgép
    \diagram{img/BMElogo}{Az osztály állapotváltozásai}{3cm}
\end{class-template-statechart}

\section{A tesztek részletes tervei, leírásuk a teszt nyelvén}
\comment{A tesztek részletes tervei alatt meg kell adni azokat a bemeneti adatsorozatokat, amelyekkel a program működése ellenőrizhető. Minden bemenő adatsorozathoz definiálni kell, hogy az adatsorozat végrehajtásától a program mely részeinek, funkcióinak ellenőrzését várjuk és konkrétan milyen eredményekre számítunk, ezek az eredmények hogyan vethetők össze a bemenetekkel.A tesztek leírásakor az előző dokumentumban (proto koncepciója) megadott szintakszist kell használni.}

\subsection{Teszteset1}
\begin{test-case-description}
    Szöveges leírás, kb 1-5 mondat
\end{test-case-description}
\begin{test-case-function}
    ...
\end{test-case-function}
\begin{test-case-input}
    A proto nyelvén megadva, pl.:
    \begin{verbatim}
    init world
    0: S 3
    ...
    print
    \end{verbatim}
\end{test-case-input}
\begin{test-case-output}
    A poro kimeneti nyelvén megadva, pl.:
    \begin{verbatim}
    worlddata
    ...
    creatures 0
    \end{verbatim}
\end{test-case-output}

\section{A tesztelést támogató programok tervei}
\comment{A tesztadatok előállítására, a tesztek eredményeinek kiértékelésére szolgáló segédprogramok részletes terveit kell elkészíteni.}