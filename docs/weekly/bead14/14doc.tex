\chapter{Összefoglalás}
\section{A projektre fordított összes munkaidő}
\begin{ertekelesOra}
    \ertekelestag{Balla Gergely}{[NEPTUN]}{52,25}
    \ertekelestag{Joób Zalán}{[NEPTUN]}{72,25}
    \ertekelestag{Nagy Alexandra}{[NEPTUN]}{70,25}
    \ertekelestag{Tóth Boldizsár}{[NEPTUN]}{123,25}
    \ertekelestag{Zelei Mátyás}{[NEPTUN]}{63,25}
    \ertekelestag{\textbf{Összesen}}{ - }{ 381,25 }
\end{ertekelesOra}

\section{A feltöltött programok forrássorainak megoszlása}
\begin{ertekelesKod}
    \ertekelestagk{Szkeleton}{1444}
    \ertekelestagk{Prototípus}{1793}
    \ertekelestagk{Grafikus változat}{3447}
    \ertekelestagk{\textbf{Összesen}}{4894}
\end{ertekelesKod}

A fázisok kódsorai az adott fázisban leadott kódsorok száma, az összeg pedig a végleges projekt által tartalmazott sorok. Mivel a kód egy része a fázisok között nem változott, így az összeg kevesebb, mint ha az egyes fázisok sorainak összegét számolnánk.

\clearpage

\section{Projekt összegzés}

\begin{itemize}
\item \textit{Mit tanultak a projektből konkrétan és általában?} \newline
    Megtapasztaltuk a csapatmunka nehézségeit. A GitHub használatát gyakoroltuk, ami nekem ugyan nem volt annyira újdonság, de a legtöbb funkcióját eddig nem használtam ilyen mélységben. Az objektum orientáltság elveit, és ugyan ezekkel volt ahol nem értettünk egyet, viszont főleg a vége felé már volt hogy teljesen természetesen jött, hogy itt melyik tervezési mintát kéne alkalmazni és azt hogyan. Ezen kívül a \LaTeX használatát is gyakoroltuk, ami nekem dokumentumok elkészítésére kényelmesebb mint a Word.
\item \textit{Mi volt a legnehezebb és a legkönnyebb?} \newline
    Legnehezebb talán a csapaton belüli kommunikáció volt, nekem mint csoport vezető nem mindig tudtam jól átadni a csapatnak, hogy mik az elvárásaim, illetve visszafelé is néha kicsit több visszajelzést, kérdést vártam volna.
    
    Legkönnyebb a szkeleton és a prototípus elkészítése volt, mivel azoknál a fázisoknál még egy elég egyszerű programot kellett elkészíteni, ami előtte nagyon részletesen meg lett tervezve. 
\item \textit{Összhangban állt-e az idő és a pontszám az elvégzendő feladatokkal?} \newline
    Nem. 
\item \textit{Ha nem, akkor hol okozott ez nehézséget?} \newline
    Leginkább a félév második felében kezdtem ezt érezni, amikor a prototípus elkészítésére volt 2 hetünk, miután 2 hónapig terveztük, így ez már csak egy néhány nap munka volt. Ezután pedig a grafikus változat tervezésére nagyon kevés volt az egy hét.
    
    Illetve úgy általában az idei feladat sokkal nehezebb volt szerintem, mint amiket korábbi évekből láttam, és persze 4 kredit lett a tárgy 3 helyett, de a feladat szerintem nem ezzel arányosan, hanem 2-3-szor lett komplexebb a korábbi évekhez képest.
\item \textit{Milyen változtatási javaslatuk van?} \newline
    Egyrészt valami olyan feladatot adnék, ahol az objektum orientált elvek és az MVC jobban alkalmazható, vagy pedig nem ennyire szigorúan elvárt. Például a typecheck-ek elkerülése volt leginkább olyan, hogy persze objektum orientáltságban nem elegáns a typecheck alkalmazása, de az semmivel sem elegánsabb, hogy objektumok az interfészük metódusainak felét üresen valósítják meg. Illetve az MVC architektúra nem annyira jól használható játékok fejlesztésére, főleg, hogy a modellezési fázisban meg volt tiltva, hogy a játék reprezentáció bármilyen formában bele kerüljön a modellbe, pedig ebben az esetben a dolgok játék világbeli helye az a kezdetektől fogva a modell része kellett volna, hogy legyen mivel részben ez alapján történnek interakciók és ezt csak a legvégén a grafikus verziónál próbáltuk utólag bevezetni.
    
    Ezen kívül jobban figyelembe kéne venni, hogy a Szoftvertechnikák tárgy az előző félévben átalakult, kisebb hangsúlyt fektet az UML diagrammok készítésére, így ebből a tárgyból sem elvárható egy ilyen komplexitású feladat megtervezése úgy, hogy a hallgatók nem szereztek ilyen tapasztalatot korábbi tárgyból. Ehelyett akár pontozási szempont lehetne például a megfelelő verziókezelés használata a fejlesztés során, illetve modernebb technikák alkalmazása, mint például Java projektek fordításához maven vagy gradle használata.
\item \textit{Milyen feladatot ajánlanának a projektre?} \newline
    Amennyiben az MVC-hez ragaszkodunk, úgy olyan játokot vagy akár más jellegű programot, ahol ez az elkülönítés sokkal természetesebben jön, mint egy ilyen jellegű játékban, amit idén készítettünk. Illetve az objektum orientált minták is természetesebben illenek hozzá, és nem csak valahogy rá kell erőltetni, mert éppen az a feladat.
\end{itemize}


