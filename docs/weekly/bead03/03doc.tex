\chapter{Analízis modell kidolgozása 1.}

\section{Objektum katalógus}

\subsection{Hallgató}
A játékos irányítja. Szobáról szobára haladnak ajtókon keresztül. Tudják, melyik szobában vannak, milyen tárgyak vannak náluk, és tudnak velük interaktálni.

\subsection{Oktató}
Szobáról szobára haladnak. Ha egy szobába kerülnek egy hallgatóval, kiejtik a játékból. 

\subsection{Logarléc}
A játék célja ezt megtalálni. Amint egy hallgató felveszi, a játék véget ér.

\subsection{Tranzisztor}
Teleportálásra lehet használni. Össze lehet kapcsonli egy másik tranzisztorral, ami letevés után tudja, melyik szobában van. A kézben lévő tranzisztor a használat után deaktiválódik. Tudja, hogy mely szobákat kapcsolja össze.

\subsection{TVSZ denevérbőrön}
Védelmet nyút az oktatókkal szemben, oktatónként használódik. Tudja, hogy hány használat van hátra.

\subsection{Szent söröspohár}
Aktiválás után 5 másodpercig nyújt védelmet az oktatókkal szemben, aztán  elveszti a képességét. Tudja, hogy még mennyi idő van hátra a hatásából.

\subsection{Nedves táblatörlő rongy}
A vele egy szobában lévő oktatókat megbénítja 15 másodpercre.
Tudja, hogy  mennyi idő van hátra a hatásának időtartamából.

\subsection{FFP2-es maszk}
A mérgesgáz ellen véd. Tudja, hogy hány használata van hátra, és hogy az adott használatból mennyi van hátra.

\subsection{Dobozolt káposztás Camambert}
Felbontáskor mézgező gázt bocsát ki. Használatkor a birtokló játékos elmondja a szobájának, hogy legyen adott ideig mérgező.

\subsection{Szoba}
Egy szoba a játékban. Ő birtokolja a tárgyakat, oktatókat, hallgatókat, amik megtalálhatók benne, illetve az ajtókat, amik belőle nyílnak. Tudja, hogy milyen különleges képessége van. Ismeri az ajtóit, hogy hány ember van benne

\subsection{Ajtó}
Tudja melyik két szoba között megy, azt, hogy egy- vagy kétirányú. Ha egyirányú, az irányt.

\section{Osztályok leírása}
\subsection{Beer}
\begin{class-template-responsibility}
A játékbeli tárgy reprezentálása a modellben.
\end{class-template-responsibility}
\begin{class-template-interface}
    Item
\end{class-template-interface}

\subsection{CodeOfStudies}
\begin{class-template-responsibility}
A játékbeli tárgy reprezentálása a modellben.
\end{class-template-responsibility}
\begin{class-template-interface}
    Item
\end{class-template-interface}
\begin{class-template-attribute}
    \classitem{- uses : int}{Megadja, hogy hányszor használható még a tárgy.}
\end{class-template-attribute}

\subsection{Camambert}
\begin{class-template-responsibility}
A játékbeli tárgy reprezentálása a modellben.
\end{class-template-responsibility}
\begin{class-template-interface}
    Item
\end{class-template-interface}

\subsection{Door}
\begin{class-template-responsibility}
    Tudja melyik két szobát köti össze.
    \end{class-template-responsibility}
\begin{class-template-method}
    \classitem{use(Person person, Room from) : void} { A személy át kíván lépni a másik szobába. Ha abban van kapacitás a szobák értesülnek az átlépésről. Különben nem történik semmi.}
\end{class-template-method}

\subsection{Item}
\begin{class-template-responsibility}
    Interfész biztosítása a különböző tárgykat reprezentáló osztályok számára.
\end{class-template-responsibility}
\begin{class-template-method}
    \classitem{accept(visitor : ItemVisitor) : void}{Visitor befogadása.}
    \classitem{use(target : GameObject) : void}{Felhasználási kérelem befogadása.}
\end{class-template-method}

\subsection{Mask}
\begin{class-template-responsibility}
A játékbeli tárgy reprezentálása a modellben.
\end{class-template-responsibility}
\begin{class-template-interface}
    Item
\end{class-template-interface}
\begin{class-template-attribute}
    \classitem{effect : RagEffect}{A maszkoz tartozó effekt.}
    \classitem{uses : int}{A maszk felhasználásainak száma}
\end{class-template-attribute}

\subsection{Person}
\begin{class-template-responsibility}
    Alaposztály szolgáltatása a mozgó játék beli entitások számára.
\end{class-template-responsibility}
\begin{class-template-attribute}
    \classitem{- name : String}{Az ember neve.}
    \classitem{- numItems : int}{Az embernél lévő tárgyak száma.}
\end{class-template-attribute}
\begin{class-template-method}
    \classitem{+dropItem(Item item) : void}{Az ember eldobja/elejti a tárgyat.}
    \classitem{+knockOut() : void}{Az ember adott időre elájul.}
    \classitem{+enterRoom(room : Room) void}{A szoba értesítése arról, hogy a játékos belépett.}
\end{class-template-method}

\subsection{Room}
\begin{class-template-responsibility}
    Ő birtokolja a benne lévő Door, Person és (nem Personnál lévő) Item objektumokat.
    Ő engedélyezi vagy tagadja meg a Personok belépését.
    Ő értesíti a benne lévő Personokat új Person belépéséről, mérges gázról, táblatörlő rongy használatáról. 
    \end{class-template-responsibility}
\begin{class-template-attribute}
    \classitem{- capacity : int}{Megadja maximum hány ember fér el a szobában.}
    \classitem{- numOfPeople : int}{A szobába jelenleg tartózkodó emberek száma.}
\end{class-template-attribute}
\begin{class-template-method}
    \classitem{visitEffects(visitor: EffectVisitor) : void}{Meghívja az összes szobára hatást gyakorló visitort.}
    \classitem{visitPeople(visitor: PeopleVisitor) : void}{Meghívja az összes szobában lévő emberre a visitort.}
    \classitem{enter(Person person) : void} { A személy belép a szobába és a többi személy értesül róla, és reagál, ha kell }
    \classitem{leave(Person peron) : void} { A szobát elhagyna egy személy, a szobában lévő emberek száma csökken. }
    \classitem{merge(Room room) : void} { Két szoba összeolvad. }
    \classitem{split() : void} {A szoba a szabályoknak megfelelően osztódik.}
\end{class-template-method}

\subsection{Student}
\begin{class-template-responsibility}
    A játékos reprezentálása a modellben.
\end{class-template-responsibility}
\begin{class-template-baseclass}
    Person
\end{class-template-baseclass}
\begin{class-template-method}
    \classitem{+beingAttacked() : void}{ Megtámadja egy oktató. Ha van védőeszköze, azt használja. Ha nem, eltávolítja magát a szobából.}
    \classitem{+ignoreTeacher(Teacher teacher) : void}{Atott tanárral szemben védi a hallgatót, amíg ugyanabban a szobában vannak.}
    \classitem{+eliminate() : void}{A hallgatót értesíti arról, hogy meghalt.}
\end{class-template-method}

\subsection{Teacher}
\begin{class-template-responsibility}
    Az ellenségek reprezentálása a modellben.
\end{class-template-responsibility}
\begin{class-template-baseclass}
    Person
\end{class-template-baseclass}

\subsection{Transistor}
\begin{class-template-responsibility}
A játékbeli tárgy reprezentálása a modellben.
\end{class-template-responsibility}
\begin{class-template-interface}
    Item
\end{class-template-interface}
\begin{class-template-method}
    \classitem{use(person : Person) : void}{}
    \classitem{use(item : Item) : void}{}
    \classitem{use(transistor : Transistor) : void}{}
    A három use függvény a Use Transistors szekvencia-diagramban van magyarázva.
    \classitem{link(other: Transistor)}{A tranzisztor összekapcsolása a párjával.}
    \classitem{setTarget(target : Room) : void}{A tranzisztor-teleportáció célszobájának beállítása}
\end{class-template-method}

\subsection{WetRag}
\begin{class-template-responsibility}
A játékbeli tárgy reprezentálása a modellben.
\end{class-template-responsibility}
\begin{class-template-interface}
    Item
\end{class-template-interface}
\begin{class-template-attribute}
    \classitem{- effect : RagEffect}{A tárgyhoz hozzárendelt effekt.}
\end{class-template-attribute}

\section{Statikus struktúra diagramok}

Az osztálydiagram létrehozásakor egy szoba alapú architektúrát modelleztünk. Architektúránkban a szobák kezelik a területükön végbemenő folyamatokat. 

A szobába való belépés egy ajtón keresztül történik. Egy ajtót a Door osztály reprezentál. Egy játékos(Student) az ajtó(Door) objektumhoz fordulva léphet be egy szobába. A belépési kérelmet az ajtó összeveti a saját irányával, majd ha a kérés kompatibilis az átengedési követelménnyel, akkor a kérést továbbítja a hozzá rendelt Room objektumnak, ami a kérésről a végső döntést meghozza.

A játékban két mozgásra képes entitás létezik: a tanárok(Teacher) és a játékosok(Student). Ezeket egy ősosztályból származtatjuk(Person).

A tárgyak(Item) kezelését Visitor tervezési minta használatával oldottuk meg, mivel így a kezelési algoritmusok teljesen elválaszthatóak az objektumoktól. Így későbbiekben a kezelés viselkedése megváltozatatható az adott osztály belső működésének megváltozattása nélkül. A különböző tárgyak egy Item interfész megvalósításai, így a tárgyak körében biztosított az encapsulation elve. Ezeket az Item-eket Külöböző visitor-ok kezelik, melyek az ItemVIsitor, EffectVisitor és PersonVisitor interfészeket valósítják meg.

\diagram{img/diagrams/class/effects}{Hatások}{14cm}
\diagram{img/diagrams/class/game-objects}{Játék objektumok}{14cm}
\diagram{img/diagrams/class/items}{Tárgyak}{14cm}       
\diagram{img/diagrams/class/Visitors}{Látogató minta megvalósítása}{14cm} 
\clearpage

\section{Szekvencia diagramok}
\diagram{img/diagrams/sequence/add-item}{Tárgyfelvétel diagram}{14cm}
\diagram{img/diagrams/sequence/beer}{Söröspohár használata diagram}{14cm}
\diagram{img/diagrams/sequence/camembert}{Camembert diagram}{14cm}
\diagram{img/diagrams/sequence/door}{Ajtó használat diagram}{14cm}
\diagram{img/diagrams/sequence/drop-item}{Tárgyelejtés diagram}{14cm}
\diagram{img/diagrams/sequence/drop-wet-rag}{Nedves táblatörlő rongy elejtése diagram}{14cm}
\diagram{img/diagrams/sequence/link-transistor}{Tranzisztor összekapcsolása diagram}{14cm}
\diagram{img/diagrams/sequence/merge-room}{Szobák egyesülése diagram}{14cm}
\diagram{img/diagrams/sequence/perform-attack}{Támadás diagram}{14cm}
\diagram{img/diagrams/sequence/person-enter-room}{Személy belépése egy szobába diagram}{14cm}
\diagram{img/diagrams/sequence/place-transistor}{Tranzisztor lehelyezése diagram}{14cm}
\diagram{img/diagrams/sequence/remove-item}{Tárgy eltávolítása diagram}{14cm}
\diagram{img/diagrams/sequence/soul-theft}{Lélek elvétel diagram}{14cm}
\diagram{img/diagrams/sequence/split-room}{Szoba osztódás diagram}{14cm}
\diagram{img/diagrams/sequence/toxic-room}{Mérgező szoba diagram}{14cm}
\diagram{img/diagrams/sequence/room-update}{Szoba frissítése diagram}{14cm}
\diagram{img/diagrams/sequence/teacher-update}{Oktató frissítése diagram}{14cm}
\diagram{img/diagrams/sequence/use-transistor}{Tranzisztor használat diagram}{14cm}
\diagram{img/diagrams/sequence/pick-up-wet-rag}{Nedves táblatörlő rongy felvétele diagram}{14cm}

