\chapter{Követelmény, projekt, funkcionalitás}

\section{Bevezetés}

\subsection{Cél}
A dokumentum célja a megrendelő tájékoztatása a szoftver funkcióiról.
A fejlesztők számára a szoftverrel szemben támasztott követelmények, a projekttel kapcsolatos feladatok, határidők, a szoftver funkcionalitásának részletes leírása.

\subsection{Szakterület}
A játék célja a szórakoztatás.

\subsection{Definíciók, rövidítések}

\begin{rov}
FFP2:  Filtering Facepiece 2
\end{rov}
\begin{rov}
JRE: Java Runtime Environment
\end{rov}
\begin{rov}
MVC: Model–view–controller
\end{rov}
\begin{rov}
TVSZ: Tanulmányi és Vizsgaszabályzat
\end{rov}

\subsection{Hivatkozások}
\urlref{https://www.iit.bme.hu/targyak/BMEVIIIAB02}{Tárgyhonlap}
\urlref{https://niif.cloud.bme.hu/ }{Kari felhő}

\subsection{Összefoglalás}
A szoftverünk egy MVC modellre épülő számítógépes játék, melyet Java programnyeven fogunk elkészíteni.

A játékot egyszerre többen is játszhatják. Minden játékos egy diák karaktert irányít. A játék célja, hogy megszerezzük a pályán elhelyezett Logarlécet a rendelkezésre álló idő alatt, mely 11 félévnyi játékidő. Ez az időtartam a valóságban 5 és fél percnek fog megfelelni. A diák karakterek ellenségei a tanárok. Egy játékos meghal, ha a karaktere egy tanárral egy szobába kerül. Ha az idő lejár vagy az összes játékos meghal, a hallgatók vesztettek és a játék véget ér.

A játékosok többféle eszközt használhatnak a játék közben, ezek mindegyike segíti őket a túlélésben, ilyen tárgyak a Tranzisztor, TVSZ denevérbőrön, Szent söröspohár, a Nedves táblatörlő rongy és az FFP2-es maszk.

A játékot minimum 2 ember játsza, akik egy meghatározott játéktérben kezdenek el mozogni.
\section{Áttekintés}
\subsection{Általános áttekintés}

A szoftver a Model-view-controller (Modell-nézet-vezérlő) programtervezési minta használatával készül. Ennek a legfontosabb része a modell, ami a játék belső logikájával és az adatok tárolásával foglalkozik. Ide tartozik a játékosok tárolása, a tárgyakkal való interakciók megvalósítása, az oktatók vezérlése. Ez a komponens kezeli a térképet, szobák egyesülését és osztódását, valamint a játék állását is. A vezérlőtől egy interfészen keresztül megkapja a felhasználói interakciókat, majd ezek alapján módosítja a tárolt adatokat (például egy játékos pozícióját frissíti). A nézet feladata a felhasználói felület megjelenítése a modellben tárolt adatok alapján, a pálya, a játékosok, vagy éppen a főmenü megrajzolása.

\subsection{Funkciók}

Feladatunk egy a Műegyetem Központi épületének alagsorában játszódó játék megvalósítása. A játék célja egy ereklye, a Logarléc megtalálása adott időintevallumon bellül. Ezt nehezíti az, hogy a játék egy mágikus labirintusban játszódik.

A játék szereplői a hallgatók és oktatók. A hallgatókat játékosok irányítják. A oktatókat egy szofisztikált algoritmus vezérli.

A játék előre meghatározott játéktérrel indul, a szobák kiosztása fix.
Az egész játéktér szobákra van felosztva amik között ajtókon keresztül lehet közlekedni. Vannak ajtók, amelyek csak egy irányban használhatók. A szobákban hallgatók és oktatók vannak, de maximum annyian amennyi a szoba kapacitása, ez után új játékos nem léphet a szobába. 

Egyes szobák mérgező levegővel rendelkeznek. Az ide belépő emberek 5 másodpercre elájulnak és elejtik a  náluk lévő tárgyakat. Az idő leteltével felébrednek és 5 másodpercük lesz kijutni a szobából, mielőtt a folyamat megismétlődik.

Vannak elátkozott szobák amiknek az összes ajtaja véletlenszerű időközönként eltűnnek majd újra megjelennek.

Szomszédos szobák összeolvadhatnak: Ekkor a kapott szoba örökli a két eredeti szoba tulajdonságait, a kapacitása pedig megegyezik a nagyobb szoba kapacitásával. Ha az így előállt szobában többen tartózkodnak, mint a szoba befogadóképessége, akkor a szobában tartózkodók közül néhányan (véletlenszerűen, akár hallgató akár oktató) a "Backrooms"-ba kerülnek, azaz eltűnik a pályáról. Amint valaki elhagyja a szobát és lesz rá kapacitás, a Backrooms-ba került személyek visszakerülnek a játékba.

Osztódó szoba: "két olyan szobára válik szét, amelyek egymás szomszédai lesznek, és megosztoznak a korábbi szoba képességein és szomszédain (a korábbi szomszédok vagy csak az egyik, vagy csak a másik “új” szobának lesznek szomszédai)". 

Egyes szobákban különleges képességekkel bíró tárgyak vannak. Ezeket a játékosok (és oktatók) magukhoz vehetik, eldobhatják és el is ejthetik. Az alábbi tárgyak vannak: 
\begin{itemize}
    \item Logarléc: a játék célja ezt megtalálni. Amikor egy játékos ezt felveszi, a játék véget ér és a játékosok nyernek.
    \item Tranzisztor: kettőt össze kell kapcsolni, majd az egyiket le kell helyezni egy szobában. Ezután a másik tranzisztor tulajdonosa a nála maradt tárgyat bekapcsolhatja és lehelyezheti, ami után egyből átkerül a másik szobába. A kiindulási helyen lévő transzisztor kikapcsol.
    \item TVSZ denevérbőrön: védelmet nyújthat oktatók ellen. Amikor a TVSZ-t birtokló játékos egy szobába kerül oktatókkal, a tárgy minden jelenlévő oktatóval használódik egyet, ami az adott oktatót megöli. Összesen háromszor használható.
    \item Szent söröspohár: 5 másodpercig nyújt védelmet az oktatókkal szemben.
    \item Nedves táblatörlő rongy: a vele egy szobában lévő oktatókat megbénítja arra az időtartamra amig egy szobában tartózkodnak vele. A hatása addig tart míg ki nem szárad (15 mp). Nem újrafelhasználható.
    \item FFP2-es maszk: védelmet nyújt a mérgesgázzal szemben. Hatása először 5 másodpercig tart, majd  ez az időtartam használatonként egy másodperccel csökken.
    \item Dobozolt káposztás Camambert: Felbontáskor mérgesgázt bocsát ki, mely 5 másodpercig kitölti a szobát.
\end{itemize}

Minden tárgy az elhasználódása után eltűnik a játékos kezéből, de bizonyos időközönként véletlenszerű helyeken újra megjelenhetnek.

\subsection{Felhasználók}

A felhasználókkal szemben nincs megkötés, amennyiben rendelkeznek a futtatáshoz szükséges környezettel.

\subsection{Korlátozások}

A megvalósítás Java nyelven történik külső könyvtárak felhasználása nélkül, parancssorból lehessen fordítani, fusson a kari felhőben megadott virtuális gépen.

\subsection{Feltételezések, kapcsolatok}
Az információszerzés fő forrása a tárgyhonlap.
A megrendelő követelményeinek a feladatleírásban feltüntetteket tekintjük.


\section{Követelmények}
\subsection{Funkcionális követelmények}
\begin{funkovetelmeny}
	%Azonosító
	{FK001}
	%Prioritás
	{Alapvető}
	%Forrás
	{Megrendelő}
	%Use case
	{A játék elindítása}
	%Ellenőrzés
	{Tesztelés}
	%Leírás
	{Van legalább két szoba.}
\end{funkovetelmeny}
\begin{funkovetelmeny}
	%Azonosító
	{FK002}
	%Prioritás
	{Alapvető}
	%Forrás
	{Megrendelő}
	%Use case
	{A játék elindítása}
	%Ellenőrzés
	{Tesztelés}
	%Leírás
	{Egy szobából legalább egy másik szobába nyílik ajtó.}
\end{funkovetelmeny}
\begin{funkovetelmeny}
	%Azonosító
	{FK003}
	%Prioritás
	{Alapvető}
	%Forrás
	{Megrendelő}
	%Use case
	{Ajtó használat}
	%Ellenőrzés
	{Tesztelés}
	%Leírás
	{Játékos csak ajtón haladhat át.}
\end{funkovetelmeny}
\begin{funkovetelmeny}
	%Azonosító
	{FK004}
	%Prioritás
	{Alapvető}
	%Forrás
	{Megrendelő}
	%Use case
	{Tágyfelvétel}
	%Ellenőrzés
	{Tesztelés.}
	%Leírás
	{Tárgy felvehető.}
\end{funkovetelmeny}
\begin{funkovetelmeny}
	%Azonosító
	{FK005}
	%Prioritás
	{Alapvető}
	%Forrás
	{Megrendelő}
	%Use case
	{Tágyeldobás}
	%Ellenőrzés
	{Tesztelés.}
	%Leírás
	{Tárgy eldobható.}
\end{funkovetelmeny}
\begin{funkovetelmeny}
	%Azonosító
	{FK006}
	%Prioritás
	{Alapvető}
	%Forrás
	{Megrendelő}
	%Use case
	{Logarléc felvétele}
	%Ellenőrzés
	{Tesztelés.}
	%Leírás
	{A Logarléc felvételével véget ér a játék.}
\end{funkovetelmeny}
\begin{funkovetelmeny}
	%Azonosító
	{FK007}
	%Prioritás
	{Alapvető}
	%Forrás
	{Megrendelő}
	%Use case
	{Kilépés a játékból}
	%Ellenőrzés
	{Tesztelés.}
	%Leírás
	{Az oktatók hallgatókat ejtenek ki a játékból.}
\end{funkovetelmeny}
\begin{funkovetelmeny}
	%Azonosító
	{FK008}
	%Prioritás
	{Alapvető}
	%Forrás
	{Megrendelő}
	%Use case
	{Ajtó használat}
	%Ellenőrzés
	{Tesztelés.}
	%Leírás
	{Vannak egyirányú ajtók.}
\end{funkovetelmeny}
\begin{funkovetelmeny}
	%Azonosító
	{FK009}
	%Prioritás
	{Alapvető}
	%Forrás
	{Megrendelő}
	%Use case
	{-}
	%Ellenőrzés
	{Tesztelés.}
	%Leírás
	{A szent söröspohár 5 mp-ig nyújt védelmet.}
\end{funkovetelmeny}
\begin{funkovetelmeny}
	%Azonosító
	{FK010}
	%Prioritás
	{Alapvető}
	%Forrás
	{Megrendelő}
	%Use case
	{-}
	%Ellenőrzés
	{Tesztelés.}
	%Leírás
	{A nedves táblatörlő rongy megbénítja a vele egy szobában lévő tanárokat 5 másodpercre.}
\end{funkovetelmeny}
\begin{funkovetelmeny}
	%Azonosító
	{FK011}
	%Prioritás
	{Alapvető}
	%Forrás
	{Megrendelő}
	%Use case
	{-}
	%Ellenőrzés
	{Tesztelés.}
	%Leírás
	{FFP2-es maszk egyre rövidebb ideig nyújt védelmet a mérgező gázos szobákban.}
\end{funkovetelmeny}
\begin{funkovetelmeny}
	%Azonosító
	{FK012}
	%Prioritás
	{Alapvető}
	%Forrás
	{Megrendelő}
	%Use case
	{-}
	%Ellenőrzés
	{Tesztelés.}
	%Leírás
	{Dobozolt káposztás camamberet felbontáskor mérgező gázt bocsájt ki, mely 5 másodpercig hat.}
\end{funkovetelmeny}
\begin{funkovetelmeny}
	%Azonosító
	{FK013}
	%Prioritás
	{Alapvető}
	%Forrás
	{Megrendelő}
	%Use case
	{A játék elindítása.}
	%Ellenőrzés
	{Tesztelés.}
	%Leírás
	{Minden tanár ugyanabban a szobában kezd.}
\end{funkovetelmeny}
\begin{funkovetelmeny}
    %Azonosító
	{FK014}
	%Prioritás
	{Alapvető}
	%Forrás
	{Megrendelő}
	%Use case
	{Szobák egyesülése.}
	%Ellenőrzés
	{Tesztelés.}
	%Leírás
	{Két szoba random időpontokban egyesülhet. Ilyenkor a szobák közti ajtók eltűnnek és az újonnani szoba területe megegyezik a két régi szoba területeinek összegével, a kapacitását pedig a nagyobb kapacitású szobától örökli.}
\end{funkovetelmeny}
\begin{funkovetelmeny}
    %Azonosító
	{FK015}
	%Prioritás
	{Alapvető}
	%Forrás
	{Megrendelő}
	%Use case
	{Szoba osztódása}
	%Ellenőrzés
	{Tesztelés.}
	%Leírás
	{Két szoba random időpontokban osztódhat. Ilyenkor az új két szoba területeinek összege megegyezik az eredeti szoba területével. A  két szoba kapacitása egyenlő lesz a nagyobb szoba kapcitásával.}
\end{funkovetelmeny}
\begin{funkovetelmeny}
    %Azonosító
	{FK016}
	%Prioritás
	{Alapvető}
	%Forrás
	{Megrendelő}
	%Use case
	{-}
	%Ellenőrzés
	{Tesztelés.}
	%Leírás
	{A denevérbőrre nyomtatott TVSZ 3 alkalommal képes megmenteni a játékos életét. 3 alkalom után erejét veszti.}
\end{funkovetelmeny}
\begin{funkovetelmeny}
    %Azonosító
	{FK017}
	%Prioritás
	{Alapvető}
	%Forrás
	{Megrendelő}
	%Use case
	{-}
	%Ellenőrzés
	{Tesztelés.}
	%Leírás
	{Két tranzisztor összekapcsolható egymással. Az összekapcsolt tranzisztorok teleportálásra használhatók.}
\end{funkovetelmeny}
\begin{funkovetelmeny}
    %Azonosító
	{FK018}
	%Prioritás
	{Alapvető}
	%Forrás
	{Megrendelő}
	%Use case
	{-}
	%Ellenőrzés
	{Tesztelés.}
	%Leírás
	{Az elhasznált tárgyak elvesznek, de két félév elteltével újra megjelennek valahol a pályán.}
\end{funkovetelmeny}
\clearpage

\subsection{Erőforrásokkal kapcsolatos követelmények}
\begin{kovetelmeny}
	{EK001} %Azonosító
	{Alapvető} %Prioritás
	{Megrendelő} %Forrás
	{Bemutatás} %Ellenőrzés
	{Java nyelven van írva.} %Leírás
\end{kovetelmeny}
\begin{kovetelmeny}
	{EK002} %Azonosító
	{Alapvető} %Prioritás
	{Megrendelő} %Forrás
	{Bemutatás} %Ellenőrzés
	{Fut a kari felhőben megadott virtuális gépen.} %Leírás
\end{kovetelmeny}
\begin{kovetelmeny}
	{EK003} %Azonosító
	{Alapvető} %Prioritás
	{Megrendelő} %Forrás
	{Bemutatás} %Ellenőrzés
	{Nem használ külső könyvtárt.} %Leírás
\end{kovetelmeny}
\begin{kovetelmeny}
	{EK004} %Azonosító
	{Alapvető} %Prioritás
	{Fejlesztők} %Forrás
	{Tesztelés} %Ellenőrzés
	{A program futtatásához a számítógépnek legalább 4GB RAM-al kell rendelkeznie.} %Leírás
\end{kovetelmeny}
\begin{kovetelmeny}
	{EK005} %Azonosító
	{Opcionális} %Prioritás
	{Fejlesztők} %Forrás
	{Bemutatás} %Ellenőrzés
	{A program futtatásához ajánlott RAM mennyiség 8GB.} %Leírás
\end{kovetelmeny}
\begin{kovetelmeny}
	{EK006} %Azonosító
	{Alapvető} %Prioritás
	{Fejlesztők} %Forrás
	{Tesztelés} %Ellenőrzés
	{A program futtatásához legalább 2 maggal rendelkező processzor szükséges.} %Leírás
\end{kovetelmeny}
\subsection{Átadással kapcsolatos követelmények}
\begin{kovetelmeny}
    {AK001} %Azonosító
    {Alapvető} %Prioritás
    {Megrendelő} %Forrás
    {Futtatás} %Ellenőrzés
    {Parancssorból fut.} %Leírás
\end{kovetelmeny}

\subsection{Egyéb nem funkcionális követelmények}
\begin{kovetelmeny}
    {XK001} %Azonosító
    {Alapvető} %Prioritás
    {Fejlesztők} %Forrás
    {Tesztek} %Ellenőrzés
    {Átmegy a teszteken.} %Leírás
\end{kovetelmeny}
\begin{kovetelmeny}
    {XK002} %Azonosító
    {Fontos} %Prioritás
    {Fejlesztők} %Forrás
    {Próbafuttatások.} %Ellenőrzés
    {Hordozható.} %Leírás
\end{kovetelmeny}
\begin{kovetelmeny}
    {XK003} %Azonosító
    {Fontos} %Prioritás
    {Fejlesztők} %Forrás
    {Próbafuttatások.} %Ellenőrzés
    {Megbízható.} %Leírás
\end{kovetelmeny}

\section{Lényeges use-case-ek}
\subsection{Use-case leírások}
\begin{use-case}
	%név
	{Ajtó használat}
	%rövid leírás
	{A játékos/oktató egyik szobából a másikba jut.}
	%aktorok
	{Egy játékos / A kontroller}
	%forgatókönyv 
        Egy ajtóhoz közel érünk.
        
        \textbf{A.1} A játékos/oktató átlképhet az ajtón: az ajtó jó irányba volt egyirányú vagy az ajtó kétirányú volt. 
        
        \textbf{A.2} A játékos/oktató nem tud átmenni az ajtón, mert a szoba tele van vagy az ellentétes irányban egyirányú.
\end{use-case}
\begin{use-case}
	%név
	{Tárgyfelvétel}
	%rövid leírás
	{A játékos/oktató fel akar venni egy tárgyat.}
	%aktorok
	{Egy játékos. / A kontroller}
	%forgatókönyv         
        \textbf{A.1} A játékos/oktató nem veheti fel a tárgyat mert már már van 5 db tárgy a birtokában.
        
        \textbf{A.2} A játékos/oktató felveszi a tárgyat.
\end{use-case}
\begin{use-case}
	%név
	{Tárgyhasználat}
	%rövid leírás
	{A játékos használni akar egy tárgyat.}
	%aktorok
	{Egy játékos.}
	%forgatókönyv         
    {A játékos használja a tágyat.}
\end{use-case}
\begin{use-case}
	%név
	{Tárgyeldobás}
	%rövid leírás
	{A játékos önszántából eldob egy a tulajdonában lévő tárgyat}
	%aktorok
	{Egy játékos.}
	%forgatókönyv         
    {Az eldobott tárgy a játékos közelében a földre kerül.}
\end{use-case}
\begin{use-case}
	%név
	{Logarléc felvétele}
	%rövid leírás
	{Az egyik játékos felveszi a Logarlécet}
	%aktorok
	{Egy játékos.}
	%forgatókönyv         
    {A játék véget ér, a játékosok nyernek.}
\end{use-case}
\begin{use-case}
	%név
	{Szoba osztódása}
	%rövid leírás
	{A program automatikusan kettéoszt egy szobát}
	%aktorok
	{A kontroller.}
	%forgatókönyv         
    \textbf{A1} Az eredeti szoba megszűnik.
    
    \textbf{A2} Helyette kettő új szoba jelenik meg.
    
    \textbf{A3} A kapacitásuk összege az eredeti szoba kapacitásásval egyezik meg.
    
    \textbf{A4} Az eredeti szobában levő tárgyak, tanárok és hallgatók véletlenszerűen elosztódnak az új szobák között.
    
    \textbf{A5} Ha az eredeti szoba mérges gázos volt, akkor az újak is azok lesznek. 
\end{use-case}
\begin{use-case}
	%név
	{Szobák egyesülése}
	%rövid leírás
	{A program automatikusan egyesít két szomszédos szobát}
	%aktorok
	{A kontroller.}
	%forgatókönyv         
    \textbf{A1} Az eredeti szobák megszűnnek.
    
    \textbf{A2} Helyettük egy új szoba jelenik meg.
    
    \textbf{A3} A kapacitása az eredeti szobák kapacitása közül a nagyobbik lesz

     
    \textbf{A4} Az eredeti szobákban levő tárgyak, tanárok és hallgatók véletlenszerűen az újba kerülnek.

    
    \textbf{A5} Ha az eredeti szobák közül legalább egyik mérges gázos volt, akkor az új is az lesz. 

    \textbf{A6} Ha így több személy lenne a szobában, mint a kapacitás az egyesülés nem történik meg.
\end{use-case}
\begin{use-case}
	%név
	{A játék elindítása}
	%rövid leírás
	{A felhasználó a parancssorból elindítja a programot.}
	%aktorok
	{Egy játékos.}
	%forgatókönyv         
    {A játékablak megjelenik a felhasználó előtt.}
\end{use-case}
\begin{use-case}
	%név
	{Kilépés a játékból}
	%rövid leírás
	{A játékos nem kívánja tovább használni a programot, vagy a játék szabályai nem engedik.}
	%aktorok
	{Egy játékos / A kontroller}
	%forgatókönyv         
        \textbf{A.1} A játékos kilép a játékból.
        
        \textbf{A.2} A játékost kiejti egy oktató.

        A játékos nem lesz jelen többé a labirintusban. Ha nincs több játékos a labirintusban, a játék véget ér.
\end{use-case}




\subsection{Use-case diagram}

\diagram{img/diagrams/use-case}{Use-case diagram}{14cm}

\section{Szótár}

%szotar.sh script, lsd. a dokumentációt!!!
\begin{szotar}
\szotaritem {Ajtó}{két szoba közötti átjáró, lehet egyirányú}
\szotaritem {Átjutás egyik szobából a másikba} {átlépés egy ajtón, tehát ahol nincs fal}
\szotaritem {Backrooms} {Egy fiktív alternatív dimenzió, amelyet eredeti változatában üres irodahelyiségek labirintusaként írtak le, ahová programhiba miatt kerülhetünk át a valóságból. A játékban a befogadóképességüket túllépő szobákban tartózkodó személyek kerülnek ide.}
\szotaritem {Befogadóképesség} {az adott szobára jellemző tulajdonság, azt adja meg, hogy hányan tartózkodhatnak egyidejűleg a szobában}
\szotaritem {Dobozolt káposztás camambert}{egy olyan tárgy, mely mérges gázt bocsát ki}
\szotaritem {Elátkozott szoba}{olyan szoba, melynek ajtajai néha eltűnnek}
\szotaritem {Fal} {a labirintust szobákra osztja fel}
\szotaritem {Hallgató} {Egy egyetemi diák. Itt egy játékos által irányított karakter.}
\szotaritem{Játékos} {A programot szórakoztatás céljából vezérlő ember. Jelen dokumentáció sokszor a játékos által irányított hallgatót röviden játékosnak nevezi.}
\szotaritem{Kapacitás} {befogadóképesség}
\szotaritem {Labirintus} {a játéktér, itt mozoghatnak a játékosok}
\szotaritem {Logarléc} {mágikus tárgy, a játék célja ezt megtalálni}
\szotaritem {Maszk} {egy olyan tárgy, amely rövid ideig védettséget biztosít a mérgező gázzal szemben}
\szotaritem {Mérgező gáz} {oktatók és hallgatók eszméletvesztését és tárgyaik elejtését okozza.}
\szotaritem {Nedves táblatörlő rongy} {olyan tárgy, mely a vele egy szobában lévő oktatókat megbénítja, egy adott ideig hat}
\szotaritem {Oktató} {a játék egy szereplője, aki a hallgatót akadályozza/elveszi a lelkét}
\szotaritem {Szent söröspohár} {olyan tárgy, mely csak egy adott ideig nyújt védelmet az oktatóktól}
\szotaritem {Szoba} {a labirintus egy része, a határait falak jelölik ki}
\szotaritem {Szoba egyesülése} {ekkor a „létrejövő szoba a korábbi két szoba tulajdonságaival és szomszédaival rendelkezik, de a befogadóképessége a nagyobb szoba befogadóképességével lesz azonos”}
\szotaritem {Szoba osztódása}{„két olyan szobára válik szét, amelyek egymás szomszédai lesznek, és megosztoznak a korábbi szoba képességein és szomszédain”}
\szotaritem {Tárgy}{olyan elemei a játéknak, amiket a játékosok és oktatók magukhoz vehetnek (egy időben max. 5 db lehet egy játékosnál), elejthetnek, lerakhatnak. Lehetnek különleges képességeik}
\szotaritem {Tranzisztor}{a szobákban találhatók, párok alkothatók belőlük. Az egyik tranzisztort egy szobában hagyva a hallgató egy másik szobából visszakerülhet ebbe a szobába. Korlátlan ideig használható}
\szotaritem {TVSZ Denevérbőrön}{olyan tárgy, mely három alkalommal nyújt védelmet az oktatókkal szemben, aztán elveszíti az erejét}
\szotaritem {Védettség} {az a státusz amit egy védettséget okozó tárgy birtoklása kölcsönöz az adott hallgatónak. Ekkor a hallgatónak nem árthatnak az oktatók. A védettség tárgytól függően mérhető időben/alkalomban.}
\end{szotar}

\section{Projekt terv}
\subsection{Projektütemterv}
\begin{terv}
		\tervitem{ febr. 26.}  {Követelmény, projekt, funkcionalitás}{ Nagy Alexandra }
        \tervitem{ márc. 4. } { Analízis modell kidolgozása 1. - beadás }{ Joób Zalán } 
        \tervitem{ márc. 11. } { Analízis modell kidolgozása 2. - beadás }{ Joób Zalán }
        \tervitem{ márc. 18. }{ Skeleton tervezése - beadás }{ Zelei Mátyás }
        \tervitem{ márc. 25. }{ Skeleton - beadás és a forráskód herculesre való feltöltése }{ Nagy Alexandra }
        \tervitem{ ápr. 8. }{ Prototípus koncepciója - beadás }{ Balla Gergely }
        \tervitem{ ápr. 15. }{ Részletes tervek - beadás }{ Tóth Boldizsár }
        \tervitem{ ápr. 29. }{ Prototípus - beadás és a forráskód, a tesztbemenetek és az elvárt kimenetek herculesre való feltöltése }{ Zelei Mátyás }
        \tervitem{ máj. 6. }{ Grafikus felület specifikációja - beadás }{ Balla Gergely }
        \tervitem{ máj. 22. }{ Grafikus változat és Összefoglalás - beadás és a forráskód herculesre való feltöltése }{ Tóth Boldizsár } 
\end{terv}
\comment{A tárgyhonlapon közzétett ütemezés szerint a Részletes tervek leadásának határideje május 15., viszont a sorrend és a korábbi évek alapján valószínűbb, hogy  április 15. }
\subsection{Erőforrások, eszközök}
A fejlesztés során felhasznált segédeszközök:
\begin{itemize}
	\item Dokumentáció: A heti beadandó feladatokhoz a \LaTeX\ sablont használjuk. Közös munkához az Overleaf online \LaTeX szerkesztő programot használjuk.
	\item Kommunikáció: Első sorban Discord-on keresztül, ezen kívül GitHub-on az Issue-k alatti kommentekben.
	\item Modellező eszköz: Modellek, diagrammok készítésére PlantUML-t használunk.
	\item Fejlesztő környezetek: IntelliJ IDEA, Visual Studio Code
	\item Forráskód megosztás, verziókezelés: Verziókezeléshez git-et használunk a repository megosztása pedig GitHub-on keresztül történik.
	\item Projekt menedzsment: Feladatok kiosztásához, kezeléséhez, haladásuk nyomonkövetéséhez a GitHub Projects-et illetve GitHub-on felvett Issue-kat használunk.
\end{itemize}
