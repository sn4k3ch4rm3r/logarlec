\chapter{Grafikus felület specifikációja}

\section{A grafikus interfész}
% \comment{A menürendszer, a kezelői felület grafikus képe. A grafikus felület megjelenését, a használt ikonokat, stb screenshot-szerű képekkel kell bemutatni. Az építészetben ez a homlokzati terv.}

\diagram{img/graphics-plan/menu}{Menü terve}{14cm}
\diagram{img/graphics-plan/game}{Játék felület terve}{14cm}

\subsection{Vezérlés}
A program indításakor a főmenü nyílik meg. A főmenüből új játék indítani, illetve kilépni lehet. Új játék indításakor a játékablak nyílik meg.
\newline Itt az aktuálisan körön lévő játékost lehet irányítani. A térképen mozgatni a 'WASD' gombokkal lehet. Egy tárgyat felvenni úgy lehet, hogy a játékost ráléptetjük a csempére, amin van. Ajtóra lépés esetén a játékos megkísérel átlépni az ajtón. Falra nem lehet lépni. Más személy által elfoglalt csempére nem lehet lépni.
\newline A játékosoknál lévő tárgyak az oldalsó panelen láthatók. 'E' gomb megnyomásakor kiválaszthatunk egy tárgyat, hogy használjuk. 'Q' gombbal hasonlóan eldobunk egy tárgyat. 'L' gombbal összelinkeljük a nálunk lévő tranzisztorokat. Space megnyomásával befejezzük a körünket, a következő játékos jön.

\section{A grafikus rendszer architektúrája}
% \comment{A felület működésének elve, a grafikus rendszer architektúrája (struktúra diagramok). A struktúra diagramokon a prototípus azon és csak azon osztályainak is szerepelnie kell, amelyekhez a grafikus felületet létrehozó osztályok kapcsolódnak.}

\subsection{A felület működési elve}
% \comment{Le kell írni, hogy a grafikai megjelenésért felelős osztályok, objektumok hogyan kapcsolódnak a meglevő rendszerhez, a megjelenítés során mi volt az alapelv. Törekedni kell az MVC megvalósításra. Alapelvek lehetnek: \textbf{push} alapú: a modell értesíti a felületet, hogy változott; \textbf{pull} alapú: a felület kérdezi le a modellt, hogy változott-e; \textbf{kevert}: a kettő kombinációja.}
A játékunkhoz az MVC mintát választottuk. 3 részre bontjuk a programkódot. A Modellre, mely kezeli az alkalmazás logikai részét. A Nézetre, mely megjeleníti a különböző objektumokat. A Vezérlőre, mely a felhasználói bementeket kezeli. 
A modellt kevert módon implementáljuk.
Az MVC modellt a számos előnye miatt választottuk. Ilyen az egyidejű fejlesztés lehetősége, a rétegek függetlensége és a könnyen változtathatóság. A hátrányait a mintának is érdemes megjegyezni, de a gyengébb olvashatóság és a több boilerplate kód ellenére is megérte ezt a mintát választanunk.

\subsection{A felület osztály-struktúrája}
% \comment{Osztálydiagram. Minden új osztály, és azon régiek, akik az újakhoz közvetlenül kapcsolódnak.}
\diagram{img/diagrams/class/mvc-model}{Modell}{14cm}
\diagram{img/diagrams/class/mvc-view}{View}{14cm}
\diagram{img/diagrams/class/mvc-controller}{Controller}{14cm}

\section{A grafikus objektumok felsorolása}


\subsubsection{AIController}
\begin{class-template-responsibility}
    Az oktatók és takarítók irályítása
\end{class-template-responsibility}
\begin{class-template-interface}
    -
\end{class-template-interface}
\begin{class-template-baseclass}
    PersonController
\end{class-template-baseclass}
\begin{class-template-attribute}
    \classitem{entity : Entity}{Az oktatót/takarítót reprezentáló entitás.}
\end{class-template-attribute}
\begin{class-template-method}
    \classitem{turn() : void}{Az oktató/takarító van soron, random irányba megpróbál lépni.}
\end{class-template-method}

\subsubsection{DoorTile}
\begin{class-template-responsibility}
    Az ajtó reprezentálása. 
\end{class-template-responsibility}
\begin{class-template-interface}
    -
\end{class-template-interface}
\begin{class-template-baseclass}
    Tile
\end{class-template-baseclass}
\begin{class-template-attribute}
    \classitem{door : Door}{Az pálya abszrakt reprezentációjában hozzá tartozó ajtó.}
\end{class-template-attribute}
\begin{class-template-method}
    \classitem{stepOn(person : Person) : boolean}{Valaki megpróbál átlépni az ajtón.}
\end{class-template-method}

\subsubsection{DoorTileView}
\begin{class-template-responsibility}
    Egy ajtó kirajzolása.
\end{class-template-responsibility}
\begin{class-template-interface}
    Drawable
\end{class-template-interface}
\begin{class-template-baseclass}
    -
\end{class-template-baseclass}
\begin{class-template-attribute}
    \classitem{doorTile : DoorTile}{A nézethez tartozó DoorTile objektum.}
\end{class-template-attribute}
\begin{class-template-method}
    -
\end{class-template-method}

\subsubsection{Drawable}
\begin{class-template-responsibility}
    Interfész biztosítása a kirajzolható objektumoknak.
\end{class-template-responsibility}
\begin{class-template-method}
    \classitem{draw(graphics : Graphics) : void}{Az objektum kirajzolása az adott grafikus kontextusban.}
\end{class-template-method}

\subsubsection{Entity}
\begin{class-template-responsibility}
    A játékban szereplő embereknek ad egy wrapper-t, ami tárolja a pozícióját.
\end{class-template-responsibility}
\begin{class-template-interface}
    -
\end{class-template-interface}
\begin{class-template-baseclass}
    -
\end{class-template-baseclass}
\begin{class-template-attribute}
    \classitem{position : Position}{Az entitás x,y koordinátái világ térben}
\end{class-template-attribute}
\begin{class-template-method}
    \classitem{getPerson() : person}{Az entitás által reprezentált abszrakt modell beli ember lekérdezése.}
    \classitem{getPosition() : Position}{Az entitás helyzetének lekérdezése.}
\end{class-template-method}

\subsubsection{FloorTile}
\begin{class-template-responsibility}
    Szobák padlóit reprezentáló csempe.
\end{class-template-responsibility}
\begin{class-template-interface}
    -
\end{class-template-interface}
\begin{class-template-baseclass}
    Tile
\end{class-template-baseclass}
\begin{class-template-attribute}
    \classitem{item : Item}{A csempén található elejtett tárgy, ha van.}
    \classitem{person : Person}{A csempén álló személy, ha van.}
    \classitem{onChangeListeners : Listener 0..*}{Az inventory megváltozását figyelő eseménykezelők.}
\end{class-template-attribute}
\begin{class-template-method}
    \classitem{stepOn(person) : boolean}{Egy ember erre a mezőre lép. Igazat ad vissza, ha sikeres (nem állt ott más).}
    \classitem{removePerson(person): void}{Az ember elhagyta a mezőt.}
    \classitem{setItem(item): void}{A tárgy hozzáadása a cellához.}
    \classitem{addOnChangeEventListener(onChange) : void}{A Tile változásának figyelése}
\end{class-template-method}

\subsubsection{FloorTileView}
\begin{class-template-responsibility}
    Egy padlóelem kirajzolása.
\end{class-template-responsibility}
\begin{class-template-interface}
    Drawable
\end{class-template-interface}
\begin{class-template-baseclass}
    -
\end{class-template-baseclass}
\begin{class-template-attribute}
    \classitem{floorTile : FloorTile}{A mező, amit ez a nézet reprezentál.}
    \classitem{itemView : ItemView}{A cellán lévő tárgy nézete (ha van).}
    \classitem{personView : PersonView}{A cellán lévő személy nézete (ha van).}
\end{class-template-attribute}
\begin{class-template-method}
    \classitem{setItem(itemView : ItemView): void}{A mezőn lévő tárgy nézetének beállítása.}
    \classitem{setPerson(personView : PersonView): void}{A mezőn álló ember nézetének beállítása.}
\end{class-template-method}

\subsubsection{Game}
\begin{class-template-responsibility}
    A játékbeli pálya és játékosok tárolása.
\end{class-template-responsibility}
\begin{class-template-interface}
    -
\end{class-template-interface}
\begin{class-template-baseclass}
    -
\end{class-template-baseclass}
\begin{class-template-attribute}
    \classitem{tiles : Tile 0..*}{A játékteret alkotó mezők.}
    \classitem{rooms : Room 2..*}{A játékteret alkotó szobák.}
\end{class-template-attribute}
\begin{class-template-method}
    \classitem{}{}
\end{class-template-method}

\subsubsection{GameController}
\begin{class-template-responsibility}
    A játék központi vezérlője, ismeri a játékállapotot, a megjelenítőket, illetve az alacsonyabb szintű vezérlőket, és ezek kapcsolatát.
\end{class-template-responsibility}
\begin{class-template-interface}
    -
\end{class-template-interface}
\begin{class-template-baseclass}
    -
\end{class-template-baseclass}
\begin{class-template-attribute}
    \classitem{players : PlayerController 1..4}{Játékosok vezérlői}
    \classitem{npcs : AiController 1..*}{Oktatók és takarítók vezérlői}
    \classitem{modelViews : Map<Object, Drawable>}{A modellbeli objektumok és őket reprezentáló nézetek kapcsolatát tárolja.}
    \classitem{renderer : Renderer}{Kirajzolja a játékot.}
    \classitem{gamePanel : GamePanel}{A kirenderelt játékot megjelenító panel.}
\end{class-template-attribute}
\begin{class-template-method}
    \classitem{addModelView(object : Object, view : Drawable) : void}{Egy modell objektum és hozzátartozó nézet összerendelése.}
    \classitem{getModelView(object : Object) : Drawable}{Az objektum nézetének lekérdezése.}
    \classitem{addRoom(room : Room) : void}{Egy szoba hozzáadása a kontrollerhez.}
    \classitem{addRoomView(roomView) : void}{Egy szoba nézetének hozzáadása a kontrollerhez.}
    \classitem{setTile(tile : Tile, position : Position) : void}{A pályán a megadott pozícióban elhelyez egy cellát.}
    \classitem{getTile(position : Position) : Tile}{A megadott pozícióban lévő csempe lekérdezése.}
    \classitem{getCurrentPlayer() : void}{A soron következő játékos lekérdezése.}
    \classitem{addEntity(Entity) : void}{Entitás hozzáadása a kontrollerhez.}
    \classitem{addPlayerController(controller : PlayerController) : void}{Játékos-kontroller hozzáadása a kontrollerhez.}
    \classitem{addAiController(controller : AiController) : void}{Oktató / takarító kontrollerének regisztrálása.}
\end{class-template-method}

\subsubsection{GamePanel}
\begin{class-template-responsibility}
    A játékot megjelenítő panel.
\end{class-template-responsibility}
\begin{class-template-interface}
    -
\end{class-template-interface}
\begin{class-template-baseclass}
    JPanel
\end{class-template-baseclass}
\begin{class-template-attribute}
    \classitem{renderer : Renderer}{A nézeteket kirajzoló objektum.}
\end{class-template-attribute}
\begin{class-template-method}
    \classitem{repaint() : void}{A renderer által rajzolt képet jeleníti meg teljes méretében.}
\end{class-template-method}

\subsubsection{GameView}
\begin{class-template-responsibility}
    A pálya és a különböző felhasználói felületek kirajzolása.
\end{class-template-responsibility}
\begin{class-template-interface}
    Drawable
\end{class-template-interface}
\begin{class-template-baseclass}
    -
\end{class-template-baseclass}
\begin{class-template-attribute}
    \classitem{game : Game}{A játékállapot.}
    \classitem{mapView : MapView}{A pálya nézete}
    \classitem{sideBarView : SideBarView}{Az oldalsó panel nézete}
\end{class-template-attribute}
\begin{class-template-method}
    \classitem{draw(graphics) : void}{Kirajzolás.   }
\end{class-template-method}

\subsubsection{InventoryController}
\begin{class-template-responsibility}
    Egy játékos tárgyainak nyilvántartása, és a változtatások kezelése.
\end{class-template-responsibility}
\begin{class-template-interface}
    -
\end{class-template-interface}
\begin{class-template-baseclass}
    -
\end{class-template-baseclass}
\begin{class-template-attribute}
    \classitem{inventoryView : InventoryView}{Az inventory nézete}
\end{class-template-attribute}
\begin{class-template-method}
    \classitem{onInventoryChanged(inventory) : void}{Az Inventory változásának kezelése.}
\end{class-template-method}

\subsubsection{InventoryView}
\begin{class-template-responsibility}
    Egy játékos tárgyainak megjelenítése.
\end{class-template-responsibility}
\begin{class-template-interface}
    Drawable
\end{class-template-interface}
\begin{class-template-baseclass}
    -
\end{class-template-baseclass}
\begin{class-template-attribute}
    -
\end{class-template-attribute}
\begin{class-template-method}
    \classitem{clearItems() : void}{Tárgynézetek törlése.}
    \classitem{addItem(itemView) : void}{Egy új tárgy megjelenítése az inventoryban.}
\end{class-template-method}

\subsubsection{ItemView}
\begin{class-template-responsibility}
    Egy tárgy kirajzolása.
\end{class-template-responsibility}
\begin{class-template-interface}
    Drawable
\end{class-template-interface}
\begin{class-template-baseclass}
    -
\end{class-template-baseclass}
\begin{class-template-attribute}
    \classitem{item : Item}{A tárgy, amihez a nézet tartozik.}
    \classitem{sprite : Sprite}{A tárgyat megjelenítő kép.}
\end{class-template-attribute}
\begin{class-template-method}
    \classitem{}{}
\end{class-template-method}

\subsubsection{KeyboardEventListener}
\begin{class-template-responsibility}
    A billentyűzet felügyelete.
\end{class-template-responsibility}
\begin{class-template-interface}
    -
\end{class-template-interface}
\begin{class-template-baseclass}
    -
\end{class-template-baseclass}
\begin{class-template-attribute}
    -
\end{class-template-attribute}
\begin{class-template-method}
    \classitem{keyPressed(key) : void}{A megfelelő esemény végrehajtása a billentyű lenyomásának hatására.}
\end{class-template-method}

\subsubsection{MapView}
\begin{class-template-responsibility}
    A pálya kirajzolása.
\end{class-template-responsibility}
\begin{class-template-interface}
    Drawable
\end{class-template-interface}
\begin{class-template-baseclass}
    -
\end{class-template-baseclass}
\begin{class-template-attribute}
    \classitem{roomViews : RoomView 0..*}{A pályát alkotó szobák nézetei.}
\end{class-template-attribute}
\begin{class-template-method}
    \classitem{draw(graphics : Graphics) : void}{Kirajzolás}
\end{class-template-method}

\subsubsection{PersonController}
\begin{class-template-responsibility}
    Ősosztály biztosítása a PlayerControllernek és az AIControllernek.
\end{class-template-responsibility}
\begin{class-template-interface}
    -
\end{class-template-interface}
\begin{class-template-baseclass}
    -
\end{class-template-baseclass}
\begin{class-template-attribute}
-
\end{class-template-attribute}
\begin{class-template-method}
-
\end{class-template-method}

\subsubsection{PersonView}
\begin{class-template-responsibility}
    Egy személy kirajzolása.
\end{class-template-responsibility}
\begin{class-template-interface}
    Drawable
\end{class-template-interface}
\begin{class-template-baseclass}
    -
\end{class-template-baseclass}
\begin{class-template-attribute}
    \classitem{person : Person}{A nézethez tartozó személy objektum.}
    \classitem{sprite : Sprite}{Az embernek megfelelő kép, amivel kirajzolódik.}
\end{class-template-attribute}
\begin{class-template-method}
-
\end{class-template-method}

\subsubsection{PlayerController}
\begin{class-template-responsibility}
    A játékos irányítása.
\end{class-template-responsibility}
\begin{class-template-interface}
    -
\end{class-template-interface}
\begin{class-template-baseclass}
    PersonController
\end{class-template-baseclass}
\begin{class-template-attribute}
    \classitem{inventoryController : InventoryController}{A játékos eszköztárának vezérlője}
\end{class-template-attribute}
\begin{class-template-method}
    \classitem{move(direction) : void}{A játékos elmozdítása a megfelelő irányba.}
\end{class-template-method}

\subsubsection{PlayerView}
\begin{class-template-responsibility}
    Egy játékos kirajzolása az oldalsó panelre.
\end{class-template-responsibility}
\begin{class-template-interface}
    Drawable
\end{class-template-interface}
\begin{class-template-baseclass}
    -
\end{class-template-baseclass}
\begin{class-template-attribute}
    \classitem{inventoryView : InventoryView}{A játékos inventoryjának nézete.}
    \classitem{personView : PersonView}{A játékoshoz tartozó kinézet.}
\end{class-template-attribute}
\begin{class-template-method}
    \classitem{}{}
\end{class-template-method}

\subsubsection{RoomView}
\begin{class-template-responsibility}
    A pálya kirajzolása.
\end{class-template-responsibility}
\begin{class-template-interface}
    Drawable
\end{class-template-interface}
\begin{class-template-baseclass}
    -
\end{class-template-baseclass}
\begin{class-template-attribute}
    \classitem{tiles : TileView 0..*}{A szobát alkotó mezők nézetei.}
    \classitem{room : Room}{A nézethez tartozó modellbeli objektum.}
\end{class-template-attribute}
\begin{class-template-method}
    \classitem{addTile(tileView) : void}{Egy új cella hozzáadása a szobához.}
\end{class-template-method}

\subsubsection{SideBarView}
\begin{class-template-responsibility}
    Az oldalsó panel kirajzolása.
\end{class-template-responsibility}
\begin{class-template-interface}
    Drawable
\end{class-template-interface}
\begin{class-template-baseclass}
    -
\end{class-template-baseclass}
\begin{class-template-attribute}
    -
\end{class-template-attribute}
\begin{class-template-method}
     \classitem{draw(graphics Graphics) : void}{Kirajzolás}
\end{class-template-method}

\subsubsection{Tile}
\begin{class-template-responsibility}
    Absztrakt osztály biztosítása a csempék számára. Az absztrakt osztály megvalósítja a csempe helyzete számontartásához szükséges funkciót.
\end{class-template-responsibility}
\begin{class-template-interface}
    -
\end{class-template-interface}
\begin{class-template-baseclass}
    -
\end{class-template-baseclass}
\begin{class-template-attribute}
    \classitem{room : Room}{A szoba, amihez a mező tartozik.}
    \classitem{onChangeListeners : Listener 0..*}{A mezőt figyelő eseménykezelők.}
\end{class-template-attribute}
\begin{class-template-method}
    \classitem{stepOn(person) : boolean}{A személy megpróbál rálépni a csempére. Ha már van rajta személy, akkor ez nem sikerül. A visszatérés a csempe, amin ezután lesz.}
    \classitem{addOnChangeEventListener(onChange) : void}{A Tile változásának figyelése}
\end{class-template-method}

\subsubsection{TileController}
\begin{class-template-responsibility}
    A játék csempéinek a kezelése.
\end{class-template-responsibility}
\begin{class-template-interface}
    -
\end{class-template-interface}
\begin{class-template-baseclass}
    -
\end{class-template-baseclass}
\begin{class-template-attribute}
    \classitem{tile : Tile}{A hozzátartozó cella.}
    \classitem{tileView : TileView}{A hozzátartozó cella nézete.}
\end{class-template-attribute}
\begin{class-template-method}
    \classitem{onTileChanged() : void}{A csempe változásának a kezelése.}
\end{class-template-method}

\subsubsection{TileView}
\begin{class-template-responsibility}
    Egy mező kirajzolása.
\end{class-template-responsibility}
\begin{class-template-interface}
    Drawable
\end{class-template-interface}
\begin{class-template-baseclass}
    -
\end{class-template-baseclass}
\begin{class-template-attribute}
    \classitem{tile : Tile}{A mező, amit ez a nézet reprezentál.}
\end{class-template-attribute}
\begin{class-template-method}
    -
\end{class-template-method}

\subsubsection{TransistorView}
\begin{class-template-responsibility}
    Egy tranzisztor kirajzolása. Azért van ennél az egy tárgynál külön szükség rá, mivel míg a többi tárgynál elég a sprite-ot létrehozásnál megadni, addig a tranzisztor állapotától függően az változhat.
\end{class-template-responsibility}
\begin{class-template-interface}
    Drawable
\end{class-template-interface}
\begin{class-template-baseclass}
    ItemView
\end{class-template-baseclass}
\begin{class-template-attribute}
    \classitem{transistor : Transistor}{A tranzisztor, amit megjelenít.}
\end{class-template-attribute}
\begin{class-template-method}
    \classitem{draw(graphics : Graphics) : void}{Kirajzolja a tranzisztort, megjelenítve azt is, hogy aktív-e.}
\end{class-template-method}

\subsubsection{WallTile}
\begin{class-template-responsibility}
    Vizuálisan is megjeleníthető pályahatárok biztosítása.
\end{class-template-responsibility}
\begin{class-template-interface}
    -
\end{class-template-interface}
\begin{class-template-baseclass}
    Tile
\end{class-template-baseclass}
\begin{class-template-attribute}
    -
\end{class-template-attribute}
\begin{class-template-method}
    \classitem{stepOn(person) : boolean}{Hamisat ad vissza, mivel falra nem lehet lépni.}
\end{class-template-method}

\subsubsection{WallTileView}
\begin{class-template-responsibility}
    Egy faldarab kirajzolása.
\end{class-template-responsibility}
\begin{class-template-interface}
    Drawable
\end{class-template-interface}
\begin{class-template-baseclass}
    -
\end{class-template-baseclass}
\begin{class-template-attribute}
-
\end{class-template-attribute}
\begin{class-template-method}
-
\end{class-template-method}

\subsubsection{ObjectFactory}
\begin{class-template-responsibility}
    Factory mintát követő objektum gyártás
\end{class-template-responsibility}
\begin{class-template-interface}
    -
\end{class-template-interface}
\begin{class-template-baseclass}
    -
\end{class-template-baseclass}
\begin{class-template-attribute}
    -
\end{class-template-attribute}
\begin{class-template-method}
    \classitem{createRoom(int capacity, position  widht, height) : Room}{Létrehoz  egy szobát és a benne lévő csempéket: Falcsempék a szélére, padlócsempék a többi helyre. Modellbeli és view-beli objektumokat is gyárt}
    \classitem{createDoor(Room fromRoom, Room toRoom, position fromPos, position toPos) : Door}{ajtó gyártása: modellbeli ajtó + modellbeli és view-beli csempék.}
    \classitem{createPlayer(position position) : Student}{játékos gyártása: Student, hozzátartozó entity és view}
    \classitem{createNPC(position position) : Teacher}{NPC gyártása: Teacher, hozzátartozó entity és view}
    \classitem{createNPCJanitor(position position) : Janitor}{NPC gyártása: Janitor, hozzátartozó entity és view}
    \classitem{createItemSlideRule(position position) : SlideRule}{Item + view gyártása}
    \classitem{createItemBeer(position position) : Beer}{Item + view gyártása}
    \classitem{createItemWetRag(position position) : WetRag}{Item + view  gyártása}
    \classitem{createItemCode(position position) : CodeOfStudies}{Item + view gyártása}
    \classitem{createItemMask(position position) : Mask}{Item + view  gyártása}
    \classitem{createItemCamembert(position position) : Camambert}{Item + view gyártása}
    \classitem{createItemAirFreshener(position position) : AirFreshener}{Item + view gyártása}
    \classitem{createItemTransistor(position position) : Transistor}{Item + view gyártása + még egy item gyártása}
\end{class-template-method}


\subsection{Megváltozott osztályok}
Az alapelvünk az átalakításnál az volt, hogy a lehető legkevesebb dolgott kelljen megváltoztatni a a már meglévő osztálystruktúrában, ezért ahol lehetett inkább új osztályokba csomagoltuk a meglévő modell objektumait.

\subsubsection{Inventory}
\begin{class-template-responsibility}
    Az entitások által birtokolt tárgyak számontartása a modellben.
\end{class-template-responsibility}
\begin{class-template-interface}
    Updatable
\end{class-template-interface}
\begin{class-template-baseclass}
    -
\end{class-template-baseclass}
\begin{class-template-attribute}
    \classitem{onChangeListeners : Listener 0..*}{Az inventory megváltozását figyelő eseménykezelők.}
\end{class-template-attribute}
\begin{class-template-method}
    \classitem{addOnChangeEventLitener(listener : Listener) : void}{Listener feliratkoztatása az inventory változás event-ére.}
    \classitem{onChanged() : void}{Értesíti a feliratkozott eseménykezelőket a változásról.}
    \classitem{add(item : Item) : boolean}{A korábbi működésen túl meghívja az \textit{onChanged()} metódust.}
    \classitem{remove(item : Item) : void}{A korábbi működésen túl meghívja az \textit{onChanged()} metódust.}
\end{class-template-method}

\section{Kapcsolat az alkalmazói rendszerrel}
% \comment{Szekvencia-diagramokon ábrázolni kell a grafikus rendszer működését. Konzisztens kell legyen az előző alfejezetekkel. Minden metódus, ami ott szerepel, fel kell tűnjön valamelyik szekvenciában. Minden metódusnak, ami szekvenciában szerepel, szereplnie kell a valamelyik osztálydiagramon.}

\diagram{img/diagrams/graphics-sequence/create-door}{Ajtó létrehozása}{14cm}
\diagram{img/diagrams/graphics-sequence/create-item}{Tárgy létrehozása}{14cm}
\diagram{img/diagrams/graphics-sequence/create-npc}{NPC létrehozása}{14cm}
\diagram{img/diagrams/graphics-sequence/create-player}{Játékos létrehozása}{14cm}
\diagram{img/diagrams/graphics-sequence/create-room}{Szoba létrehozása}{14cm}
\diagram{img/diagrams/graphics-sequence/drop-item}{Tárgy eldobása}{14cm}
\diagram{img/diagrams/graphics-sequence/entity-movement}{Mozgás}{14cm}
\diagram{img/diagrams/graphics-sequence/inventory-changed}{Inventory megváltozása}{14cm}
\diagram{img/diagrams/graphics-sequence/move-player}{Játékos mozgatása}{14cm}
\diagram{img/diagrams/graphics-sequence/render-game-view}{Pálya kirajzolása}{14cm}
\diagram{img/diagrams/graphics-sequence/render-game}{Játék kirajzolása}{14cm}
\diagram{img/diagrams/graphics-sequence/render-side-panel}{Oldalsó panel kirajzolása}{14cm}
\diagram{img/diagrams/graphics-sequence/renderer}{Renderer működése}{14cm}
