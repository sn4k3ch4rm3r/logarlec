\chapter{Analízis modell kidolgozása 2.}

\section{Objektum katalógus}

\subsection{Hallgató}
A játékos irányítja. Szobáról szobára haladnak ajtókon keresztül. Tudják, melyik szobában vannak, milyen tárgyak vannak náluk, és tudnak velük interaktálni.

\subsection{Oktató}
Szobáról szobára haladnak. Ha egy szobába kerülnek egy hallgatóval, kiejtik a játékból. 

\subsection{Logarléc}
A játék célja ezt megtalálni. Amint egy hallgató felveszi, a játék véget ér.

\subsection{Tranzisztor}
Teleportálásra lehet használni. Össze lehet kapcsonli egy másik tranzisztorral, ami letevés után tudja, melyik szobában van. A kézben lévő tranzisztor a használat után deaktiválódik. Tudja, hogy mely szobákat kapcsolja össze.

\subsection{TVSZ denevérbőrön}
Védelmet nyút az oktatókkal szemben, oktatónként használódik. Tudja, hogy hány használat van hátra.

\subsection{Szent söröspohár}
Aktiválás után 5 másodpercig nyújt védelmet az oktatókkal szemben, aztán  elveszti a képességét. Tudja, hogy még mennyi idő van hátra a hatásából.

\subsection{Nedves táblatörlő rongy}
A vele egy szobában lévő oktatókat megbénítja 15 másodpercre.
Tudja, hogy  mennyi idő van hátra a hatásának időtartamából.

\subsection{FFP2-es maszk}
A mérgesgáz ellen véd. Tudja, hogy hány használata van hátra, és hogy az adott használatból mennyi van hátra.

\subsection{Dobozolt káposztás Camambert}
Felbontáskor mézgező gázt bocsát ki. Használatkor a birtokló játékos elmondja a szobájának, hogy legyen adott ideig mérgező.

\subsection{Szoba}
Egy szoba a játékban. Ő birtokolja a tárgyakat, oktatókat, hallgatókat, amik megtalálhatók benne, illetve az ajtókat, amik belőle nyílnak. Tudja, hogy milyen különleges képessége van. Ismeri az ajtóit, hogy hány ember van benne

\subsection{Ajtó}
Tudja melyik két szoba között megy, azt, hogy egy- vagy kétirányú. Ha egyirányú, az irányt.

\section{Osztályok leírása}

\subsection{Beer}
\begin{class-template-responsibility}
    A játékbeli tárgy reprezentálása a modellben.
\end{class-template-responsibility}
\begin{class-template-interface}
    -
\end{class-template-interface}
\begin{class-template-baseclass}
    Item
\end{class-template-baseclass}
\begin{class-template-attribute}
    -
\end{class-template-attribute}
\begin{class-template-method}
    \classitem{use() : void}{Egy BeerEffectet rak a személyre, akinél van, majd önmegsemmisít.}
\end{class-template-method}

\subsection{BeerEffect}
\begin{class-template-responsibility}
    A játékbeli effektus reprezentálása a modellben.
\end{class-template-responsibility}
\begin{class-template-interface}
    -
\end{class-template-interface}
\begin{class-template-baseclass}
    Effect
\end{class-template-baseclass}
\begin{class-template-attribute}
    -
\end{class-template-attribute}
\begin{class-template-method}
    -
\end{class-template-method}

\subsection{Camembert}
\begin{class-template-responsibility}
    A játékbeli tárgy reprezentálása a modellben.
\end{class-template-responsibility}
\begin{class-template-interface}
    -
\end{class-template-interface}
\begin{class-template-baseclass}
    Item
\end{class-template-baseclass}
\begin{class-template-attribute}
    -
\end{class-template-attribute}
\begin{class-template-method}
    \classitem{use() : void}{Egy GasEffectet rak a szobára, ahol van, majd önmegsemmisít.}
\end{class-template-method}

\subsection{CodeOfStudies}
\begin{class-template-responsibility}
    A játékbeli tárgy reprezentálása a modellben.
\end{class-template-responsibility}
\begin{class-template-interface}
    -
\end{class-template-interface}
\begin{class-template-baseclass}
    Item
\end{class-template-baseclass}
\begin{class-template-attribute}
      \classitem{- uses : int}{Megadja, hogy hányszor használható még a tárgy.}
\end{class-template-attribute}
\begin{class-template-method}
    \classitem{useAgainst(target : Teacher) : void}{A személyen, akinél van, meghívja a protectFromTeacher(target) metódust, és csökkenti a uses attribútumot.}
\end{class-template-method}

\subsection{Door}
\begin{class-template-responsibility}
    Tudja, melyik két szobát köti össze.
\end{class-template-responsibility}
\begin{class-template-interface}
    -
\end{class-template-interface}
\begin{class-template-baseclass}
    -
\end{class-template-baseclass}
\begin{class-template-attribute}
    -
\end{class-template-attribute}
\begin{class-template-method}
    \classitem{use(person : Person, from : Room) : void} {A személy át kíván lépni a másik szobába. Ha abban az irányban járható az ajtó, a szobák értesülnek az átlépésről. Különben nem történik semmi.}
    \classitem{move(from : Room, to : Room) : void} {Az ajtó áthelyeződik két másik szoba közé.}
    \classitem{hide() : void}{Az ajtó eltűnik.}
    \classitem{show() : void}{Az ajtó megjelenik.}
\end{class-template-method}

\subsection{Effect}
\begin{class-template-responsibility}
    Ősosztály biztosítása a játékban található effektusoknak.
\end{class-template-responsibility}
\begin{class-template-interface}
    Updatable
\end{class-template-interface}
\begin{class-template-baseclass}
    -
\end{class-template-baseclass}
\begin{class-template-attribute}
    \classitem{- timeRemaining : double}{A hatásból hátralévő idő.}
\end{class-template-attribute}
\begin{class-template-method}
    \classitem{applyToStudent(target : Student) : void}{Az effektus rátétele egy hallgatóra.}
    \classitem{applyToTeacher(target : Teacher) : void}{Az effektus rátétele egy oktatóra.}
\end{class-template-method}

\subsection{GasEffect}
\begin{class-template-responsibility}
    A játékbeli effektus reprezentálása a modellben.
\end{class-template-responsibility}
\begin{class-template-interface}
    -
\end{class-template-interface}
\begin{class-template-baseclass}
    Effect
\end{class-template-baseclass}
\begin{class-template-attribute}
    -
\end{class-template-attribute}
\begin{class-template-method}
    -
\end{class-template-method}

\subsection{GameObject}
\begin{class-template-responsibility}
    Alaposztály biztosítása a játék "szereplőinek".
\end{class-template-responsibility}
\begin{class-template-interface}
    Updatable
\end{class-template-interface}
\begin{class-template-baseclass}
    -
\end{class-template-baseclass}
\begin{class-template-attribute}
    -
\end{class-template-attribute}
\begin{class-template-method}
    \classitem{addItem(item : Item) : void}{Egy tárgy hozzáadása az objektumhoz.}
    \classitem{removeItem(item : Item) : void}{Egy tárgy elvétele az objektumtól.}
    \classitem{applyEffect(effect : Effect) : void}{Effect hatásának alkalmazása az objektumra}
    \classitem{addEffect(effect : Effect) : void}{Effect elhelyezése objektumon}
    \classitem{removeEffect(effect : Effect) : void}{Egy effektus levétele az objektumról.}
    \classitem{interactTeacher(teacher : Teacher) : void}{Az objektum interakciója egy oktatóval.}
\end{class-template-method}

\subsection{Inventory}
\begin{class-template-responsibility}
    Tárgyak tárolását teszi lehetővé Person objektumok számára.
\end{class-template-responsibility}
\begin{class-template-interface}
    Updatable
\end{class-template-interface}
\begin{class-template-baseclass}
    -
\end{class-template-baseclass}
\begin{class-template-attribute}
    \classitem{- itemCount: int}{Hány tárgy van benne.}
\end{class-template-attribute}
\begin{class-template-method}
    \classitem{add(item : Item) : bool}{Tárgy hozzáadása az inventoryhoz.}
    \classitem{remove(item : Item) : void}{Tárgy eltávolítása az inventoryból.}
    \classitem{setRoom(room : Room) : void}{Az összes tárgy értesítése a szoba váltásról.}
    \classitem{protectFrom(teacher : Teacher) : void}{Az inventory-ban tárolt tárgyakkal megpróbál védekezni oktató ellen.}
\end{class-template-method}

\subsection{Item}
\begin{class-template-responsibility}
    Ősosztály biztosítása, valamint metódusok deklarációja a különböző tárgykat reprezentáló osztályok számára.
\end{class-template-responsibility}
\begin{class-template-interface}
    Updatable
\end{class-template-interface}
\begin{class-template-baseclass}
    -
\end{class-template-baseclass}
\begin{class-template-attribute}
    -
\end{class-template-attribute}
\begin{class-template-method}
    \classitem{use() : void}{Az ember használja a tárgyat.}
    \classitem{useAgainst(target : Teacher) : void}{A tárgyat használják egy oktató ellen.}
    \classitem{usePassive() : bool}{A tárgy használódik.}
    \classitem{useItem(item : Item) : void}{Tárgy használata egy másik tárggyal.}
    \classitem{link(other : Transistor) : void}{A tárgy összekapcsolása egy tranzisztorral.}
    \classitem{drop() : void}{A tárgy ledobása a földre.}
    \classitem{setRoom(room : Room) : void}{A tárgy tartózkodási helyének beállítása.}
    \classitem{setPerson(person : Person) : void}{A tárgy tulajdonosának beállítása.}
\end{class-template-method}

\subsection{Mask}
\begin{class-template-responsibility}
    A játékbeli tárgy reprezentálása a modellben.
\end{class-template-responsibility}
\begin{class-template-interface}
    -
\end{class-template-interface}
\begin{class-template-baseclass}
    Item
\end{class-template-baseclass}
\begin{class-template-attribute}
    \classitem{- uses : int}{Megadja, hogy hányszor használható még a tárgy.}
\end{class-template-attribute}
\begin{class-template-method}
    \classitem{use() : void}{Egy MaskEffectet rak a személyre, akinél van.}
\end{class-template-method}

\subsection{MaskEffect}
\begin{class-template-responsibility}
    A játékbeli effektus reprezentálása a modellben.
\end{class-template-responsibility}
\begin{class-template-interface}
    -
\end{class-template-interface}
\begin{class-template-baseclass}
    Effect
\end{class-template-baseclass}
\begin{class-template-attribute}
    -
\end{class-template-attribute}
\begin{class-template-method}
    -
\end{class-template-method}

\subsection{Person}
\begin{class-template-responsibility}
    Alaposztály szolgáltatása a mozgó játékbeli entitások számára.
\end{class-template-responsibility}
\begin{class-template-interface}
    -
\end{class-template-interface}
\begin{class-template-baseclass}
    GameObject
\end{class-template-baseclass}
\begin{class-template-attribute}
    \classitem{- name : String}{Az ember neve.}
    \classitem{- knockOutTime : double}{Az elájulás időtartama.}
\end{class-template-attribute}
\begin{class-template-method}
    \classitem{protectFromTeacher(target : Teacher) : void}{Egy személyt valami megment egy oktatótól.}
    \classitem{enterRoom(Room room) : void}{Az ember belép egy szobába.}
    \classitem{dropItem(item : Item) : void}{Az ember ledobja egy tárgyát a földre.}
    \classitem{setKnockOut() : void}{Az elájulás időtartamának beállítása.}
\end{class-template-method}

\subsection{RagEffect}
\begin{class-template-responsibility}
    A játékbeli effektus reprezentálása a modellben.
\end{class-template-responsibility}
\begin{class-template-interface}
    -
\end{class-template-interface}
\begin{class-template-baseclass}
    Effect
\end{class-template-baseclass}
\begin{class-template-attribute}
    -
\end{class-template-attribute}
\begin{class-template-method}
    -
\end{class-template-method}

\subsection{Room}
\begin{class-template-responsibility}
    Ő birtokolja a benne lévő Door, Person és (nem Personnál lévő) Item objektumokat.
    Ő engedélyezi vagy tagadja meg a Personok belépését.
    Ő értesíti a benne lévő Personokat új Person belépéséről, mérges gázról, táblatörlő rongy használatáról. 
\end{class-template-responsibility}
\begin{class-template-interface}
    -
\end{class-template-interface}
\begin{class-template-baseclass}
    GameObject
\end{class-template-baseclass}
\begin{class-template-attribute}
     \classitem{- capacity : int}{Megadja, hogy maximum hány ember fér el a szobában.}
    \classitem{- numOfPeople : int}{A szobába jelenleg tartózkodó emberek száma.}
\end{class-template-attribute}
\begin{class-template-method}
    \classitem{enter(Person person) : bool} { A személy belép a szobába és a többi személy értesül róla, és reagál, ha kell.}
    \classitem{leave(Person person) : void} { A szobát elhagyna egy személy, a szobában lévő emberek száma csökken.}
    \classitem{merge(Room room) : void} { Két szoba összeolvad.}
    \classitem{split() : void} {A szoba a szabályoknak megfelelően osztódik.}
    \classitem{moveContents(Room room) : void} {A szobában található dolgokat átrakja egy másikba.}
    \classitem{addDoor(Room room) : void} {Ajtó adódik a szobához.}
    \classitem{removeDoor(Room room) : void} {Ajtó eltávolítása a szobából.}
    \classitem{hideDoors() : void} {Az eltűnő ajtók eltűnnek.}
    \classitem{showDoors() : void} {Az eltűnt ajtók megjelennek.}
\end{class-template-method}

\subsection{SlideRule}
\begin{class-template-responsibility}
    A játékbeli tárgy reprezentálása a modellben.
\end{class-template-responsibility}
\begin{class-template-interface}
    -
\end{class-template-interface}
\begin{class-template-baseclass}
    Item
\end{class-template-baseclass}
\begin{class-template-attribute}
    -
\end{class-template-attribute}
\begin{class-template-method}
    -
\end{class-template-method}

\subsection{Student}
\begin{class-template-responsibility}
    A játékos reprezentálása a modellben.
\end{class-template-responsibility}
\begin{class-template-interface}
    -
\end{class-template-interface}
\begin{class-template-baseclass}
    GameObject \baseclass Person
\end{class-template-baseclass}
\begin{class-template-attribute}
    \classitem{- eliminated : bool}{Kiesett-e a játékból.}
\end{class-template-attribute}
\begin{class-template-method}
    \classitem{setEliminated(value : bool) : void}{A kiesés beállítása.}
\end{class-template-method}

\subsection{Teacher}
\begin{class-template-responsibility}
    Az ellenségek reprezentálása a modellben.
\end{class-template-responsibility}
\begin{class-template-interface}
    -
\end{class-template-interface}
\begin{class-template-baseclass}
    GameObject \baseclass Person
\end{class-template-baseclass}
\begin{class-template-attribute}
    \classitem{- peaceful : bool}{Békés-e az oktató (támad-e hallgatót).}
\end{class-template-attribute}
\begin{class-template-method}
    \classitem{setPeaceful(value : bool) : void}{A békésség attribútum beállítása.}
\end{class-template-method}

\subsection{Transistor}
\begin{class-template-responsibility}
    A játékbeli tárgy reprezentálása a modellben.
\end{class-template-responsibility}
\begin{class-template-interface}
    -
\end{class-template-interface}
\begin{class-template-baseclass}
    Item
\end{class-template-baseclass}
\begin{class-template-attribute}
    -
\end{class-template-attribute}
\begin{class-template-method}
    \classitem{setTarget(target : Room) : void}{A tranzisztor-teleportáció célszobájának beállítása}
    \classitem{setPair(pair : Transistor) : void}{A tranzisztor párjának beállítása.}
\end{class-template-method}

\subsection{Updatable}
\begin{class-template-responsibility}
    Interfész nyújtása minden olyan osztálynak, melynek az idő múlása számít,
\end{class-template-responsibility}
\begin{class-template-method}
    \classitem{update(deltaTime : double) : void}{Az eltelt idő jelzése.}
\end{class-template-method}

\subsection{WetRag}
\begin{class-template-responsibility}
    A játékbeli tárgy reprezentálása a modellben.
\end{class-template-responsibility}
\begin{class-template-interface}
    -
\end{class-template-interface}
\begin{class-template-baseclass}
    Item
\end{class-template-baseclass}
\begin{class-template-attribute}
    -
\end{class-template-attribute}
\begin{class-template-method}
    \classitem{setRoom(room : Room) : void}{Amikor egy ilyen objektum vált szobát, az effektje vele együtt átvándorol a másik szobába.}
\end{class-template-method}
\clearpage
\section{Statikus struktúra diagramok}

\diagram{img/diagrams/class/game-objects}{A játékban szereplő objektumok}{14cm}
\diagram{img/diagrams/class/items}{A játékban szereplő tárgyak}{14cm}
\diagram{img/diagrams/class/effects}{A játékban szereplő hatások}{14cm}
\clearpage
\section{Szekvencia diagramok}

\diagram{img/diagrams/sequence/door-move}{Ajtó mozgatás}{14cm}
\diagram{img/diagrams/sequence/door-use}{Ajtó használat}{14cm}
\diagram{img/diagrams/sequence/effect-beer-apply}{Szent Söröspohár védelem}{14cm}
\diagram{img/diagrams/sequence/effect-gas-apply}{Mérgezés}{14cm}
\diagram{img/diagrams/sequence/effect-mask-apply}{Maszk védelem}{14cm}
\diagram{img/diagrams/sequence/effect-rag-apply}{Oktató elkábítása}{14cm}
\diagram{img/diagrams/sequence/effect-update}{Hatás frissítése}{14cm}
\diagram{img/diagrams/sequence/game-object-add-effect}{Hatás hozzáadása a Game Object-hez}{14cm}
\diagram{img/diagrams/sequence/inventory-set-room}{title}{14cm}
\diagram{img/diagrams/sequence/item-beer-use}{Szent Söröspohár használat}{14cm}
\diagram{img/diagrams/sequence/item-camembert-use}{Camembert használat}{14cm}
\diagram{img/diagrams/sequence/item-code-of-studies-use}{TVSZ használat}{14cm}
\diagram{img/diagrams/sequence/item-mask-use}{Maszk használat}{14cm}
\diagram{img/diagrams/sequence/item-wet-rag-set-room}{Nedes Táblatörlő rongy hatása}{14cm}
\diagram{img/diagrams/sequence/person-add-item}{Tárgy hozzáadása emberhez}{14cm}
\diagram{img/diagrams/sequence/person-drop-item}{Tárgy eldobása}{14cm}
\diagram{img/diagrams/sequence/person-enter-room}{Ember belép a szobába}{14cm}
\diagram{img/diagrams/sequence/person-update}{Ember frissítése}{14cm}
\diagram{img/diagrams/sequence/room-hide-doors}{Atjó elrejtése}{14cm}
\diagram{img/diagrams/sequence/room-merge}{Szobák egyeülése}{14cm}
\diagram{img/diagrams/sequence/room-show-doors}{Ajtó megjelenése}{14cm}
\diagram{img/diagrams/sequence/room-split}{Szoba osztódása}{14cm}
\diagram{img/diagrams/sequence/room-update}{Szoba frissítése}{14cm}
\diagram{img/diagrams/sequence/student-apply-effect}{Hatás a hallgatón}{14cm}
\diagram{img/diagrams/sequence/student-interact-teacher}{Léleklopás}{14cm}
\diagram{img/diagrams/sequence/teacher-apply-effect}{Hatás az oktatón}{14cm}
\diagram{img/diagrams/sequence/teacher-update}{Oktató frissítése}{14cm}
\diagram{img/diagrams/sequence/transistor-link}{Tranzisztorok összekapcsolása}{14cm}
\diagram{img/diagrams/sequence/transistor-use}{Tranzisztorok használata}{14cm}
\clearpage

\section{State-chartok}

\diagram{img/diagrams/state/door}{Ajtó állapota}{14cm}
\diagram{img/diagrams/state/person}{Ember állapota}{14cm}
\diagram{img/diagrams/state/teacher}{Oktató állapota}{14cm}
\diagram{img/diagrams/state/room}{Szoba állapota}{14cm}
\diagram{img/diagrams/state/transistor}{Tranzisztor állapota}{14cm}
